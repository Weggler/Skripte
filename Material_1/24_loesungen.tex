\section*{Differentiation L�sungen}\label{MA1-24-Aufgaben-Loesungen}

\ifdefined\MOODLE 
	Regeln und Beispiele, die zum Verst�ndnis der L�sungswege hilfreich sind, finden Sie in \ref{MA1-24}.
\else
	Regeln und Beispiele, die zum Verst�ndnis der L�sungswege hilfreich sind, finden Sie im Skript.
\fi 

\begin{exercisebox}[Differenzierbarkeit]
  \begin{itemize}
	  \item[(a)] Untersuchen Sie, ob $f$ an der Stelle  $x_0 = 1$ differenzierbar ist:
		  \[ f:\mathbb R \to \mathbb R, x\mapsto  
			  \begin{cases} 1-x^2, \; x \leq 1\,, \\
    x^2 -1, \; x > 1\,.\end{cases} 
      \]
      \item[(b)] Bestimmen Sie die Parameter $a,b\in\mathbb R$ so, dass $g$ auf ganz $\mathbb R$ differenzierbar ist: 
		  \[ g:\mathbb R \to \mathbb R, x\mapsto  
			  \begin{cases} a\cdot x^2, \; x \geq 1\,, \\
    x +b, \; x < 1\,.\end{cases} 
      \]
  \end{itemize}

\noindent{\bf L�sung:}
\begin{itemize}
	\item[(a)] $f$ ist an der Stelle $x_0 = 1$ stetig aber nicht differenzierbar, denn 
		\[ 
		\lim\limits_{x\downarrow 1} f(x) = \lim\limits_{x\downarrow 1} x^2-1 = 0 = f(1), \quad \lim\limits_{x\downarrow 1} f'(x) = \lim\limits_{x\downarrow 1} 2\cdot x = 2 \not= f'(1) = -2 \,.   \]
	\item[(b)] $g$ ist f�r $a=1/2 \land b=-1/2$ differenzierbar, denn 
		\begin{eqnarray*}
		\lim\limits_{x\uparrow 1} g(x) &=& \lim\limits_{x\uparrow 1} x+b = 1+ b \stackrel{!}{=} f(1) = a \Leftrightarrow \; 1+b=a, \quad   (\textnormal{Stetigkeit})\\
			\lim\limits_{x\uparrow 1} g'(x) &=& \lim\limits_{x\uparrow 1} 1 \stackrel{!}{=}  g'(1) = 2 \cdot a \Leftrightarrow \; a = \frac{1}{2} \quad (\textnormal{Differenzierbarkeit})\,.   
	\end{eqnarray*}
	Also zusammen $a = \frac{1}{2} = -b$.
	\end{itemize}
\end{exercisebox} 

\begin{exercisebox}[Differenzierbarkeit]
	Bestimmen Sie die Ableitung der Funktion $f:x\mapsto \sqrt{x}$ mit Hilfe des Differenzenquotienten.

\vspace{0.3cm}
\noindent{\bf L�sung:}\newline
\begin{eqnarray*} 
	\frac{\sqrt{x+h} - \sqrt{x} }{h} &=& \frac{ ( \sqrt{x+h} - \sqrt{x} )\cdot (\sqrt{x+h} + \sqrt{x}) }{h \cdot (\sqrt{x+h} + \sqrt{x} )} \\
	&=& \frac{ (x+h) - x }{h \cdot (\sqrt{x+h} + \sqrt{x} )} \\ 
	&=& \frac{ h }{h \cdot (\sqrt{x+h} + \sqrt{x} )} \\
	&& \Rightarrow \lim\limits_{h\to 0} \frac{\sqrt{x+h} - \sqrt{x} }{h} \stackrel{(*)}{=} \frac{\lim\limits_{h\to 0} h}{\sqrt{ x + \lim\limits_{h\to 0} h} + \sqrt{x}} = \frac{1}{2\sqrt{x}} = f'(x)\,. 
\end{eqnarray*}
$(*)$ $\sqrt{\cdot}: [0,\infty) \to [0,\infty)$ ist stetig.
\end{exercisebox}

\begin{exercisebox}[Bedeutung der Ableitung]
	Skizzieren Sie die Ableitung f�r Funktion.
	\begin{center}
	\includegraphics[width=0.75\textwidth]{../Mathematik_1/Bilder/Funktion_Ableitung.png}
\end{center}

\vspace{0.3cm}
\noindent{\bf L�sung:}\newline
	\begin{center}
	\includegraphics[width=0.75\textwidth]{../Mathematik_1/Bilder/Funktion_Ableitung_Loesung.png}
\end{center}
\end{exercisebox} 

\begin{exercisebox}[Differenzieren]
	Berechnen Sie von den folgenden Funktionen die Ableitungen. Es ist $a\in \mathbb R$ ein Parameter.
	\begin{eqnarray*}
		\begin{array}{ll}
			(a)\; f(x) = \mathrm{cosh}(x) = \displaystyle{\frac{e^x + e^{-x}}{2}} &\quad (b)\; g(u) = \displaystyle{\frac{u^2}{u^2+a^2}} \\[2ex]
			(c)\; h(p) = \mathrm{exp}\left(-p^2/(2a)^2\right) &\quad (d)\; i(x) = \displaystyle{\frac{(2x+3)^2-2}{\sqrt{x-1}}} \\
		\end{array}
	\end{eqnarray*}

	\vspace{0.3cm}
\noindent{\bf L�sung:}
\begin{itemize}
	\item[(a)] $f'(x) = \displaystyle{\frac{e^x - e^{-x}}{2}} = \mathrm{sinh}(x)$
	\item[(b)] $g'(u) = \displaystyle{\frac{2a^2 u}{(u^2+a^2)^2}}$
	\item[(c)] $h'(p) = \displaystyle{-\frac{p}{2 \cdot a^2}} \cdot \mathrm{exp}\left( -p^2/(2\cdot a)^2\right)$
\item[(d)] $i'(x) = \displaystyle{ \frac{4(2x+3)}{\sqrt{x} - 1 }} - \displaystyle{ \frac{(2x+3)^2 - 2 }{2(x-1)^{3/2} }}$
	\end{itemize}
\end{exercisebox}

\begin{exercisebox}[Anwendungen: Monotonie-Satz]
Angenommen Sie wissen �ber eine Funktion $f$, dass ihre Ableitung auf einem Intervall $I$ negativ ist, also $f'(x) \leq 0$. Was k�nnen Sie dann �ber das Monotonie-Verhalten aussagen?

\vspace{0.3cm}
\noindent{\bf L�sung:}\newline
$f$ ist auf $I$ monoton fallend. Es gelten die folgenden Monotonies�tze:
\begin{enumerate}
	\item $\forall x \in I: \; f'(x) = 0 \Leftrightarrow \, f$ ist konstant.
	\item $\forall x \in I: \; f'(x) \geq 0 \Leftrightarrow \, f$ monoton wachsend.
	\item $\forall x \in I: \; f'(x) \leq 0 \Leftrightarrow \, f$ monoton fallend.
	\item $\forall x \in I: \; f'(x) > 0 \Rightarrow \, f$  streng mononton wachsend.
	\item $\forall x \in I: \; f'(x) < 0 \Rightarrow \, f$  streng mononton fallend.
\end{enumerate}
\end{exercisebox}

\begin{exercisebox}[Anwendungen: Lineare N�herung]
      Die Mittellinie einer Rennstrecke der Breite $2\, \mathrm{m}$ wird durch Funktion $x\mapsto 4-\frac12 x^2$ beschrieben. Wir beobachten einen Rennwagen der die Bahnkurve im Uhrzeigersinn entlang rast. Bei spiegelglatter Fahrbahn rutscht das Fahrzeug und landet im Punkt $Y(0|6)$ in den Strohballen. Wo hat das Fahrzeug die Stra�e verlassen? 

\vspace{0.3cm}
\noindent{\bf L�sung:}\newline
er Rennwagen verl�sst seine Bahn in einem bestimmten Punkt in tangentialer Richtung. Er bewegt sich anschlie�end in etwa geradlinig fort und landet im Punkt $Y(0|6)$ in den Strohballen. Eine Zeichnung l�sst vermuten, dass  $Y(0|6)$ auf der Tangente liegt, die die Bahnkurve im Punkt $B(-2|2)$ schneidet. Diese Tangente hat die Gleichung
\[ y = m x + b \; \textnormal{mit}\; m = f'(-2) = \frac{6-2}{0-(-2)} \land b = 6 \,.\]

      Der Punkt, in dem der Rennwagen die Stra�e verl�sst ist also der Schnittpunkt der Randkurve der Stra�e mit der Tangente am Punkt $B(-2|2)$. Die Gleichung der Randkurve ist \[ y=5-\frac12 x^2\] und damit erh�lt man 
      \[ 5-\frac12 x^2 = 2x + 6 \Leftrightarrow x_{1,2} = \frac{-4 \pm \sqrt{8}}{2}\,.\] Gesucht ist der Punkt $( \frac{-4 + \sqrt{8}}{2}| 2+\sqrt{8})$.
\end{exercisebox} 
