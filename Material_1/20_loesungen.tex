\section*{Lineare Gleichungssysteme L�sungen}\label{MA1-20-Aufgaben-Loesungen}

\ifdefined\MOODLE 
	Regeln und Beispiele, die zum Verst�ndnis der L�sungswege hilfreich sind, finden Sie in \ref{MA1-20}.
\else
	Regeln und Beispiele, die zum Verst�ndnis der L�sungswege hilfreich sind, finden Sie im Skript.
\fi 


\begin{exercisebox}[Rechenregeln]
Wie muss $\boldsymbol b$ gew�hlt werden, dass gilt $A\boldsymbol x =\boldsymbol b$ mit 
\[ \left( \begin{array}{ccccc} -1 & 0 & 0 & 3 & -2 \\  2 & -1 & 1 & 0 & 0 \\ 1 & 1 & -1 & -2 & 0 \end{array}\right), \quad \boldsymbol x = \left( -1, 2,-2,3,5\right)^\intercal\,.\]

  \hspace{0.3cm}
  \newline
  {\bf L�sung:}
  Zu berechnen ist die Matrix-Vektor Multiplikation
  \begin{eqnarray*}
	  \left( \begin{array}{ccccc} -1 & 0 & 0 & 3 & -2 \\  2 & -1 & 1 & 0 & 0 \\ 1 & 1 & -1 & -2 & 0 \end{array}\right) \left(\begin{array}{c} -1 \\ 2 \\ -2 \\3 \\5 \end{array} \right) 
%&=& \left( \begin{array}{c} 
% (-1)\cdot (-1) + 0\cdot 2 + 0\cdot (-2) + 3 \cdot 3 + (-2) \cdot 5 \\
% 2\cdot (-1) + (-1)\cdot 2 + 1\cdot (-2) + 0 \cdot 3 + 0 \cdot 5 \\
% 1\cdot (-1) + 1\cdot 2 + (-1)\cdot (-2) + (-2) \cdot 3 + 0 \cdot 5 \\
%  \end{array}\right) \\ 
  &=&  \left( \begin{array}{c} 0 \\ -6 \\ -3\end{array}\right) \,.
  \end{eqnarray*}
\end{exercisebox}

\begin{exercisebox}[L�sbarkeitsanalyse]
	Gegeben seien \[ A = \left( \begin{array}{cc} 1 & 1 \\ 2 & \alpha \end{array}\right), \quad \boldsymbol b = \left( \begin{array}{c} -2 \\ \beta \end{array}\right)\] mit Parametern $\alpha, \beta\in \mathbb R$. 
	\begin{itemize}
		\item[(a)] Bestimmen Sie den Rang der Matrix $A$ in Abh�ngigkeit von $\alpha$.
		\item[(b)] Bestimmen Sie den Rang der erweiterten Matrix $(A|b)$ in Abh�ngigkeit von $\alpha$ und $\beta$.
		\item[(c)] Bestimmen Sie in Abh�ngigkeit der beiden Parameter, wann das lineare Gleichugnssystem $A\boldsymbol x = \boldsymbol b$ l�sbar ist und wann es genau eine L�sung gibt.
	\end{itemize}

  \hspace{0.3cm}
  \newline
  {\bf L�sung:}
  F�r die Bestimmung von $\mathrm{rk}(A)$ und $\mathrm{rk}(A_{\boldsymbol b}$ bringen wir die erweiterte Koeffizientenmatrix $\mathrm{rk}(A_{\boldsymbol b})$ auf Dreiecksform und lesen die gesuchten Gr��en ab: 
		  \begin{eqnarray*}
			  \left( \begin{array}{cc|c} 1 & 1 & -2 \\ 2 & \alpha & \beta \end{array}\right) \Leftrightarrow 
			  \left( \begin{array}{cc|c} 1 & 1 & -2 \\ 0 & \alpha-2 & \beta+4 \end{array}\right) \,. 
		  \end{eqnarray*}
		  \begin{itemize}
			  \item[(a)] 
				  $\mathrm{rk}(A) = \begin{cases} 1 \Leftrightarrow \alpha = 2\,,\\ 2  \Leftrightarrow \alpha \not= 2\,. \end{cases} $
		  \item[(b)] 
			  $\mathrm{rk}(A_{\boldsymbol b}) = \begin{cases} 1 \Leftrightarrow \alpha = 2 \land \beta = -4\,,\\ 2  \Leftrightarrow \alpha \not= 2 \lor \beta \not=-4\,. \end{cases}$ 
		  \item[(c)]  Um zu sehen f�r welche Werte von $\alpha$ und $\beta$ das Gleichungssystem l�sbar ist, bestimmen wir den Fall, f�r den das Gleichungssystem nicht l�sbar ist und negieren dann die Aussage:
			  \begin{eqnarray*}
				  \textnormal{Gleichungssystem nicht l\"osbar } &\Leftrightarrow& \; \alpha = 2 \land \beta \not= -4 \\ 
				  \Rightarrow \; \textnormal{Gleichungssystem l\"osbar } &\Leftrightarrow& \; \neg \left(  \alpha = 2 \land \beta \not= -4\right) \Leftrightarrow    \alpha \not= 2 \lor \beta = -4
			  \end{eqnarray*}
			  Das lineare Gleichungssystem ist eindeutig l�sbar genau dann, wenn $\mathrm{rk}(A) = \mathrm{rk}(A_b) = 2$, also falls $\alpha \not=2 $ und $\beta\not=-4$.
	  \end{itemize}
\end{exercisebox}
\begin{exercisebox}[Gleichungssysteme]
	Analysieren Sie die L�sbarkeit der folgenden Gleichungssysteme und bestimmen Sie die L�sungsmenge:
	\begin{eqnarray*}
		\begin{array}{lll} 
			(a) \; \begin{cases}
				x_1 -x_2 - x_3 &= 0\\
				x_1 +x_2 + 3x_3 &= 4\\
				     x_2 + 2x_3 &= 2\\
				     2x_1 + x_3 &= 3\\
    \end{cases} 
    &\;\;  
    (b)\; \begin{cases}
	    x_1 + x_2 - x_3 + x_4 - x_5 &= 1\\
    -x_1 + x_2 - 3 x_3 + 3x_5 &= 2\\
    x_2 - 2 x_3 + x_4 - x_5 &= -1\\
    2x_1 + 2 x_3 + x_4 - 2 x_5 &= 1
  \end{cases}
  \;\; (c)\; \begin{cases}
	x_1 + x_2 - x_3 &=1 \\
	2x_1 + 3x_2 + px_3 &= 3 \\
	x_1 + px_2 + 3x_3 &= 2
    \end{cases}    \\[3ex]
	(d)\; \begin{cases}
    3x_1 + x_2-4x_3 &= 2\\
    x_1 + x_2&= 0\\
    4x_1 +5x_3 + x_4 &= 1 \\
    6x_2 +x_3 + 2x_4 &= 1
  \end{cases}
    &\;\;  
  (e)\;\begin{cases}
	  (2+i)\cdot z_1 + i\cdot z_3 &= -4+3i \\ 
		(1+3i)\cdot z_1 + (-1-i)\cdot z_2  - z_3 &= -6-2i \\ 
		(4+2i)\cdot z_2 + (3+3i)\cdot z_3 &= -2+4i 
	\end{cases} 
  \end{array}
  \end{eqnarray*}



  \hspace{0.3cm}
  \newline
  {\bf L�sung:}
	\begin{itemize}
		\item[(a)]  $x_1 = 1, x_2 = 0, x_3 = 1$ 
		\item[(b)] $x_1 = 2-\lambda, x_2 = 1+2\lambda, \; x_3 = \lambda, x_4=-1, x_5=1$ 
		\item[(c)]  
			\begin{eqnarray*}
				\left( \begin{array}{ccc|c} 1 & 1 & - 1 &1 \\ 2 & 3 & p & 3 \\ 1 & p & 3 & 2 \end{array}\right) {\Leftrightarrow }
				\left( \begin{array}{ccc|c} 1 & 1 & - 1 &1 \\ 0 &1 & 2+p & 1 \\ 0 & p-1 & 4 & 1 \end{array}\right) \stackrel{ p\not=1}{\Leftrightarrow }
				\left( \begin{array}{ccc|c} 1 & 1 & - 1 &1 \\ 0 & 1 & 2+p & 1 \\ 0 & 0 & 4 + (1-p)(2+p) & 2-p \end{array}\right)\,, 
			\end{eqnarray*}
			wobei die faktorisierte Form des Eintrags an Position mit Index \(33\) lautet: 
			\[ 4+(1-p)(2+p) = -p^2-p+6 = -(x-2)(x+3)\,.\]
			Das hei�t, dass die F�lle \( p=1\), \(p=2\), \(p=-3\) und \(p\in \mathbb R\setminus\{1,2,-3\}\) separat zu untersuchen sind: 
			\begin{enumerate}
				\item F�r $p=1$ lautet die L�sung $x_1=1, x_2 = x_3 = \displaystyle{\frac14}\,.$  
				\item F�r $p=-3$ ist das System nicht l�sbar.  
				\item F�r $p=2$ hat das System unendlich viele L�sungen: $x_1 5\lambda, \, x_2 = 1-4\lambda, \; x_3 = \lambda, \; \lambda \in \mathbb R\,.$  
				\item F�r $p\in \mathbb R\setminus\{1,-3,2\}$ lautet die eindeutige L�sung $x_1 = 1, x_2 = x_3 =\displaystyle{ \frac{1}{3+p} }\,.$
			\end{enumerate}
			
  \item[(d)]  
			\begin{eqnarray*}
				\left( \begin{array}{cccc|c} 3 & 1 & -4 & 0 & 2 \\ 1 & 1 & 0 & 0 & 0 \\ 4 & 0 & 5 & 1 & 1 \\ 0 & 6 & 1 & 2 & 1 \end{array}\right) &\Leftrightarrow & 
				\left( \begin{array}{cccc|c}  1 & 1 & 0 & 0 & 0 \\ 3 & 1 & -4 & 0 & 2 \\4 & 0 & 5 & 1 & 1 \\ 0 & 6 & 1 & 2 & 1 \end{array}\right) \\
				&\Leftrightarrow & 
				\left( \begin{array}{cccc|c}  1 & 1 & 0 & 0 & 0 \\ 0 & 1 & 2 & 0 & -1 \\0 & 0 & 2 & 3 & 4 \\ 0 & 0 & 0 & 37 & 58 \end{array}\right)  \Rightarrow \left( \begin{array} {c} x_1 \\x_2 \\x_3 \\ x_4 \end{array}\right) =  \frac{1}{37}\left( \begin{array} {c} 41 \\-41 \\ 2 \\ 58 \end{array}\right) 
			\end{eqnarray*}

  \item[(e)] 
			\begin{eqnarray*}
				\left( \begin{array}{ccc|c} 2+i & 0 & j & -4+3i \\ 1+3i & -1-i & -1 & -6 -2i \\ 0 & 4+2i & 3+3i & -2 + 4i \end{array}\right) &\Leftrightarrow & 
				\left( \begin{array}{ccc|c} 2+i & 0 & j & -4+3i \\ 0 & -1-i & -i & -6 1-i \\ 0 & 4+2i & 3+3i & -2 + 4i \end{array}\right) \\ &\Leftrightarrow  &
				\left( \begin{array}{ccc|c} 2+i & 0 & j & -4+3i \\ 0 & -1-i & -i & -6 1-i \\ 0 & 0 & 2 & 0 \end{array}\right)
			\end{eqnarray*}
			L�sen wir die Gleichungen von unten nach oben auf, dann erh�lten wir 
			\[ z_1 = -1+2i, \; z_2 = i, \; z_3 = 0\,.\]
	  \end{itemize}
\end{exercisebox}
\begin{exercisebox}[Gleichungssysteme]
	Gesucht ist eine Matrix $X\in \mathbb R^{2\times 2}$ so, dass \[ A X + X A^\intercal = C, \quad A=\left( \begin{array}{cc} 1 & -1 \\ 2 & 2 \end{array}\right), \; C = \left( \begin{array}{cc} 4 & 0 \\ 6 & 4 \end{array}\right) \,. \]
	Bestimmen Sie ein lineares Gleichungssystem f�r die unbekannten Koeffizienten $x_{ij},\; i,j=1,2$ der Matrix $X$ und l�sen Sie es.

  \hspace{0.3cm}
  \newline
  {\bf L�sung:}
  Es muss gelten 
  \[ 
  \left( \begin{array}{cc} -x_{12} - x_{21} + 2 x_{11} & -x_{22}+3x_{12} + 2x_{11} \\ -x_{22} + 3 x_{21} + 2 x_{11}  & 4 x_{22} + 2 x_{12} + 2 x_{21}  \end{array}\right) = \left( \begin{array}{cc} 4 & 0 \\6 & 4 \end{array}\right) \,. \]
  Also 
  \[ 
  \left( \begin{array}{cccc| c } 2 & -1 & -1 & 0 & 4 \\ 2 & 0 & 3 & -1 & 0 \\ 2 & 3 & 0 & -1 &  6 \\ 0 & 2 & 2 & 4 & 4 \end{array}\right) \]
  Dieses System l�sen wir mittels Gau�-Verfahren und erhalten 
  \[ \Leftrightarrow 
  \left( \begin{array}{cccc| c } 2 & -1 & -1 & 0 & 4 \\ 0 & 1 & 4 & -1 & 4 \\ 0 & 0 & -5 & 1 &  6 \\ 0 & 0 & 0 & 1 & 1 \end{array}\right) \Rightarrow x_{22}= 1, x_{21} = -1, x_{12} = 1, x_{11} = 2\,. \]
  \end{exercisebox}
