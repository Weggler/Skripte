\section*{Komplexe Zahlen I}\label{MA1-14-Aufgaben-Loesungen}

\ifdefined\MOODLE 
	Regeln und Beispiele, die zum Verst�ndnis der L�sungswege hilfreich sind, finden Sie in \ref{MA1-14}.
\else
	Regeln und Beispiele, die zum Verst�ndnis der L�sungswege hilfreich sind, finden Sie im Skript.
\fi 

\begin{exercisebox}[Rechenregeln] %Andreas
	Gegeben sinde die komplexen Zahlen $z_1: = -2 + 4i$ und $z_2:= 1-3i$.
	\begin{itemize}
		\item[(a)] Berechnen Sie $\quad i\cdot z_1 - 2 \cdot z_2, \quad z_1 \cdot z_2, \quad \displaystyle{\frac{z_1}{z_2}}, \quad \displaystyle{\frac{2z_1 z_2}{z_1^2+z_2^2}}$.
		\item[(b)] Bestimmen Sie $\quad |z_1|,\quad |z_2|,\quad |z_1\cdot z_2|$.
		\item[(c)] Bestimmen Sie $\quad \overline{z_1},\quad \overline{z_2},\quad \overline{z_1}\cdot \overline{z_2}$.
		\item[(d)] Stellen Sie die folgenden Zahlen in der kartesischen Form dar ($a,b\in\mathbb R$):
			\[ \frac{3-21i}{4-3i} + 3(i-8), \quad (2-4i)^2+\frac{|1-\sqrt{3}i|}{i}, \quad \left| \frac{a+bi}{a-bi} + \frac{a-bi}{a+bi}\right| \]
	\end{itemize}
  {\bf L�sung:}
  \begin{itemize}
	  \item[(a)] $-6+4i, \quad  10+2i, \quad -\dfrac{7}{5} - \dfrac{1}{5}i, \quad \dfrac{2z_1 z_2}{z_1^2+z_2^2} = \dfrac{-132+90i}{221}$
	  \item[(b)] $\sqrt{20}, \quad \sqrt{10}, \quad \sqrt{200}$
	  \item[(c)] $1+3i, \quad -2-4i, \quad 10-10i$
	  \item[(d)] $-21, \quad -12 + 18i, \quad 2\Big|\displaystyle{\frac{a^2-b^2}{a^2+b^2}}\Big|$
  \end{itemize}
\end{exercisebox}

\begin{exercisebox}[Ungleichungen] %Lucy
Bestimmen Sie die Punkte $z\in\mathbb C$, die folgende Eigenschaften besitzen:
\begin{itemize}
	\item[(a)] $0<|z+i|<2$ 
	\item[(b)] $|z-z_1| = |z-z_2|$ 
	%\item $|z|=1$ (Kreislinie)
	%\item $|z|\leq 1$ (Kreisschreibe)
	%\item $|z|> 1$  %($\mathbb R^2 \setminus \{\textnormal{Kreisschreibe}\}$)
	%\item $0<\mathrm{arg}( z) < \pi/4$  %(Ecke, $\sin(\pi/4) = \cos(\pi/4)$)
% 	\item $\mathrm{Re}\frac{1}{z} = \frac{1}{2a}, \quad a>0, a\in \mathbb R$ (verschobene Kreislinie)\par
%
%			Beachte:
%			\begin{eqnarray*}
%				\frac{1}{x+iy} = \frac{x+iy}{x^2 - y^2} = \frac{x}{x^2 + y^2} -i \frac{y}{x^2 + y^2} = \frac{x}{x^2 + y^2}  
%			\end{eqnarray*} 
%			Also:
%               \begin{eqnarray*}
%	            \mathrm{Re}\left( \frac{1}{x+iy}\right) = \frac{x}{x^2-y^2} &=& \frac{1}{2a} \\
%	            \Leftrightarrow 2ax &=& x^2 + y^2 \Leftrightarrow  x^2 - 2ax + a^2 - a^2 + y^2 = 0 \\
%	            \Leftrightarrow (x-a)^2 + y^2 &=& a^2
%                \end{eqnarray*}
	\item[(c)] $ |z^2-\overline z^2| \geq 4 $
	%\item[(d)] $ |z^2+\overline z^2| \geq 4 $
	%\item[(e)] Welche Punktmenge der Gau�'schen Zahlenebene erf�llen (c) und (d)?
	\end{itemize}
  {\bf L�sung:}
  \begin{itemize}
	  \item[(a)] Die Menge aller $z\in\mathbb C$ mit $0<|z+i|<2$ entspricht einer Kreisscheibe mit Radius $2$, deren Mittelpunkt  $M(0|-i)$ ist. Der Mittelpunkt selbst ist in der Menge nicht enthalten.
			\begin{eqnarray*}
				0^2 = 0 <  (z+i)\overline{(z+i)}  = (x+i (y+1))\cdot(x-i(y+1)) = x^2+(y+1)^2 < 2^2=4  
			\end{eqnarray*}
	\item[(b)] Die Menge aller Punkte $z\in\mathbb C$ mit $|z-z_1| = |z-z_2|$ sind die Punkte, deren Abstand zu dem gegebenen Punkt $z_1$ gleich gro� ist wie zu dem gegebenen Punkt $z_2$. Also die Mittelsenkrechte zwischen den gegebenen Punkten $z_1$ und $z_2$.
	\item[(c)]  $ \{ z = x+iy \in \mathbb C:   |z^2-\overline z^2| \geq 4 \} $ 
	\begin{eqnarray*}
		 |z^2-\overline z^2|\geq 4 \Leftrightarrow | \underbrace{( z-\overline z)}_{\displaystyle{2\cdot y }} \cdot \underbrace{(z+\overline z)}_{\displaystyle{2\cdot x}}| \geq 4  \Leftrightarrow |4 \cdot x\cdot y |\geq 4 \Leftrightarrow |x\cdot y |\geq 1 
	\end{eqnarray*}
	\begin{minipage}{0.3\textwidth }
	\includegraphics[width=\textwidth]{../Mathematik_1/Bilder/KomplexeZahlen_1.png} \hspace{0.5cm}
\end{minipage}
\begin{minipage}{0.7\textwidth }
	Fallunterscheidung: 
	\begin{eqnarray*}
		\begin{cases} x>0, y>0:\quad |x\cdot y | = x\cdot y {\geq} 1  & \Leftrightarrow y \geq \displaystyle{\frac1x} \\[1ex]
			x>0, y<0:\quad |x\cdot y | = -x\cdot y {\geq} 1 & \Leftrightarrow y \leq -\displaystyle{\frac1x} \\[1ex] 
			x<0, y>0:\quad |x\cdot y | = -x\cdot y {\geq} 1 & \Leftrightarrow y \geq -\displaystyle{\frac1x} \\[1ex]
		x<0, y<0:\quad |x\cdot y | = x\cdot y {\geq} 1  & \Leftrightarrow y \leq \displaystyle{\frac1x}  
	\end{cases}
	\end{eqnarray*}
\end{minipage}
  \end{itemize}
\end{exercisebox}
%\item[(d)]  $ \{ z = x+iy \in \mathbb C:   |z^2+\overline z^2| \leq 4 \} $ 
%	\begin{eqnarray*}
%		 |z^2+\overline z^2|\geq 4 \Leftrightarrow | ( x+iy)^2 + (x-iy)^2| \leq 4  \Leftrightarrow | 2\cdot x^2 - 2 \cdot y^2 |\leq 4 \Leftrightarrow |x^2 -  y^2 |\leq 2 
%	\end{eqnarray*}
%	Fallunterscheidung: 
%	\begin{eqnarray*}
%		\begin{cases} 
%			x^2-y^2 < 0 \Leftrightarrow x^2<y^2 \Leftrightarrow |x| < |y|: &\quad |x^2-y^2 | = -x^2+y^2 = {\leq} 2  \\[1ex] & \Leftrightarrow |y|  \leq \sqrt{x^2+2} \Rightarrow \begin{cases} y \leq \sqrt{x^2+2},\; y \geq 0 \\[1ex] y \geq  -\sqrt{x^2+2}, \; y\leq 0 \end{cases} \\[1ex]
%			x^2-y^2 > 0 \Leftrightarrow x^2>y^2 \Leftrightarrow |x| > |y|: &\quad |x^2-y^2 | = x^2-y^2 = {\leq} 2 \\[1ex] & \Leftrightarrow |y|  \geq \sqrt{x^2-2} \Rightarrow \begin{cases} y \geq \sqrt{x^2-2} \land |x|>\sqrt{2},\; y \geq 0 \\[1ex] y \leq  -\sqrt{x^2-2} \land |x|>\sqrt{2} \; y\leq 0 \end{cases} \\[1ex]
%	\end{cases}
%	\end{eqnarray*}
%	\includegraphics[width=0.3\textwidth]{../Mathematik_1/Bilder/KomplexeZahlen_2.png} \hspace{0.5cm}
%	\includegraphics[width=0.3\textwidth]{../Mathematik_1/Bilder/KomplexeZahlen_3.png}

\begin{exercisebox}[Die imagin�re Einheit] %Lucy
	Berechnen Sie $i^{172}$ und $i^{175}$.

  \vspace{0.3cm}
  {\bf L�sung:}
	\begin{eqnarray*} 
		i^{4m} &=& 1, \; m\in \mathbb N_0 \\
		i^{4m+1} &=& i,  \; m\in \mathbb N_0 \\
		i^{4m+2} &=& -1,  \; m\in \mathbb N_0 \\
		i^{4m+3} &=& -i,  \; m\in \mathbb N_0 \\
		&&\\
		\Rightarrow && i^{172} = \left(i^4\right)^{43} = 1 \\
		\Rightarrow && i^{175} = \left(i^4\right)^{43}\cdot i^3 = -i
	\end{eqnarray*}
\end{exercisebox}

\begin{exercisebox}[Kartesische und Polarkoordinaten] %Andreas
	\begin{itemize}
		\item[(a)] Bestimmen Sie von folgenden komplexen Zahlen die Polarkoordinaten:
			\[ z_1=i,\quad z_2=\sqrt{3}-i, \quad z_3=-1-i, \quad z_4=x-\sqrt{3}x\,i\; (x>0) \]
		\item[(b)] Bestimmen Sie von folgenden komplexen Zahlen die kartesische Form:
			%\[ z_1= 1\cdot e^{-i4\pi/3},\quad z_2=\sqrt{2}a \cdot e^{-i\pi/4} \]
			\[ z_1 \; \widehat =\; \left( 1, -4\pi/3\right) ,\quad z_2 \; \widehat =\; \left( \sqrt{2}a,  -\pi/4\right) \]
	  \end{itemize}

\noindent{\bf L�sung:}
\begin{itemize}
	\item[(a)] 
\begin{eqnarray*}
	z_1 &\widehat =& \left( 1, \pi/2 \right)\\[1ex]
	z_2 &\widehat =& \left( 2,  \mathrm{atan2}(-1, \sqrt{3})= -\frac{\pi}{6}\right)\\[1ex]
	z_3 &\widehat =& \left(\sqrt{2}, \mathrm{atan2}(-1,-1)=-\frac34\pi \right)\\[1ex]
	z_4 &\widehat =& \left( 2x, \mathrm{atan2}(-\sqrt{3}x,x)=-\frac{\pi}{3} \right)\\[1ex]
\end{eqnarray*}
	\item[(b)] Nach Pythagoras gilt \(a = |z|\cos(\varphi)\) und \(b=|z|\sin(\varphi)\)
%\begin{eqnarray*}
%	z_1 &\widehat =& e^1 \cdot ( \cos(-4\pi/3) + i \sin(-4\pi/3)= -\frac{e}{2} + i \frac{\sqrt{3}e}{2}\\[1ex]
%		z_2 &\widehat =& \sqrt{2}a \cdot ( \cos(-\pi/4) + i \sin(-\pi/4)) = 2a-2ai\,.
%	\end{eqnarray*}
\begin{eqnarray*}
	z_1  &=&  \cos(-4\pi/3) + i \sin(-4\pi/3)= -\frac{e}{2} + i \frac{\sqrt{3}e}{2}\\[1ex]
		z_2 &=& \sqrt{2}a \cdot ( \cos(-\pi/4) + i \sin(-\pi/4)) = 2a-2ai\,.
	\end{eqnarray*}
\end{itemize}
\end{exercisebox}

\begin{exercisebox}[Kartesische und Polarkoordinaten] %Lucy
	Bestimmen Sie mit Hilfe der Polarkoordinaten die kartesische Form der folgenden komplexen Zahl: \[z= \left( \frac12 + \frac{\sqrt{3}}{2}i\right)^{60}\,.\]
\noindent{\bf L�sung:}
\[ \tilde z = \frac{1}{2} + \frac{\sqrt{3}}{2} i \Rightarrow \; \tilde z \;  \widehat{=} \; \left( 1, \pi/3 \right) \; \Rightarrow z = \tilde z^{60} \; \widehat = \;\left( 1^{60},  60\cdot \pi/4\right) = \left( 1^{60},  20\cdot \pi\right) \Rightarrow \, z = 1\,.\]
\end{exercisebox}

\begin{exercisebox}[Gleichungssystem] %Andreas
	L�sen Sie das folgende lineare Gleichungssystem
	\begin{eqnarray*}
		\begin{array}{r c r c l} 
			(3-4i)\cdot z_1 &+& (2-i)\cdot z_2 &=& 4-7i \\[1ex]
	(1-2i)\cdot z_1 &+& 4i\cdot z_2 &=& -11-15i 
\end{array}
\end{eqnarray*}
  {\bf L�sung:}
  Die erweiterte Koeffizientenmatrix lautet 
  \begin{eqnarray*}
	  &\left( \begin{array}{cccc} 3-4i & 2-i &|& 4-7i \\ 1-2i & 4i & | & -11-15i \end{array}\right) \\
	  \stackrel{(*)}{ \Leftrightarrow} \;& \left( \begin{array}{cccc} 3-4i & 2-i &|& 4-7i \\ 0 & -0.8 + 4.6i & | & 12.2-11.6i \end{array}\right) 
  \end{eqnarray*}
  wobei $(*)$ die folgende �quivalenzumformung meint: Addiere das $-(1-2i)/(3-4i)$-fache der ersten Zeile zur zweiten Zeile.
  Die zweite Zeile liefert eine lineare Gleichung f�r $z_2$: 
  \begin{eqnarray*} \begin{array}{clcr r}
	  & (-0.8+4.6i) z_2 &=& 12.2-11.6i\; & |:(-0.8+4.6i) \\
	  \Leftrightarrow & z_2 &=& -2+3i 
  \end{array}\end{eqnarray*}
  Setzt man $z_2$ in die erste Zeile ein, so erh�lt man eine lineare Gleichung f�r $z_1$:
  \begin{eqnarray*} \begin{array}{clcr r}
		  & (3-4i)z_1 + (-1+8i) &=& 4-7i\; & |-(-1+8i) \\
		  \Leftrightarrow & (3-4i)z_1 &=& 5-15i \; & |:(3-4i) \\
	  \Leftrightarrow & z_1 &=& 3-i 
  \end{array}\end{eqnarray*}
  Die L�sung lautet $z-1=3-i, \; z_2=-2+3j\,.$
\end{exercisebox}

