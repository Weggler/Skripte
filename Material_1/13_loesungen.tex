\section*{Trigonometrische Funktionen Aufgaben mit L�sungen}\label{MA1-13-Aufgaben-Loesungen}

\ifdefined\MOODLE 
	Regeln und Beispiele, die zum Verst�ndnis der L�sungswege hilfreich sind, finden Sie in \ref{MA1-13}.
\else
	Regeln und Beispiele, die zum Verst�ndnis der L�sungswege hilfreich sind, finden Sie im Skript.
\fi 

\begin{exercisebox}[Harmonische Schwingungen] %Andreas
	Bestimmen Sie die Amplitude, die Periode und die Nullphase der folgenden harmonischen Schwingungen und skizzieren Sie den Funktionsverlauf.
	\begin{eqnarray*}
		(a)\; t\mapsto 2\cdot \sin\left(3t-\frac{\pi}{6}\right) \quad \quad \quad \quad  (b)\; t\mapsto 10\cdot \sin(\pi t - 3\pi)
	\end{eqnarray*}

  \hspace{0.3cm}
  \newline
  {\bf L�sung:}
  \begin{center}
	  \includegraphics[width=0.8\textwidth]{../Mathematik_1/Bilder/Harmonische_Schwingungen.png}
  \end{center}  
  \begin{itemize}
	  \item[(a)] Die Amplitude $A$, die Periode $T$ und die Nullphase $\phi$ sind \[ A=2, \quad T=\frac{2\pi}{3}, \quad \phi=-\frac{\pi}{6} \]
	  \item[(b)]  Die Amplitude $A$, die Periode $T$ und die Nullphase $\phi$ sind \[ A=10, \quad T=\frac{2\pi}{\pi}=2, \quad \phi=-3\pi \]
  \end{itemize}
\end{exercisebox}

\begin{exercisebox}[Interpolation] %Andreas
	Eine Funktion der Form \[ f: x\mapsto \frac{a_0}{2} + a_1 \cos(x) + b_1\sin(x)\] soll durch die Punkte $(x_i, y_i),\, i=1,2,3$ mit \[ (0|1), (\pi/6|-1), (\pi/2|-3)\]
	gelegt werden. Bestimmen Sie ein Gleichungssystem f�r die Koeffizienten $a_0, a_1, b_1$ und l�sen Sie es.

  \hspace{0.3cm}
  \newline
  {\bf L�sung:}
  Allgemein muss gelten 
  \begin{eqnarray*}
	  \left( \begin{array}{ccc} 1/2 & \cos x_1 & \sin x_1 \\ 1/2 & \cos x_2 & \sin x_2 \\ 1/2 & \cos x_3 & \sin x_3 \end{array}\right) \cdot  \left( \begin{array}{c} a_0 \\ a_1 \\ a_2 \end{array}\right) =  \left( \begin{array}{c} y_1 \\ y_2 \\ y_3 \end{array}\right)  
  \end{eqnarray*}
  Im speziellen Fall ergibt sich 
  \begin{eqnarray*}
	  \left( \begin{array}{ccc} 1/2 & 1 & 0 \\ 1/2 & \sqrt{3}/2 & 1/2 \\ 1/2 & 0 & 1 \end{array}\right) \cdot  \left( \begin{array}{c} a_0 \\ a_1 \\ a_2 \end{array}\right) =  \left( \begin{array}{c} 1 \\ -1 \\ -3 \end{array}\right)  
  \end{eqnarray*}
  Die L�sung lautet $ a_0=2, \; a_1=0, \; a_2=-4 $ und f�r die Funktion $f(x) = 1-4\sin(x)\,.$
\end{exercisebox}

\begin{exercisebox}[Gleichungen] %Andreas
	Bestimmen Sie alle L�sungen der folgenden Gleichungen:
	\begin{eqnarray*}
		\begin{array}{ll}
			(a) \; \sin(2x+5) = 0,4 \quad \quad \quad & (b) \;\tan\left( 2(x+1)\right) = 1 \\[1ex] (c)\; \sqrt{\cos(x-1)} = 2^{-1/4} \quad \quad \quad & (d) \; \sin(x) = \sqrt{1-\sin^2(x)} 
		\end{array}
	\end{eqnarray*}

  \hspace{0.3cm}
  \newline
  {\bf L�sung:}
  \begin{itemize}
	  \item[(a)] Substituiere $u=2x+5$. Dann gilt $\sin(u)=0,4$ falls 
	  \[ u=\arcsin(0,4) + 2k\pi \;\textnormal{oder}\; u=\pi-\arcsin(0,4)+2k\pi, \]
		  $k\in \mathbb Z$. Wegen $x=(u-5)/2$ l�sen 
		  \[ x=\left( \arcsin(0,4) +2k\pi -5\right)/2 \;\textnormal{oder}\; x=\left(\pi -  \arcsin(0,4) +2k\pi -5\right)/2 \] die Gleichung.
	  \item[(b)] $u=2(x+1)$. Dann gilt $\tan(u) =1$ falls 
		  \[ u=\pi/4 + k\pi, \quad k\in \mathbb Z\,. \] 
		  Wegen $x=u/2-1$ folgt \[ x=\pi/8 + k\pi/2-1, \quad k\in \mathbb Z\,.\]
	  \item[(c)] Es gilt \begin{eqnarray*} \sqrt{\cos(x-1)} &=& 2^{-1/4} \quad |\cdot^2 \\ \Rightarrow \quad \cos(x-1) &=& 1/\sqrt{2} \end{eqnarray*} 
		  Setze $u=x-1$. Dann gilt $\cos(u) = 1/\sqrt{2}$ falls \[ u=\pi/4+2k\pi\quad \textnormal{oder}\quad u=-\pi/4 + 2k\pi, \quad k\in \mathbb Z\,.\] 
		  Mit $x=u+1$ folgt \[ x= \pi/4+1+2k\pi \quad \textnormal{oder}\quad x=1-\pi/4+2k\pi,\quad k\in \mathbb Z\,.\]
	  \item[(d)] Es gilt \begin{eqnarray*} \begin{array}{l rcl l} & \sin(x)&=& \sqrt{1-(\sin(x))^2} \quad &|\cdot^2 \\ \Rightarrow & (\sin(x))^2 &=& 1-(\sin(x))^2 &\\ \Leftrightarrow & (\sin(x))^2 &=& 1/2 & |\sqrt{\cdot} \\ \Leftrightarrow & \sin(x) &=& \pm 1/\sqrt{2} \end{array}\end{eqnarray*}
		  Wegen der ersten Gleichung muss $\sin(x)$ nichtnegativ sein, da die Wurzel auf der rechen Seite immer nichtnegativ ist. Daher muss $\sin(x)=1/\sqrt{2}$ gelten. Die L�sungen sind 
		  \[ x=\pi/4+2k\pi \quad \textnormal{oder}\quad x=3\pi/4+2k\pi,\quad k\in \mathbb Z\,.\] 
		  Die Probe ergibt 
		  \begin{eqnarray*} \sin(\pi/4 +2k\pi) &=& \sin(\pi/4) = 1/\sqrt{2} \\\sqrt{1-(\sin(\pi/4+2k\pi))^2} &=& \sqrt{1-1/2} = 1/\sqrt{2} \end{eqnarray*} 
		  und analog 
		  \begin{eqnarray*} \sin(3\pi/4 +2k\pi) &=& \sin(3\pi/4) = 1/\sqrt{2} \\\sqrt{1-(\sin(3\pi/4+2k\pi))^2} &=& \sqrt{1-1/2} = 1/\sqrt{2} \end{eqnarray*} 
		  Demnach sind alle L�sungen durch  \[ x=\pi/4+2k\pi \quad \textnormal{oder}\quad 3\pi/4 + 2k\pi,\quad k\in \mathbb Z\] gegeben.
  \end{itemize}
\end{exercisebox}

\begin{exercisebox}[Kreisgleichung] %Lucy
	Bestimmen Sie die Schnittpunkte der angegebenen Geraden $g$ mit dem Kreis $k:=\{(x,y):\; x^2+y^2=4\}$:
	\begin{eqnarray*}
		\begin{array}{lll}
		(a)\; g_1: x\mapsto x-2 \quad\quad (b) \; g_2: x\mapsto x-4 \quad\quad (c)\; g_3: x\mapsto 2 \end{array}
	\end{eqnarray*}


  \hspace{0.3cm}
  \newline
  {\bf L�sung:}
	\begin{itemize}
		\item[(a)] Zu l�sen ist die quadratische Gleichung $x^2-2x=0$. F�r $x_1=0$ erh�lt man den ersten Schnittpunkt $S_1=(0,-2)$ und f�r $x_2=2$ erh�lte man den zweiten Schnittpunkt $S_2=(2|0)$.
		\item[(b)] Zu l�sen ist die quadratische Gleichung $x^2-4x+6=0$. Diese hat keine reellen L�sungen, also schneiden sich Kreis und Gerade nicht.
		\item[(c)] Zu l�sen ist die quadratische Gleichung $x^2 = 0$. Die L�sung $x=0$ f�hrt auf den Schnittpunkt $(0,2)$.
	\end{itemize}
\end{exercisebox}

\begin{exercisebox}[Kreisgleichung] %Lucy
	Welche der Gleichungen beschreiben einen Kreis? Bestimmen Sie gegebenenfalls die Mittelpunkte und die Radien.
			\begin{eqnarray*} 
				\begin{array}{lll} 
					(a) \; (x+2)^2 + y^2 = 64 \quad \quad & (b)\; (x-5)^2 + (y+2)^2 = 0 \quad \quad & (c) \; x^2 + y^2 - 2x + 4y -20 = 0 \\[1ex] (d) \;x^2 + y^2 - 2x + 2y + 14 = 0 \quad \quad & (e) \; x^2 + y^2 +y = 0 &
				\end{array}
			\end{eqnarray*}

  \hspace{0.3cm}
  \newline
  {\bf L�sung:}
	\begin{itemize}
		\item[(a)] $(x+2)^2 + y^2 = 64 $ Kreis mit Radius $8$ und Mittelpunkt $M(-2|0)$ %b 
		\item[(b)] $(x-5)^2 + (y+2)^2 = 0 $ Diese Gleichung bestimmt einen einzigen Punkt, n�mlich $P(5|-2)$ %c
		\item[(c)] $x^2 + y^2 - 2x + 4y -20 = 0 \Leftrightarrow (x-1)^2+(y+2)^2 = 20$ Kreis mit Radius $\sqrt{20}$ und Mittelpunkt $M(1|-2)$ %e
		\item[(d)] $x^2 + y^2 - 2x + 2y + 14 = 0$ Diese Gleichung bestimmt weder Kurve noch Punkt %f
		\item[(e)] $x^2 + y^2 +y = 0 \Leftrightarrow x^2 + \left( y+\frac12\right)^2 = \frac{1}{4}$ Kreis mit Radius $\frac12$ und Mittelpunkt $M(0|-\frac12)$ %j
	\end{itemize}
\end{exercisebox}

\begin{exercisebox}[Qualitativer Graph] %Lovis
	\begin{itemize}
		\item[(a)] Zeichnen Sie die Funktion $f: x\mapsto 2\cdot \sin(x-\frac{\pi}{2})-2$ auf dem Intervall  $[-\pi,3\pi]$ und lesen Sie aus dem Graphen den Bildbereich, die Nullstellen und die Extremstellen von $f$ ab.
		\item[(b)] Zeichnen Sie die Funktion $f: x\mapsto -\sin(x-\pi)$ auf dem Intervall  $[0,\frac{5\pi}{2}]$ und lesen Sie aus dem Graphen den Bildbereich, die Nullstellen und die Extremstellen von $f$ ab.
	\end{itemize}

  \hspace{0.3cm}
  \newline
  {\bf L�sung:}
	\begin{itemize}
		\item[(a)] Die Sinuskurve ist entlang der $x$-Achse um $\pi/2$ nach rechts und entlang der $y$-Achse um $2$ nach oben verschoben und hat die Amplitude $2$. Der Wertebereich ist $[-4,0]$, die Nullstellen sind $\{-\pi,\pi,3\pi\}$ und die Extremstellen sind $\{-\pi, 0, \pi, 2\pi, 3\pi\}$.
		\item[(b)] Die Sinuskurve ist entlang der $x$-Achse um $\pi$ nach rechts und an $y$-Achse gespiegelt. Der Wertebereich ist $[-1,1]$, die Nullstellen sind $\{0,\pi,2\pi\}$ und die Extremstellen sind $\{\frac{\pi}{2}, \frac{3\pi}{2}, \frac{5\pi}{2}\}$.
	\end{itemize}
\end{exercisebox}
