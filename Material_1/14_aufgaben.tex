
\ifdefined\MOODLE 
\section*{Komplexe Zahlen I Aufgaben}\label{MA1-14-Aufgaben}

	Regeln und Beispiele, die zur Bearbeitung der Aufgaben hilfreich sind, finden Sie in \ref{MA1-14} und, falls Sie nicht weiterkommen, schauen Sie \hyperref[MA1-14-Aufgaben-Loesungen]{hier}.
\else
\section{Komplexe Zahlen I Aufgaben}\label{MA1-14-Aufgaben}

	Regeln und Beispiele, die zur Bearbeitung der Aufgaben hilfreich sind, finden Sie im Skript und, falls Sie nicht weiterkommen, schauen Sie \hyperref[MA1-14-Aufgaben-Loesungen]{hier}.
\fi 

\begin{exercisebox}[Rechenregeln] %Andreas
	Gegeben sinde die komplexen Zahlen $z_1: = -2 + 4i$ und $z_2:= 1-3i$.
	\begin{itemize}
		\item[(a)] Berechnen Sie $\quad i\cdot z_1 - 2 \cdot z_2, \quad z_1 \cdot z_2, \quad \displaystyle{\frac{z_1}{z_2}}, \quad \displaystyle{\frac{2z_1 z_2}{z_1^2+z_2^2}}$.
		\item[(b)] Bestimmen Sie $\quad |z_1|,\quad |z_2|,\quad |z_1\cdot z_2|$.
		\item[(c)] Bestimmen Sie $\quad \overline{z_1},\quad \overline{z_2},\quad \overline{z_1}\cdot \overline{z_2}$.
		\item[(d)] Stellen Sie die folgenden Zahlen in der kartesischen Form dar ($a,b\in\mathbb R$):
			\[ \frac{3-21i}{4-3i} + 3(i-8), \quad (2-4i)^2+\frac{|1-\sqrt{3}i|}{i}, \quad \left| \frac{a+bi}{a-bi} + \frac{a-bi}{a+bi}\right| \]
	\end{itemize}
\end{exercisebox}

\begin{exercisebox}[Ungleichungen] %Lucy
Bestimmen Sie die Punkte der Gau�schen Zahlenebene, die folgende Eigenschaften besitzen:
\begin{itemize}
	%\item $|z|=1$ (Kreislinie)
	%\item $|z|\leq 1$ (Kreisschreibe)
	%\item $|z|> 1$  %($\mathbb R^2 \setminus \{\textnormal{Kreisschreibe}\}$)
	%\item $0<\mathrm{arg}( z) < \pi/4$  %(Ecke, $\sin(\pi/4) = \cos(\pi/4)$)
	\item[(a)]$0<|z+i|<2$ %(verschobene Kreisscheibe ohne $(0,-i)$) 
			%\begin{eqnarray*}
			%	|z-z_0| < r \Rightarrow (z+i)\overline{(z+i)} = z\overline z + i(-i) = z^2+1 < 2 \Leftrightarrow z^2 < 1 \Leftrightarrow |z| <1 
			%\end{eqnarray*}
	\item[(b)] $|z-z_1| = |z-z_2|$ %(Mittelsenkrechte zwischen $z_1$ und $z_2$)
% 	\item $\mathrm{Re}\frac{1}{z} = \frac{1}{2a}, \quad a>0, a\in \mathbb R$ (verschobene Kreislinie)\par
%
%			Beachte:
%			\begin{eqnarray*}
%				\frac{1}{x+iy} = \frac{x+iy}{x^2 - y^2} = \frac{x}{x^2 + y^2} -i \frac{y}{x^2 + y^2} = \frac{x}{x^2 + y^2}  
%			\end{eqnarray*} 
%			Also:
%               \begin{eqnarray*}
%	            \mathrm{Re}\left( \frac{1}{x+iy}\right) = \frac{x}{x^2-y^2} &=& \frac{1}{2a} \\
%	            \Leftrightarrow 2ax &=& x^2 + y^2 \Leftrightarrow  x^2 - 2ax + a^2 - a^2 + y^2 = 0 \\
%	            \Leftrightarrow (x-a)^2 + y^2 &=& a^2
%                \end{eqnarray*}
	\item[(c)] \(|z^2-\overline z^2| \geq 4\)
	%\item[(d)] \( |z^2+\overline z^2| \geq 4\)
	%\item[(e)] Welche Punktmenge der Gau�'schen Zahlenebene erf�llen (c) und (d)?
	\end{itemize}
\end{exercisebox}

\begin{exercisebox}[Die imagin�re Einheit] %Lucy
	Berechnen Sie $i^{172}$ und $i^{175}$.
\end{exercisebox}

\begin{exercisebox}[Kartesische und Polarkoordinaten] %Andreas
	\begin{itemize}
		\item[(a)] Bestimmen Sie von folgenden komplexen Zahlen die Polarkoordinaten:
			\[ z_1=i,\quad z_2=\sqrt{3}-i, \quad z_3=-1-i, \quad z_4=x-\sqrt{3}x\,i\; (x>0) \]
%$z_1=\cos(\pi/2) + i\sin(\pi/2)$
%\begin{eqnarray*}
%	z_2 = 2\cdot \left( \cos\left(-\frac{\pi}{6}\right) + i \sin\left(-\frac{\pi}{6}\right)\right)
%\end{eqnarray*}
		\item[(b)] Bestimmen Sie von folgenden komplexen Zahlen die kartesische Form:
			\[ z_1 \; \widehat= \; 1\cdot e^{-i4\pi/3},\quad z_2 \; \widehat = \; \sqrt{2}a \cdot e^{-i\pi/4} \]
	  \end{itemize}
\end{exercisebox}


\begin{exercisebox}[Kartesische und Polarkoordinaten] %Lucy
	Bestimmen Sie mit Hilfe der Polarkoordinaten die kartesische Form der folgenden komplexen Zahl: \[z= \left( \frac12 + \frac{\sqrt{3}}{2}i\right)^{60}\,.\]
\end{exercisebox}

\begin{exercisebox}[Gleichungssystem] %Andreas
	L�sen Sie das folgende lineare Gleichungssystem
	\begin{eqnarray*}
		\begin{array}{r c r c l} 
			(3-4i)\cdot z_1 &+& (2-i)\cdot z_2 &=& 4-7i \\[1ex]
	(1-2i)\cdot z_1 &+& 4i\cdot z_2 &=& -11-15i 
\end{array}
\end{eqnarray*}
\end{exercisebox}


%\begin{exercisebox}[Schatzsuche]
%\end{exercisebox}

%\begin{exercisebox}[Binonialkoeffizient]
%	\begin{eqnarray*} 
%	(z_1+z_2)^n = \sum\limits_{k=0}^{n} \binom{n}{k} z_1^{n-k} z_2^k\,.
%	\end{eqnarray*}
%\end{exercisebox}

%\begin{exercisebox}[Parallelogramm]
%	In einem Parallelogramm ist die Sume der Quadrate �ber den Diagonalen gleich der Summe der Quadrate �ber allen der Seiten.
%	\begin{eqnarray*}
%		|z_1+z_2|^2 + |z_1 - z_2|^2 &=& (z_1+z_2)\overline{(z_1+z_2)} + (z_1-z_2)\overline{(z_1-z_2)} \\ 
%		&=& (z_1+z_2)\left( \overline{z_1} + \overline{z_2}\right) +   (z_1-z_2)\left( \overline{z_1} - \overline{z_2}\right)\\  
%	&=& z_1\overline{z_1} + z_1\overline{z_2} + z_2\overline{z_1} + z_2\overline{z_2} + z_1\overline{z_1} - z_2\overline{z_1} - z_1\overline{z_2} + z_2\overline{z_2} \\  
%	&=& 2 z_1\overline{z_1} + 2 z_2\overline{z_2} = 2\left( |z_1|^2 + |z_2|^2\right)\,.
%\end{eqnarray*}
%\end{exercisebox}

%\begin{exercisebox}[Kr�fteparallelogramm/ Addition von Vektoren]
%\end{exercisebox}

%\begin{exercisebox}[Zahlenbereiche]
%  Welche Aussage ist richtig? 
%  \begin{enumerate}
%	  \item $\mathbb R \subset \mathbb C$?
%	  \item $z\in \mathbb R \Rightarrow z = \overline{z}$
%  \end{enumerate}
%\end{exercisebox}
%
%\begin{exercisebox}[Rechnen mit komplexen Zahlen]
%  Welche Aussage ist richtig?
%  \begin{enumerate}
%	  \item $\overline{\overline z} = z$
%  	  \item $\frac{\overline{z_1}}{\overline{z_2}} = \overline{\frac{z_1}{z_2}}$
%  	  \item $\overline{z_1} \cdot \overline{z_2} = \overline{z_1 \cdot z_2}$
%  	  \item $\overline{z_1} + \overline{z_2} = \overline{z_1 + z_2}$
%  \end{enumerate}
%\end{exercisebox}

%\begin{exercisebox}[Komplexe Zahlen]
%	Es sei \(z\in \mathbb C\):\begin{enumerate}
%		\item Welche Punktmenge der Gau�'schen Zahlenebene wird durch die folgende Bedingung bestimmt?\[ |z^2-\overline z^2| \geq 4\]
%		\item Welche Punktmenge der Gau�'schen Zahlenebene wird durch die folgende Bedingung bestimmt?\[ |z^2+\overline z^2| \geq 4\]
%		\item Welche Punktmenge der Gau�'schen Zahlenebene erf�llt beide zuvor genannten Bedingungen?
%	\end{enumerate}
%
%	\vspace{0.3cm}
%	\noindent{\bf L�sung:} \newline
%
%\end{exercisebox}


