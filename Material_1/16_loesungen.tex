\section*{Vektoren L�sungen}\label{MA1-16-Aufgaben-Loesungen}

\ifdefined\MOODLE 
	Regeln und Beispiele, die zum Verst�ndnis der L�sungswege hilfreich sind, finden Sie in \ref{MA1-16}.
\else
	Regeln und Beispiele, die zum Verst�ndnis der L�sungswege hilfreich sind, finden Sie im Skript.
\fi 

\begin{exercisebox}[Rechenregeln]
	Gegeben seien die Vektoren 
	\[ \boldsymbol a = \left( \begin{array}{c} -4 \\ 3 \end{array}\right)\,, \;  
		\boldsymbol b = \left( \begin{array}{c} 0 \\ -3 \end{array}\right)\,,  \;
	\boldsymbol c = \left( \begin{array}{c} -2 \\ -2 \end{array}\right)\,. \] 
	\begin{itemize}
		\item[(a)] Berechnen Sie folgende Ausdr�cke: 
			\[ -\boldsymbol a + 3\cdot \boldsymbol b, \quad (2\cdot \boldsymbol a - \boldsymbol b) + 3\cdot \boldsymbol c, \quad (3\cdot \boldsymbol c - \boldsymbol b) + 2\cdot \boldsymbol a \]
		\item[(b)] Berechnen Sie die folgenden Betr�ge (L�ngen, Normen): 
			\[ |\boldsymbol a|\,,\quad |\boldsymbol b|\,,\quad |\boldsymbol a+ \boldsymbol b|\,,\quad |-\boldsymbol a+3\cdot \boldsymbol b| \]
		\item[(c)] Bestimmen Sie einen Vektor der L�nge $10$, der in die gleiche Richtung wie der Gegenvektor von $\boldsymbol a$ zeigt. 
		\item[(d)] Berechnen Sie die folgenden Ausdr�cke:
			\[\boldsymbol a\boldsymbol \cdot \boldsymbol b, \quad 
			\boldsymbol a\boldsymbol \cdot \boldsymbol c, \quad 
			\boldsymbol a\boldsymbol \cdot ( 4\cdot \boldsymbol c), \quad 
			\boldsymbol a\boldsymbol \cdot ( \boldsymbol b\boldsymbol \cdot \boldsymbol c), \quad 
		(\boldsymbol a\boldsymbol \cdot  \boldsymbol b)\boldsymbol \cdot \boldsymbol c \]
		\item[(e)] Bestimmen Sie die Winkel im Bogenma� zwischen folgenden Vektoren \[ \boldsymbol a\; \textnormal{und}\; \boldsymbol b, \quad \boldsymbol a \; \textnormal{und}\; \boldsymbol c\]
		\item[(f)] Bestimmen Sie alle orthogonalen Vektoren zu $\boldsymbol a$.
		\item[(g)] Bestimmen Sie einen Vektor der Form $\boldsymbol b+\lambda\cdot \boldsymbol c$ mit $\lambda \in \mathbb R$, der senkrecht zu $\boldsymbol a$ ist.
		\item[(h)] Bestimmen Sie die orthogonale Projektion von $\boldsymbol b$ auf $\boldsymbol a$, das hei�t die Vektoren $\boldsymbol b_\parallel$ und $\boldsymbol b_\perp$.
	\end{itemize}
	\noindent{\bf L�sung:} 
	\begin{itemize}
	  \item[(a)] Die L�sungen lauten 
		  \[ \left( \begin{array}{c} 4 \\ -12 \end{array}\right), \quad 
		   \left( \begin{array}{c} -14 \\ 3 \end{array}\right), \quad 
	   \left( \begin{array}{c} -14 \\ 3 \end{array}\right) \]
   \item[(b)] Die L�sungen lauten $5, 3, 4$ und $4\sqrt{10}$.
   \item[(c)] Gesucht ist ein Vektor $\boldsymbol b$, der in Richtung $-\boldsymbol a$ zeigt und dessen L�nge $10$ ist: \[\left\{ \begin{array}{rcl} |\boldsymbol b| &=& 10 \\ \boldsymbol b &\parallel& -\boldsymbol  a\end{array} \right\} \Rightarrow \boldsymbol b \stackrel{!}{=} \lambda ( \boldsymbol -a) \wedge |\boldsymbol b| = |\lambda ( \boldsymbol -a)| = 10.\]  Der gesuchte Vektor $\boldsymbol b$ ist also ein skalares Vielfaches des Vektors $-\boldsymbol a$. $\lambda$ ist durch die Bedingung an die L�nge von $\boldsymbol b$ eindeutig festgelegt, denn es gilt:
	   \[ |\boldsymbol b| = |\lambda \cdot ( - \boldsymbol a)|=\lambda^2 \cdot ( - \boldsymbol a)| = 10 \Leftrightarrow |\lambda| = 10/|-\boldsymbol a| = 10/|\boldsymbol a| = 10/5 = 2 \; \Rightarrow \; \boldsymbol b = 2\cdot(-\boldsymbol a) = \left( \begin{array}{c} 8\\-6\end{array}\right)\,.  \]
	   Bemerkung: Das Sklarprodukt ist bilinear, das hei�t f�r $\boldsymbol a, \boldsymbol b \in \mathbb R^n$ und $\lambda \in \mathbb R$ gilt 
	   \[(\lambda \cdot \boldsymbol a)\boldsymbol \cdot \boldsymbol b 
		   = \boldsymbol a\boldsymbol \cdot (\lambda \cdot \boldsymbol b) 
	   = \lambda ( \boldsymbol a\boldsymbol \cdot \cdot \boldsymbol b)  \Rightarrow \; |\lambda \cdot \boldsymbol a |=  \sqrt{(\lambda \cdot \boldsymbol a) \boldsymbol \cdot (\lambda \cdot \boldsymbol a)} = \sqrt{\lambda^2}\cdot \sqrt{ \boldsymbol a \boldsymbol \cdot \boldsymbol a} = |\lambda| \cdot |\boldsymbol a|\,.\]
	   \item[(d)] Die ersten drei Ausdr�cke ergeben $-9, 2$ und $8$. Es gilt 
		   \[ (\boldsymbol a\boldsymbol \cdot \boldsymbol b)  \cdot \boldsymbol c = \left( \begin{array}{c} 18 \\ 18 \end{array}\right), \quad \boldsymbol a \cdot (\boldsymbol b \boldsymbol \cdot \boldsymbol c) = \left( \begin{array}{c} -24 \\ 18 \end{array}\right)\,.\]
		   Die Reihenfolge der Verkn�pfungen ist also entscheidend!
	   \item[(e)]
		   Die Winkel ergeben sich zu \[ \pi - \arccos\left( \frac35\right) \approx 2.2142974, \quad \arccos\left( \frac{\sqrt{2}}{10} \right)  \approx 1.42889927 \]
		   \item[(f)] Die Vektoren, die zu $\boldsymbol a$ orthogonal sind, ist die folgende Menge: \[ M_{\boldsymbol a_\perp}:= \left\{ \boldsymbol a_\perp = \lambda \left( \begin{array}{c} 3 \\ 4 \end{array}\right), \; \lambda \in \mathbb R\right\} \,. \]
		   \item[(g)] Suche $\lambda \in \mathbb R$ mit \[ (\boldsymbol b + \lambda \cdot \boldsymbol c) \boldsymbol \cdot \boldsymbol a = 0 \Leftrightarrow \; \lambda = \frac92 \]
			   Der Vektor lautet \[ \left( \begin{array}{c} -9 \\ -12 \end{array}\right)\,. \] 
		   \item[(h)] Es gilt \[ \boldsymbol b_\parallel = \frac{1}{25} \cdot \left( \begin{array}{c} 36 \\ -27 \end{array}\right), \quad \boldsymbol b_\perp = \frac{1}{25} \cdot \left( \begin{array}{c} -36 \\ -48 \end{array}\right)\,. \]
  \end{itemize}
\end{exercisebox}

\begin{exercisebox}[Rechenregeln]
	Gegeben sind die Vektoren \[ \boldsymbol a = \left( \begin{array}{c} -2 \\ 0 \\ -4\end{array}\right), \quad \boldsymbol b = \left( \begin{array}{c} -2 \\1 \\ -2 \end{array}\right), \quad \boldsymbol c = \left( \begin{array}{c} -1 \\ 1 \\0 \end{array}\right) \,. 
	\]
		\begin{itemize}
			\item[(a)] Berechnen Sie $-3\cdot(-2\cdot \boldsymbol a + \boldsymbol b) - \boldsymbol c$.
			\item[(b)] Berechnen Sie die folgenden Betr�ge (L�ngen, Normen): $|\boldsymbol a|, |\boldsymbol b|, |-3\cdot(-2\cdot \boldsymbol a + \boldsymbol b) - \boldsymbol c|$.
			\item[(c)] Berechnen Sie den Einheitsvektor in Richtung $\boldsymbol a$.
			\item[(d)] Berechnen Sie $\boldsymbol a\boldsymbol \cdot \boldsymbol b$ und $(-3\cdot \boldsymbol a)\boldsymbol \cdot \boldsymbol b$.
			\item[(e)] Berechenn Sie den Winkel $\varphi$ im Bogenma� zwischen $\boldsymbol a$ und $\boldsymbol b$ (Skalarprodukt!).
			\item[(f)] Berechnen Sie $\boldsymbol a\times \boldsymbol b$ und $\boldsymbol c\times (\boldsymbol a\times \boldsymbol b)$.
			%\item[(g)] Pr�fen Sie anhand der definierten Vektoren, dass gilt \[ \boldsymbol c\times ( \boldsymbol a\times \boldsymbol b) = (\boldsymbol c\boldsymbol \cdot \boldsymbol b)\boldsymbol \cdot \boldsymbol a -(\boldsymbol c\boldsymbol \cdot \boldsymbol a)\boldsymbol \cdot \boldsymbol b\] 
			\item[(g)] Berechnen Sie die Fl�che des durch $\boldsymbol a$ und $\boldsymbol b$ aufgespannten Parallelogramms. 
\end{itemize}
  {\bf L�sung:}
  \begin{itemize}
	  \item[(a)] $\left( -5, -4, -18 )\right)^\intercal $
  \item[(b)] $2\sqrt{5}, 3, \sqrt{365} $
  \item[(c)] $\left(-1/\sqrt{5}, 0, -2/\sqrt{5}\right)^\intercal$
  \item[(d)] $12, -144$
  \item[(e)] $\arccos(2/\sqrt{5}) \approx 0.4636476$
  \item[(f)] $\left( 4, 4, -2\right)^\intercal, \; \left( -2, -2, -8\right)^\intercal $
  %\item[(g)] $\left( -2, -2, -8\right)^\intercal $
  \item[(g)] $6$
	  \end{itemize}
\end{exercisebox}

\begin{exercisebox}[Rechenregeln]
	Gegeben sind die folgenden Vektoren
	\[ \boldsymbol a = \left( \begin{array}{c} 1+2i \\ -i \\ 0 \\3 \end{array}\right), \quad \boldsymbol b = \left( \begin{array}{c} -3-i \\ 2i \\ 0 \\ -i\end{array}\right), \quad 
	\boldsymbol c = \left( \begin{array}{c} -2+i \\ 1+2i \end{array}\right), \quad \boldsymbol d = \left( \begin{array}{c} -i \\ -1-3i\end{array}\right)  \,. \]
	\begin{itemize}
		\item[(a)] Berechnen Sie $-i\boldsymbol a + (2-i)\boldsymbol b$.
		\item[(b)] Berechnen Sie $\langle \boldsymbol a, \boldsymbol b\rangle$.
		\item[(c)] Berechnen Sie $|\boldsymbol a|$.
		\item[(d)] Berechnen Sie die folgenden Ausdr�cke \[ |\boldsymbol c|, \quad \langle \boldsymbol c, \boldsymbol d\rangle, \quad \langle \boldsymbol d, \boldsymbol c\rangle\]
		\item[(e)] Bestimmen Sie eine Zerlegung von $\boldsymbol d$ der folgenden Art \[ \boldsymbol d = \boldsymbol d_\perp + \boldsymbol d_\parallel, \quad \boldsymbol d_\perp \perp \boldsymbol c\,.\]
	\end{itemize}
  {\bf L�sung:}
  \begin{itemize}
	  \item[(a)] $\left( -5, 1+4i, 0, -1-5i\right)^\intercal$
	  \item[(b)] $ -7+2i$
	  \item[(c)] $\sqrt{15}$
	  \item[(d)] $\sqrt{10}, \; -8+i, -8-i$
	  \item[(e)] Es gilt \[ \boldsymbol b_\parallel = \frac{\langle \boldsymbol a, \boldsymbol b\rangle }{ | \boldsymbol a|^2} \cdot \boldsymbol a, \quad \boldsymbol b_\perp = \boldsymbol b - \boldsymbol b_\parallel\,.\]
		  Damit folgt \[ \boldsymbol b_\parallel = \left( \begin{array}{c} 3/2 - i \\ -1-3/2 i \end{array} \right), \quad \boldsymbol b_\perp = \left( \begin{array}{c} 3/2 \\-3/2 i \end{array}\right)\,.
		  \]
  \end{itemize}
\end{exercisebox}


%\begin{exercisebox}[Kosinussatz]
%	Betrachten Sie ein Dreieck mit den Kantenvektoren $\boldsymbol a, \boldsymbol b$ und Winkel $\alpha$ und zeigen Sie, dass der Kosinussatz gilt: 
%	\[c^2 = a^2 + b^2 - 2\, a\, b \, \cos(\alpha)\,.\]
%
%	\vspace{0.3cm}
%	\noindent{\bf L�sung:}\newline
%Betrachtet man ein Dreieck mit den Kantenvektoren $\boldsymbol a, \boldsymbol b$ und Winkel $\alpha$ dann gilt:
%$$
%|\boldsymbol c|^2 =  |\boldsymbol b -\boldsymbol a|^2 = |\boldsymbol a|^2 + |\boldsymbol b|^2 - 2 \boldsymbol a \boldsymbol \cdot \boldsymbol b =  |\boldsymbol a|^2 + |\boldsymbol b|^2 - 2 |\boldsymbol a|\cdot | \boldsymbol b| \cdot \cos(\alpha)\,,
%$$
%Die L�ngen der Vektoren sind genau die Seitenl�ngen im Dreieck, also:
%\[ c^2 = a^2 + b^2 - 2ab\cos(\alpha)\,.\]
%\end{exercisebox}


%\begin{exercisebox}[Satz des Thales]
%	Jeder Winkel im Halbkreis ist ein rechter Winkel. 
%	
%	\vspace{0.3cm}
%	\noindent{\bf L�sung:}\newline
%  \begin{center}
%	  \includegraphics[width=0.8\textwidth]{../Mathematik_1/Bilder/thales.png}
%  \end{center}  
%  Betrachten wir einen Halbreis mit Radius $r$ und Mittelpunkt $M$. Wir definieren die Vektoren $\boldsymbol u:= \vec{CM}$ und $\boldsymbol v:=\vec{MB}$. Mit Hilfe von $\boldsymbol u$ und $\boldsymbol v$ k�nnen wir nun $\vec{CA}$ und $\vec{CB}$ ausdr�cken: \[ \vec{CA}= \boldsymbol u - \boldsymbol v \wedge \vec{CB} = \boldsymbol u + \boldsymbol v \] und damit das Skalarprodukt zwischen diesen beiden Vektoren �berpr�fen - falls es null wird muss der Winkel zwischen der Seite $b$ und der Seite $a$ neunzig Grad sein: \[ \vec{CA}\boldsymbol \cdot \vec{CB} = (\boldsymbol u - \boldsymbol v)\boldsymbol \cdot (\boldsymbol u + \boldsymbol v) = |\boldsymbol u|^2 - |\boldsymbol v|^2 = r-r = 0\,.\]
%\end{exercisebox}
%
%\begin{exercisebox}[Kreuzprodukt] 
%	Zeigen Sie allgemein, dass f�r $\boldsymbol u, \boldsymbol v\in \mathbb R^3$ der Vektor $\boldsymbol v$ senkrecht auf $\boldsymbol u \times \boldsymbol v$ steht. 
%
%	\vspace{0.3cm}
%	\noindent{\bf L�sung:}\newline
%  \begin{eqnarray*}
%	  \boldsymbol w: &=& \left( \begin{array}{c} u_2 v_3 - u_3 v_2 \\ u_3 v_1 - u_1 v_3 \\u_1 v_2 - u_2 v_1 \end{array}\right) \\  &&\Rightarrow \boldsymbol v \boldsymbol \cdot \boldsymbol w = v_1\cdot (u_2 v_3 - u_3 v_2) + v_2\cdot( u_3 v_1 - u_1 v_3) + v_3\cdot (u_1 v_2 - u_2 v_1) = \textcolor{blue}{v_1u_2v_3} \textcolor{green}{- v_1u_3v_2} \textcolor{green}{+v_2 u_3 v_1} + v_2 u_1 v_3 + v_3 u_1 v_2 \textcolor{blue}{-v_3 u_2 v_1} = 0 \,.
%  \end{eqnarray*}
%\end{exercisebox}

