\section*{Exponential- und Logarithmusfunktionen L�sungen}\label{MA1-10-Aufgaben-Loesungen}

\ifdefined\MOODLE 
	Regeln und Beispiele, die zum Verst�ndnis der L�sungswege hilfreich sind, finden Sie in Abschnitt \ref{MA1-10}.
\else
	Regeln und Beispiele, die zum Verst�ndnis der L�sungswege hilfreich sind, finden Sie im Skript.
\fi 

\begin{exercisebox}[Interpolation]
	Bestimmen Sie die Parameter $a$ und $b$ der Funktion \[ f:x\mapsto a\cdot e^{-bx}\] so, dass die Punkt $(0|8)$ und $(5|3)$ auf der Kurve liegen. Runden Sie die Parameter auf drei Nachkommastellen.

	\vspace{0.3cm}
	\noindent{\bf L�sung:}\newline
	Es muss gelten \[ a\cdot \underbrace{ e^{-b\cdot 0}}_{=1} = 8 \land a\cdot e^{-b\cdot 5} = 3 \]
	Aus der ersten Bedingung folgt $a=8$ und eingesetzt in die zweite Bedingung folgt \[ 8\cdot e^{-b\cdot 5} = 3 \Leftrightarrow -5\cdot b = \ln(3/8) \Leftrightarrow b = \frac{\ln(8/3)}{5} \,.\]
	Damit ergibt sich $b\approx 0,196165850 \approx 0,196$
\end{exercisebox}

\begin{exercisebox}[$f:x\mapsto 2^x$]
  Gegeben ist die Funktion $f$ mit $f: x\mapsto 2^x$
  \begin{itemize}
	  \item[(a)] Ermitteln Sie den maximalen Definitionsbereich $D$ und die Wertemenge $W$
	  \item[(b)] Berechnen Sie $f(-\frac14)$ auf $2$ Dezimalen genau. 
    \item[(c)] F�r welches $x\in D$ ist $f(x) = 8$?
    \item[(d)] F�r welche $x\in D$ ist $f(x) \leq 16$?
    \item[(e)] Zeigen Sie $f(x) \cdot f(-x) = 1$ f�r alle $x\in D$
    \item[(f)] Zeigen Sie $f(x+1) = 2\cdot f(x)$ f�r alle $x\in D$
  \end{itemize}

	\vspace{0.3cm}
	\noindent{\bf L�sung:}\newline
  \begin{itemize}
	  \item[(a)]  $f: \mathbb R \to \mathbb R, \; x\mapsto 2^x$
	  \item[(b)] Mit dem Taschenrechner erh�lt man sofort $f(-\frac14) = \displaystyle{\frac{1}{\sqrt{\sqrt{2}} }} \approx 0.84$.$^{(*)}$
	  \item[(c)] $f(3) = 8$
    \item[(d)] $f(x) = 2^x \leq 16 \Leftrightarrow x \leq 4$
    \item[(e)] $f(x) \cdot f(-x) = 2^x \cdot 2^{-x} = 2^{x-x} = 2^0 = 1$
    \item[(f)] $f(x+1) =  2^{x+1} = 2^x \cdot 2 = 2 \cdot f(x)$ 
  \end{itemize}
$(*)${ \small Man kann das auch mit einer Intervallschachtelung machen: hierzu konstruieren wir eine Folge $\{ a_n\}$ rationaler Zahlen, $a_n\in \mathbb Q$, die die Zahl $ 2^{-\frac14}$ von oben und von unten beschr�nkt. Die Intervallschachtelung basiert auf der Idee, dass die Anwendung einer Potenzfunktion auf positive Zahlen eine �quivalenzrelation ist. Schauen wir uns ein paar Schritte beispielhaft an: es sei $x:= 2^{\frac14}$. Das einzige, was wir �ber $x$ wissen, ist, dass $x^4=2$ ist.  
	\begin{eqnarray*}
		\begin{array}{lll}
			1. \mathrm{Intervall:} \;\; &\left( 1\right)^4 = 1 < x^4=2 < \left( 2\right)^4 = 16 &\Rightarrow x \in [1,2] \\[1ex]
			2. \mathrm{Intervall:} \;\; &\left( 1\right)^4 = 1 < x^4=2 < \left( \frac32\right)^4 \approx 5,06 &\Rightarrow x \in [1,\frac32] \\[1ex]
			3. \mathrm{Intervall:} \;\; &\left( 1\right)^4 = 1 < x^4=2 < \left( \frac54\right)^4 \approx 2,44 &\Rightarrow  x\in [1,\frac32] \\[1ex] 	
			4. \mathrm{Intervall:} \;\; &\left( \frac98\right)^4 = 1,6 < x^4=2 < \left( \frac54\right)^4 \approx 2,44 &\Rightarrow  x\in [\frac98,\frac54] \\[1ex]
			5. \mathrm{Intervall:} & \dots&  	
			  \end{array}
		  \end{eqnarray*}
Eigentlich sollten wir den Kehrwert approximieren. Es gilt nach dem $4.$ Schritt entsprechend $\frac1x \in [\frac45,\frac89]$ mit  $\frac45 = 0,8 < \frac1x < \frac89 = 0,\bar 8$.
}
\end{exercisebox}

\begin{exercisebox}[Rechenregeln]
	\begin{itemize}
		\item[(a)] Vereinfachen Sie die folgenden Ausdr�cke ohne Taschenrechner:
			\[ \mathrm{lg}(4)+2\mathrm{lg}(5), \; e^{5\ln(2)}, \; \mathrm{lg}(3000)- \mathrm{lg}(3)\,.\]
		\item[(b)] In folgender Umformung ist ein Fehler - finden Sie ihn?
			\[ e^{0.5(\ln x )^2} = \left( e^{(\ln x)^2}\right)^{0.5} = e^{(\ln x)^{2\cdot 0.5}} = e^{(\ln x)} =  x\,.\]
	\end{itemize}

	\vspace{0.3cm}
	\noindent{\bf L�sung:}\newline
  \begin{itemize}
	  \item[(a)] \begin{eqnarray*} 
			  \mathrm{lg}(4)+2\mathrm{lg}(5) &=& \mathrm{lg}(4)+\mathrm{lg}(5^2) =  = \mathrm{lg}4+\mathrm{lg}(25) = \mathrm{lg}(4\cdot 25) = \mathrm{lg}(100) = 2 \\
			  e^{5\ln(2)}&=& e^{\ln(2^5)} = 2^5 = 32 \\
			  \mathrm{lg}(3000) - \mathrm{lg}(3) &=& \mathrm{lg}(3000/3) = \mathrm{lg}(1000) = 3
		  \end{eqnarray*}
	  \item[(b)]  \( \displaystyle{  \left( e^{(\ln x)^2}\right)^{0.5} \not= e^{(\ln x)^{2\cdot 0.5}} }\)\,. 
  \end{itemize}
\end{exercisebox}

\begin{exercisebox}[Gleichungen]
	L�sen Sie die folgenden Gleichungen:
	\begin{eqnarray*}
		\begin{array}{ll} 
			(a)\; \displaystyle{e^{x^2 - 2x}}=1 \quad & (b)\; e^x+2\cdot e^{-x} = 3 \\ (c)\; \ln\left( \sqrt{x}\right) + 1,5\cdot \ln(x) = \ln(2x) \quad &(d)\; \left( \mathrm{lg}(x)\right)^2 - \mathrm{lg}(x) = 2 
		\end{array}
	\end{eqnarray*}

	\vspace{0.3cm}
	\noindent{\bf L�sung:}\newline
	\begin{itemize}
		\item[(a)] Es gilt \begin{eqnarray*}
				\begin{array}{crcll}
						& e^{x^2-2x} &=& 1  \; & |\mathrm{ln}\\
						\Leftrightarrow & x^2-2x &=& 0 \; &|+1 \\
					\Leftrightarrow & (x-1)^2 &=& 1 & \end{array}
				\end{eqnarray*}
				Damit ist $\mathbb L = \{ 2, 0\}$.
			\item[(b)] Substituiere $u=e^x$. Dann gilt
				\begin{eqnarray*}
				\begin{array}{crcll}
					& e^x+2\cdot e^{-x} &=& 3 \; &||\mathrm{ln}(\cdot) \\
						\Leftrightarrow & u+2/u &=& 3 \; &||\cdot u, \; u\not=0\\
						\Leftrightarrow & u^2+2 &=& 3u\;&||-3u\\
						\Leftrightarrow & u^2-3u+2 &=& 0&
					\end{array}
				\end{eqnarray*}
				Mit Hilfe der allgemeinen L�sungsformel erhalten wir $u=1\lor u=2$. R�cksubstitution ergibt als L�sungen f�r $x$: $x=\ln(1)=0$ oder $x=\ln(2)$, also $\mathbb L = {0,\ln(2)}$.
			\item[(c)] Es gilt 
				\begin{eqnarray*}
				\begin{array}{crcll}
						& \ln\left( \sqrt{x}\right) + 1,5\cdot \ln(x) &=& \ln(2x) \; & \\
						\Leftrightarrow & \frac12 \ln(x) + 1,5\ln(x) &=& \ln(2) + \ln(x) &\\
						\Leftrightarrow & \left( \frac12 + 1,5 -1 \right) \ln(x) &=& \ln(2) &\\
						\Leftrightarrow & \ln(x) &=& \ln(2) & |e^{\cdot} \\
						\Leftrightarrow & x&=& 2 & 
					\end{array}
				\end{eqnarray*}
			\item[(d)] Substitution $u=\lg(x)$ f�rht auf $ u^2-u=2 $ und diese quadratische Gleichung besitzt die L�sungen $u=2$ oder $u=-1$ - also  $x=100$ oder $x=0,1$.
		\end{itemize}
\end{exercisebox}

\begin{exercisebox}[Anwendung: Radioaktiver Zerfall]
	Eine radioaktive Substanz zerf�llt nach dem Gesetz \[ n(t) = n_0 \cdot e^{-\lambda t} \,.\]
	mit $n_0, \lambda \in \mathbb R^+$. Die halbwertszeit $\tau$ ist definiert durch $n(\tau) = n(0)/2$. 
	\begin{itemize}
		\item[(a)] Bestimmen Sie die Umkehrfunktion mit Definitionsbereich, Wertebereich und Abbildungsvorschrift. 
		\item[(b)] Geben Sie eine allgemeine Formel f�r $\tau$ an. 
		\item[(c)] Berechnen Sie die Halbwertszeitf�r Radon mit $\lambda = 2,0974\cdot 10^{-6}s^{-1}\,.$
	\end{itemize}

	\vspace{0.3cm}
	\noindent{\bf L�sung:}\newline
	\begin{itemize}
		\item[(a)] Es gilt \[ n: \mathbb R \to \mathbb R^+, \quad t\mapsto n_0 \cdot e^{-\lambda t} \]
			F�r die Umkehrfunktion l�se 
			\begin{eqnarray*}
				\begin{array}{clcll}
					& n_0\cdot e^{-\lambda t} &=& n \; & |:n_0, \; n_0\not=0 \\
					\Leftrightarrow & e^{-\lambda t} &=& n/n_0 \; &| \ln(\cdot),\; n>0\\
					\Leftrightarrow & - \lambda t &=& \ln(n/n_0)/ \lambda &|:(-\lambda), \; \lambda \not=0\\
					\Leftrightarrow & t = -\ln(n/n_0)/\lambda &\\
					\Leftrightarrow & t &=& \ln(n_0/n)/\lambda 
				\end{array}
			\end{eqnarray*}
			Die Funktion ist f�r $n>0$ umkehrbar und die Umkehrfunktion lautet 
			\[ n^{-1}:\mathbb R^+ \to \mathbb R, \quad n\mapsto \ln(n_0/n)/\lambda\,.\]
		\item[(b)] Es gilt $ \tau = n^{-1}\left( \displaystyle{\frac{n_0}{n_0/2}}\right) / \lambda = \ln(2/\lambda)$.
		\item[(c)] Es gilt $\tau = \displaystyle{ \frac{\ln(2)}{2,0974\cdot 10^{-6}\,\mathrm{s}}} \approx 330,479\cdot 10^3\,\mathrm{s} \approx 3,824 \,\mathrm{d}\,.$
	\end{itemize}
\end{exercisebox}

