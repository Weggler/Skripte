\section*{Rationale Funktionen Aufgaben mit L�sungen}\label{MA1-11-Aufgaben-Loesungen}

\ifdefined\MOODLE 
	Regeln und Beispiele, die zum Verst�ndnis der L�sungswege hilfreich sind, finden Sie in Abschnitt \ref{MA1-11}.
\else
	Regeln und Beispiele, die zum Verst�ndnis der L�sungswege hilfreich sind, finden Sie im Skript.
\fi 

\begin{exercisebox}[Abbildungsvorschrift] %LS p186 -Aufgabe 4
	Geben Sie eine gebrochenrationale Funktion an mit 
	\begin{itemize}
		\item[(a)] Nullstelle $1$ $\Rightarrow \; x\mapsto \displaystyle{\frac{x-1}{x}}$
		\item[(b)] Polstelle $3$ mit Vorzeichenwechsel $\Rightarrow \; x\mapsto \displaystyle{\frac{1}{x-3}}$
		\item[(c)] Polstelle $3$ ohne Vorzeichenwechsel $\Rightarrow \; x\mapsto \displaystyle{\frac{1}{(x-3)^2}}$
		\item[(d)] Nullstelle $1$ und Polstelle $3$ ohne Vorzeichenwechsel $\Rightarrow \; x\mapsto \displaystyle{\frac{x-1}{(x-3)^2}}$
		\item[(e)] Nullstellen $2$ und $3$, Polstelle $4$ mit Vorzeichenwechsel $\Rightarrow \; x\mapsto \displaystyle{\frac{(x-2)\cdot (x-3)}{x-4}}$
		\item[(f)] Nullstelle $-1$, Polstelle $-3$ mit Vorzeichenwechsel, Polstelle $4$ ohne Vorzeichenwechsel $\Rightarrow \; x\mapsto \displaystyle{\frac{(x+1)}{(x+3)\cdot (x-4)^2}}$
	\end{itemize}
\end{exercisebox}

\begin{exercisebox}[Asymptoten f�r \(|x|\to \infty\)] %LS p.31 Aufgabe 3
  Untersuchen Sie das Verhalten der Funktion f�r $x\to \pm\infty$. Geben Sie gegebenenfalls die Gleichung der waagerechten Asymptote an.
	\begin{eqnarray*} \begin{array}{lll}
		(a)\; f(x)= \frac{7}{x}  &\quad 
     (b)\; f(x)= \dfrac{-3x^3+4x+16}{4x^2} &\quad 
     (c)\; f(x)= \dfrac{x^3+1}{x-1} \\
     (d)\; f(x)= \frac{2}{x}+\sqrt{x} &\quad 
     (e)\; f(x)= \frac{2}{(x-1)^2} &\quad 
     (f)\; f(x)= \frac{4}{\sqrt{x-2}} 
     \end{array}
     \end{eqnarray*}

	\vspace{0.3cm}
	\noindent{\bf L�sung:}\newline
	\begin{itemize}
		\item[(a)] $\displaystyle{\frac{7}{x} \to 0}$ f�r ${x\to \pm \infty} \; \Rightarrow \; y= 0$ ist waagerechte Asymptote,
	%\item[(b)] $\displaystyle{\frac{5}{3x-1} \to 0}$ f�r  ${x\to \pm\infty}0 \; \Rightarrow \; y= 0$ ist waagerechte Asymptote, 
	%\item[(c)] $\displaystyle{\frac{2}{x-2}-3} \to -3$ f�r  ${x \to \pm\infty} -3 \; \Rightarrow \; y=-3$ ist waagerechte Asymptote,
		\item[(b)]  \(\dfrac{-3x^3+4x+16}{4x^2} = -\dfrac34 x + \dfrac{4+x}{x^2} \to -\dfrac34 x\)  f�r \(x\to \pm\infty \; \Rightarrow \; y(x) = -\dfrac34 x\) ist die Asymptotengerade,
		\item[(c)] \(\dfrac{x^3+1}{x-1} = x^2+x+1 + \dfrac{2}{x-1} \to x^2+x+1 \) f�r \(x\to \pm \infty\; \Rightarrow \; y(x) = x^2+x+1\) ist die asymptotische Kurve,
		\item[(d)] $\displaystyle{\frac{2}{x}+\sqrt{x}} \to +\infty $ f�r  ${{x\to +\infty}} \; \Rightarrow$ keine waagerechte Asymptote,
		\item[(e)] $\displaystyle{\frac{2}{(x-1)^2}} \to 0 $ f�r  ${x\to \pm\infty} 0 \; \Rightarrow \; y=0$ ist waagerechte Asymptote.
		\item[(f)] $\displaystyle{\frac{4}{\sqrt{x-2}}} \to 0$ f�r  ${x\to +\infty} 0 \; \Rightarrow \; y=0$ ist waagerechte Asymptote.
	\end{itemize}
    %\item $f(x)= 2- \frac{3}{\sqrt{x+3}}$
    %\item $f(x)= \frac{1}{x}+2\sin(x)$
\end{exercisebox}


%\begin{exercisebox}[Wolfram Alpha] %p31, Aufgabe 5 -- Pingo
%  Wie kann man mithilfe von Wolfram Alpha zu Vermutungen �ber den Grenzwert f�r $x\to\pm \infty$ gelangen? Geben Sie f�r $f$ solche Vermutungen an.
%  \begin{enumerate}
%    \item $f(x)= \frac{x}{2^x}$
%    \item $f(x)= \frac{x^2}{2^x}$
%    \item $f(x)= \frac{x^3}{1,5^x}$
%    \item $f(x)= \frac{x+2}{x^2-2}$
%    \item $f(x)= \frac{2x-1}{x+1}$
%    \item $f(x)= \frac{10\sin(x)}{x}$
%    \item $f(x)= 2^{-x} \cdot \sin(x)$
%    \item $f(x)= x^{\frac{1}{x}}, \; x>0$
%  \end{enumerate}
%\end{exercisebox}

\begin{exercisebox}[Vertikale Asymptoten] %LS p33, Aufgabe 6
  Untersuchen Sie das Verhalten von $f$ bei Ann�herung an die Definitionsl�cke. Geben Sie die Geichung der senkrechten Asymptote an.
	\begin{eqnarray*} \begin{array}{lll}
			(a)\; f(x) = \frac{2}{x} & \quad 
			(b)\; f(x) = -\frac{2}{x^2} &\quad 
			(c)\; f(x) = \frac{1}{x-4} \\[1ex]
			(d) \; f(x) = \frac{2}{4-x} &\quad 
     (e)\; f(x) = 1- \frac{1}{x} &\quad 
     (f) \; f(x) = \frac{3}{(x-1)^2} 
     \end{array}
     \end{eqnarray*}

	\vspace{0.3cm}
	\noindent{\bf L�sung:}\newline
	\begin{itemize}
		\item[(a)] F�r $x\to 0$ und $x>0$ gilt: $\frac2x \to +\infty$, f�r $x\to 0$ und $x<0$ gilt: $\frac2x \to -\infty$, $x=0$ ist senkrechte Asymptote,
		\item[(b)] F�r $x\to 0$ und $x>0$ gilt: $-\frac{2}{x^2} \to -\infty$, f�r $x\to 0$ und $x<0$ gilt: $-\frac{2}{x^2} \to -\infty$, $x=0$ ist senkrechte Asymptote,
		\item[(c)] F�r $x\to 4$ und $x>4$ gilt: $\frac{1}{4-x} \to -\infty$, f�r $x\to 4$ und $x<4$ gilt: $\frac{1}{4-x} \to +\infty$, $x=4$ ist senkrechte Asymptote,
		\item[(d)] F�r $x\to 4$ und $x>4$ gilt: $\frac{2}{4-x} \to -\infty$, f�r $x\to 4$ und $x<4$ gilt: $\frac{2}{4-x} \to +\infty$, $x=4$ ist senkrechte Asymptote,
		\item[(e)] F�r $x\to 0$ und $x>0$ gilt: $1-\frac1x \to -\infty$, f�r $x\to 0$ und $x<0$ gilt: $1-\frac1x \to +\infty$, $x=0$ ist senkrechte Asymptote,
		\item[(f)] F�r $x\to 1$ und $x>1$ gilt: $\frac{3}{(x-1)^2} \to +\infty$, f�r $x\to 1$ und $x<1$ gilt: $\frac{3}{(x-1)^2} \to +\infty$, $x=1$ ist senkrechte Asymptote,
	\end{itemize}
\end{exercisebox}

\begin{exercisebox}[Asymptoten] %LS p. 189, Aufgabe 2
  Geben Sie die Gleichungen aller Asymptoten an
	\begin{eqnarray*} \begin{array}{lll}
			(a)\; f(x) = \frac{4}{3x^2} \quad \quad  & %a) 
	  (b)\; f(x) = \frac{2x+1}{x^2+3x} \quad\quad\quad & % f) 
	  (c)\; f(x) = \frac{x^4-x^2-1}{x^3-1} % p)
     \end{array}
  \end{eqnarray*}

	\vspace{0.3cm}
	\noindent{\bf L�sung:}\newline
	\begin{itemize}
		\item[(a)] vertikale Asymptote an der Stelle $x=0$, horizontale Asymptotengerade $g: x\mapsto 0$
		\item[(b)] vertikale Asymptoten an den Stllen $x=0$ und $x=-3$, horizontale Asymptotengerade $g: x\mapsto 0$
		\item[(c)] vertikale Asymptote an der Stelle $x=1$, horizontale Asymptotengerade $g: x\mapsto x$
	\end{itemize}
\end{exercisebox}


\begin{exercisebox}[Qualitativer Graph] %p. 186 Aufgabe 2 a)-f)
	Ermitteln Sie von den folgenden Funktionen die 
	\begin{itemize}
		\item die Definitionsmenge,
		\item die Achsenabschnitte,
		\item die Nullstellen,
		\item die Polstellen inklusive der Analyse des Verhaltens von $f$ an jeder Polstelle und 
		\item fertigen Sie mit diesen Informationen eine qualitative Skizze des jeweiligen Funktionsgraphen an. 
	\end{itemize}
	\begin{eqnarray*} \begin{array}{lll}
			(a)\; f(x) = \frac{3x-1}{x-1} \quad & 
			(b)\; f(x) = \frac{x^2+x}{x+1} \quad & 
			(c)\; f(x) = \frac{x^2-9}{(x-3)^2} 
			%\\[1ex]
			%(d)\; f(x) = \frac{x^2-2x-15}{x-5} \quad &
     %(e)\; f(x) = \frac{3x-3}{x-1} \quad &
     %(f)\; f(x) = \frac{x^2+5x+2}{(x+1)^2} 
     \end{array}
  \end{eqnarray*}

	\vspace{0.3cm}
	\noindent{\bf L�sung:}\newline
	\begin{itemize}
		\item[(a)] $D=\mathbb R\setminus\{1\}$, Nullstelle $x_1=\frac13 \Rightarrow N(\frac13|0)$, Polstelle: $x_2 = 1$ mit VZW, $y$-Achsenabschnitt f�r $y=1$
		\item[(b)] $D=\mathbb R\setminus\{-1\}$, Nullstelle $x_1=0 \Rightarrow N(0|0)$, \(x=-1\) ist keine Polstelle, da der Linearfaktor \( (x+1) \) in Z�hler und Nenner auftaucht, $y$-Achsenabschnitt f�r $y=0$
		\item[(c)] $D=\mathbb R\setminus\{3\}$, Nullstelle \(x=-3\), Polstelle: $x_1 = -3$ mit VZW, $y$-Achsenabschnitt f�r $y=-1$
	\end{itemize}
\end{exercisebox}

%\begin{exercisebox}[Vermischte Aufgaben] -- Aufgabensammlung p. 38, Aufg. 8
%  Gegeben ist die Funktion $f$ mit $f(x) = \frac{1}{2(x+2)}+1$.
%  \begin{enumerate}
%    \item Bestimmen Sie die maximale Definitionsmenge von $f$.
%    \item Untersuchen Sie den Graphen von $f$ auf waagerechte und senkrechte Asymptoten.
%    \item Wo schneidet der Graph die $x$-Achse?
%    \item Skizzieren Sie den Graphen von $f$ mit seinen Asymptoten.
%    \item Entnehmen Sie der Sikzze eine Vermutung �ber die Symmetrie des Graphen. Beweisen Sie die Vermutung rechnerisch.
%  \end{enumerate}
%\end{exercisebox}

%\begin{exercisebox}[Asymptoten horizontal und vertikal] --> Aufgabensammlung p. 33, Aufg. 9
%  Gegeben ist die Funktion $f$ mit $f(x) = \frac{1}{(x+1)(x-2)}$
%  \begin{enumerate}
%    \item Geben Sie den maximalen Definitionsbereich der Funktion $f$ an.
%    \item Untersuchen Sie das Verhalten von $f$ bei Ann�herung an die Definitionsl�cken.
%    \item Untersuche Sie das Verhalten von $f$ f�r $x\to \pm \infty$
%    \item Skizzieren Sie den Graphen von $f$ mithilfe der Asymptoten.
%  \end{enumerate}
%\end{exercisebox}
