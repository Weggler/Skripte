
\ifdefined\MOODLE 
\section*{Numerische N�herung Aufgaben}\label{MA1-26-Aufgaben}

	Regeln und Beispiele, die zur Bearbeitung der Aufgaben hilfreich sind, finden Sie in \ref{MA1-26} und, falls Sie nicht weiterkommen, schauen Sie \hyperref[MA1-26-Aufgaben-Loesungen]{hier}.
\else
\section{Numerische N�herung Aufgaben}\label{MA1-26-Aufgaben}

	Regeln und Beispiele, die zur Bearbeitung der Aufgaben hilfreich sind, finden Sie im Skript und, falls Sie nicht weiterkommen, schauen Sie \hyperref[MA1-26-Aufgaben-Loesungen]{hier}.

\fi 

\begin{exercisebox}[Zentrale Differenz und zweite Differenz]
	\begin{itemize}
		\item[(a)] Zeigen Sie, dass sich die erste Ableitung von Polynomen vom Grad $2$ mit der zentralen Differenz exakt berechnen l��t.
		\item[(b)] Zeigen Sie, dass sich die zweite Ableitung von Polynomen vom Grad $4$ mit der zweiten Differenz exakt berechnen l��t.
	\end{itemize}
\end{exercisebox}

\begin{exercisebox}[Bestapproximation]
	Gegeben seien Messungen $x_1, x_2, \cdots, x_n\in\mathbb R$, $n\in \mathbb N$ einer Gr��e $x\in \mathbb R$. Gesucht ist eine Approximation $\tilde x\in \mathbb R$, deren Abstand zu allen Messwerten gleichzeitig minimal ist, also 
	\[ \sum\limits_{k=1}^n |x- x_k|^2 \stackrel{!}{=} \mathrm{min}\,.\]
		Bestimmen Sie eine Formel f�r $\tilde x$. 
\end{exercisebox} 

\begin{exercisebox}[Taylorentwicklung]
	Gegeben sei die Funktion $f: x\mapsto \cos(x)$. Bestimmen Sie die Taylorentwicklung $T_7$ von $f$ an der Stelle $x_0=0$ und sch�tzen Sie den Betrag des Lagrangeschen Restglieds $R_8$ im Intervall $x\in I= [ -\pi/6, \pi/6]$ ab. 
\end{exercisebox} 

\begin{exercisebox}[Interpolation (Wiederholung)]
Gegeben sind die Punkte 
\begin{center}
	\begin{tabular}{c|cccc}
		$i$ & $0$ & $1$ & $2$ & $3$ \\
		\hline
		$x_i$ & $-3$ & $-1$ & $0$ & $2$ \\
		\hline
		$y_i$ & $-3$ & $1$ & $0$ & $82$ \\
	\end{tabular}
\end{center}
\begin{itemize}
	\item[(a)]  Bestimmen Sie den Grad $n$ des Interpolationspolynoms. 
	\item[(b)]  Bestimmen Sie das lineare Gleichungssystem f�r die Koeffizienten des Interpolationspolynoms in der Darstellung \[ p(x) =a_nx^n+\cdots +a_1 x+a_0\] und l�sen Sie es.
	%\item[(c)]  Bestimmen Sie den Wert des Polynoms an der Zwischenstelle $\tilde x= 1$.
	%\item[(d)]  Bestimmen Sie das lineare Gleichungssystem f�r die Koeffizienten des Interpolationspolynoms in der Darstellung \[ q(x) =c_0+c_1(x-x_0) +c_2(x-x_0)(x-x_1) +\cdots +c_n(x-x_0)\cdots(x-x_{n-1})\] und l�sen Sie es mit Matlab.
	%\item[(e)]  Berechnen Sie den Wert des Polynoms $q$ an der Zwischenstelle $\tilde x= 1$. Wie k�nnnen Sie den Wert effizient berechnen?
\end{itemize}
\end{exercisebox}


\begin{exercisebox}[Interpolation (Wiederholung)]
	Eine Funktion der Form \[ f: x \mapsto \frac{a_0}{2} + a_1 \cos(x) + b_1 \sin(x) \,,\] soll durch die Punkte $(x_i, y_i), i=1,2,3$ mit \[ (0|1), \, (\pi/6|-1),\, (\pi/2|-3)\] gelegt werden. Bestimmen Sie ein Gleichungssystem f�r die Koeffizienten $a_0, a_1, b_1$ und l�sen Sie es.

\end{exercisebox} 

%\begin{exercisebox}[Lineare Regression (Wiederholung)]
%  F�r Wohnungen verlangt eine Baufirma die angegebenen Kaufpreise. Wie lautet die lineare Regressionsgerade? Berechnen Sie den Kaufpreis f�r eine $88\mathrm{m}^2$ gro�e Wohnung.
%  \begin{center}
%  \begin{tabular}{c|cccc}
%    Wohnfl�che in $\mathrm{m}^2$ & $40$ & $60$ & $80$ & $100$\\ 
%    \hline
%    Preis in $\mathrm{T} $\euro & $120$ & $165$ & $210$ & $255$ 
%  \end{tabular}
%  \end{center}
%
%  %\hspace{0.3cm}
%  %\newline
%  %{\bf L�sung:}
%  %F�hren wir die Variable $x$ zur Bezeichnung der Wohnfl�che und die Variable $y$ zur Bezeichnung des Preises ein. Es gilt \[ \frac{\Delta y}{\Delta x} = \frac{y_{i+1}- y_i}{x_{i+1} - x_i} = \frac{ 4}{9 }\] und daas hei�t, dass der Preis linear von der Gr��e der Wohnung abh�ngt. Damit ist die Ausgleichsgerade die interpolierende Gerade mit der Gleichung \[ g(x) = \frac49 (x-40) + 120 \,.\]
%\end{exercisebox} 

