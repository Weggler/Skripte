\section{Einf�hrung}

i\subsection{Aufgaben}
Brainstorming zu den Themen der Multiple Choices
\begin{enumerate}
	\item Funktionale Abh�ngigkeiten beschreiben
	\item Differenzierbarkeit
	\item Stetigkeit
	\item Extrema (lokal und global)
	\item Wendepunkte
	\item Nullstellen 
	\item Defintionsl�cke
	\item Vektoren 
	\item Definitionsbereich und Bild
\end{enumerate}
Aufgaben zu
\begin{enumerate}
	\item Grundrechenarten trainieren
	\item Binomische Formeln
	\item L�sung quadratischer Gleichung
	\item Betragsfunktion
	\item Nenner und Z�hler analysieren
	\item Division durch Null nicht erlaubt
	\item L�sungsmenge angeben
\end{enumerate}

%\begin{exabox}[Terme vereinfachen]
%	Vereinfachen Sie so weit wie m�glich:
%	\begin{itemize}
%		%\item $b-(a+2-(c+d-(3-a))+b)$
%		%\item $(a+b+c)(-a+b+c)+(2c+b-a)^2-2(b+c)^2$
%		%\item $\displaystyle{\frac{3-x}{x^2-4} + \frac{2x+1}{x^2-4x+4}}$
%		%\item $(a+b)(a-c(b-c))-(a-b(c-a))(b-c)$
%		%\item $a+c-(d+a+2-(b+c-(-d+c)$
%		%\item $\displaystyle{\frac{3}{\displaystyle{\frac{x+3}{x-2}}} + \frac{\displaystyle{\frac{x-102}{x-4}}}{x+3}}$
%	\end{itemize}
%\end{exabox}

%\begin{exabox}[MC]
%	Welche Ausdr�cke stimmen �berein?
%	\begin{itemize}
%		\item $4a^2+24ab+9b^2$
%		\item $(2a+6b)^2-27b^2$
%		\item $(2a+3b)^2+12ab$
%		\item $(3b+4a)^2-12a^2$
%	\end{itemize}
%\end{exabox}

%\begin{exabox}[Faktorisieren]
%	Schreiben Sie die folgenden Ausdr�cke in der Form $(x+p)(x+q)$, wobei $p,q\in \mathbb N_0$
%	\begin{itemize}
%		\item $x^2+x-56$
%		\item $x^2-1024$
%	\end{itemize}
%\end{exabox}


\subsection{Multiple Choice}
Anhand von Bildern:
\begin{enumerate}
	\item Lineares Wachstum 
	\item Quadratisches Wachstum 
	\item Exponentielles Wachstum 
	\item Stetigkeit
	\item Differenzierbarkeit
	\item Extrema und Wendepunkte
	\item Nullstellen
\end{enumerate}

\subsection{Details und Extras}
\begin{mybox}[Beweise f�hren]
	\begin{enumerate}
		\item Beweis durch Vollst�ndige Induktion (Folgen $n\to n+1$)
		\item Zeige Behauptung explizit (im Allgemeinen unm�chlich da parametrisiert �ber unendliche Dimensionalen Vektorraum)
		\item Widerspruchsbeweis (Drehe die Behauptung um und finde ein Gegenbeispiel)
		\item Nachrechnen der Definition ( manchmal sehr abstrakt zum Beispiel Beiweis der Vollst�ndigkeit eines Raumes, Stetigkeit)
	\end{enumerate}
\end{mybox}

\begin{mybox}[Typen von Zahlen]
	\begin{itemize}
		\item nat�rliche Zahlen
		\item ganze Zahlen
		\item rationale Zahlen
		\item reelle Zahlen
		\item komplexe Zahlen
	\end{itemize}
\end{mybox}
			
\begin{mybox}[Grundlagen Vektorrechnung]
	\begin{itemize}
		\item Basis
		\item Vektoren
		\item Lineare Abbildungen von Vektoren
	\end{itemize}
\end{mybox}
			
\begin{mybox}[Grundlagen Differentialrechnung]
	\begin{itemize}
		\item Wiederholung spezieller Funktionen (Trainieren!)
		\item Analyse der analytischen Eigenschaften von Funktionen
	\end{itemize}
\end{mybox}
			
\begin{mybox}[Numerische Verfahren]
	\begin{itemize}
		\item Newton-Verfahren (Nullstellensuche)
		\item Polynomiale N�herungen 
		\item L�sen linearer Gleichungssysteme
	\end{itemize}
\end{mybox}
			
\begin{mybox}[Typen von Gleichungen]
	\begin{itemize}
		\item lineare Gleichungen 
		\item quadratische Gleichungen (L�sungsformel)
		\item allgemeine algebraische Gleichungen 
		\item Systeme linearer Gleichungen (Matrizen)
		\item lineare Differentialgleichungen (Differenzenverfahren)
		\item Integralgleichungen (Matrizen)
		\item partielle Differentialgleichungen (Matrizen)
	\end{itemize}
\end{mybox}
			

