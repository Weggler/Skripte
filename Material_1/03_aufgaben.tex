\ifdefined\MOODLE 
\section*{�quivalenzumformungen Aufgaben}\label{MA1-03-Aufgaben}

Regeln und Beispiele, die zur Bearbeitung der Aufgaben hilfreich sind, finden Sie in Abschnitt \ref{MA1-03} und, falls Sie nicht weiterkommen, schauen Sie \hyperref[MA1-03-Aufgaben-Loesungen]{hier}. 

\else
\section{�quivalenzumformungen Aufgaben}\label{MA1-03-Aufgaben}

Regeln und Beispiele, die zur Bearbeitung der Aufgaben hilfreich sind, finden Sie im Skript und, falls Sie nicht weiterkommen, schauen Sie \hyperref[MA1-03-Aufgaben-Loesungen]{hier}. 

\fi 

\begin{theorembox}{Allgemeine L�sungsformel}{ }
	Zur L�sung von Gleichungen \(ax^2+bx+c=0\) mit \(a,b,c\in \mathbb R\) berechnet man die sogenannte Diskriminante \(D=b^2-4ac\). F�r \(D\geq 0\) besitzt die Gleichung zwei reelle L�sungen. Sie lauten
	\[ x_{1,2} = \frac{-b\pm \sqrt{b^2-4ac} }{2a}\,.\]
	Mit den L�sungen \(x_{1,2}\) kann die Linearfaktorzerlegung des quadratischen Terms \(ax^2+bx+c\) angegebenen werden: 
	\[ ax^2+bx+c = a\cdot (x-x_1)\cdot (x-x_2)\,.\]
\end{theorembox}

\begin{exercisebox}[Gleichungen und Ungleichungen]
	Bestimmen Sie die L�sungsmengen der folgenden Gleichungen und Ungleichungen:
	\begin{eqnarray*}
	\begin{array}{ll}
		(a)\; \displaystyle{\frac{x+52}{x+2}} = 11  & \quad \quad (b)\; \displaystyle{\frac{x^2+2x-3}{x+3} = 1} \\[2ex] 
		(c)\; \displaystyle{\frac{(n+2)!}{n!}} = 72, n\in\mathbb N_0 &\quad\quad (d)\; \displaystyle{|2x-6|} = |4-5x|  \\[1ex] 
		(e)\; 3x-3 < 7 -2x & \quad\quad (f)\; \displaystyle{\frac{3x-5}{x-2}} \leq 3 \\[2ex] 
		(g)\; |6x+1|<|2x-1| &\quad\quad  (h)\; \displaystyle{\Big|\frac{3x+2}{x-2} \Big|} < 1 
	\end{array}
	\end{eqnarray*}
\end{exercisebox}

\begin{exercisebox}[Gleichungssystem]
	Bestimmen Sie jeweils die L�sungsmenge der folgenden Gleichungssysteme
		\begin{eqnarray*}
			\begin{array}{lll}
			
				(a)\; \begin{cases} 
		2y-4x &= -6 \\
		5y-4x &= -21
	\end{cases} &\quad  
	(b)\;\begin{cases}
		2x+y &= 7 \\
		2x+y &= 7
	\end{cases}  & \quad 
	(c)\;	\begin{cases}
		x^2+y^2 &= 10 \\
		x+y &= 4
	\end{cases}
\end{array}
\end{eqnarray*}
\end{exercisebox}

\begin{exercisebox}[Gleichungssystem mit Parameter]
	L�sen Sie das folgende lineare Gleichungssystem f�r \(p\in \mathbb R\)
		\begin{eqnarray*}
			\begin{array}{rcrcr}
		x_1 &+& p x_2 &=& 1 \\
		(p-1)x_1 &+& x_2 &=& -1
	\end{array}
	\end{eqnarray*}
\end{exercisebox}

\begin{exercisebox}[Brainteaser]
	In einer Familie hat jeder Sohn gleich viele Schwestern und Br�der. Jede Tochter hat aber doppelt so viele Br�der wie Schwestern. Wie viele S�hne und T�chter hat die Familie?
\end{exercisebox}

\begin{exercisebox}[Brainteaser]
	Zu einem Fest auf dem Land fahren mehrere Pferdewagen mit der jeweils gleichen Anzahl an Personen. Auf halbem Weg fallen zehn Wagen aus, sodass jeder der �brigen Wagen eine weitere Person aufnehmen muss. Vor Antritt des R�ckwegs fallen weitere \(15\) Wagen aus, was zur Folge hat, dass in jedem Wagen drei Personen mehr sind als bei der Abfahrt am Morgen. Wie viele Personen nahmen an dem Fest teil?
\end{exercisebox}
