
\ifdefined\MOODLE 
\section*{Stetigkeit Aufgaben}\label{MA1-23-Aufgaben}

	Regeln und Beispiele, die zur Bearbeitung der Aufgaben hilfreich sind, finden Sie in \ref{MA1-23} und, falls Sie nicht weiterkommen, schauen Sie \hyperref[MA1-23-Aufgaben-Loesungen]{hier}.
\else
\section{Stetigkeit Aufgaben}\label{MA1-23-Aufgaben}

	Regeln und Beispiele, die zur Bearbeitung der Aufgaben hilfreich sind, finden Sie im Skript und, falls Sie nicht weiterkommen, schauen Sie \hyperref[MA1-23-Aufgaben-Loesungen]{hier}.
\fi 

\begin{exercisebox}[Stetigkeit]
	\begin{itemize}
		\item[(a)] Es sei $f:D \subseteq \mathbb R\to \mathbb R$ und $x_0 \in D$. Was bedeutet die folgende Aussage:
		\[ \exists \varepsilon > 0 \forall \delta > 0 \exists x\in D : |x-x_0| < \delta \land |f(x) - f(x_0)|\geq \varepsilon\,. \]
	\item[(b)] Betrachten Sie die Funktion $f:\mathbb R\to \mathbb R$ mit $f: x \mapsto \frac13 x$. Gibt es eine Stelle $x_0 \in \mathbb R$, an der die Bedingung unter $(a)$ erf�llt ist?
\end{itemize}
\end{exercisebox}


\begin{exercisebox}[Stetigkeit]
	Untersuchen Sie die Funktion $f$ mit \[f: x \mapsto \begin{cases} 1,  &\quad x \in \mathbb Q\,, \\ 0, &\quad x \in \mathbb R\setminus \mathbb Q \,,\end{cases}\] auf Stetigkeit an einer beliebigen Stelle $x_0$.
\end{exercisebox} 


\begin{exercisebox}[Stetigkeit]
	F�r welche $x\in [0,\infty)$ ist $f$ stetig?
  \begin{eqnarray*}
	  f: x\mapsto \begin{cases}
      1 \quad &x>1\\
      \frac{1}{|x|} \quad & 0 <x\leq 1\\
      0 \quad & x\leq 0\\
    \end{cases}
  \end{eqnarray*}
\end{exercisebox} 

\begin{exercisebox}[Stetigkeit]
  Bestimmen Sie $t\in \mathbb R$ so, dass die Funktion $f$ an der Stelle $x_0$ stetig ist.
  \begin{enumerate}
	  \item[(a)] Es sei $x_0=1$ und 
      $
  	  f: x\mapsto \begin{cases}
  	    x+1 \quad &x\leq 1\\
  	    x^2+t \quad & x > 1
	  \end{cases}
	    $
    \item[(b)]  Es sei $x_0 = t$ und 
      $
  	  f: x \mapsto \begin{cases}
  	    x^2-2tx \quad &x\geq t\\
  	    2x-t \quad & x < t
	  \end{cases}
	    $
  \end{enumerate}
\end{exercisebox} 

\begin{exercisebox}[Stetigkeit]
	Bestimmen Sie f�r $f$ den Parameter $a$ so, dass $f$ auf ganz $\mathbb R$ stetig ist:
	\[ f: x\mapsto = \begin{cases} \frac12 x, &\quad x\geq 1 \\2\cdot \exp(a\cdot x), &\quad x< 1 \,. \end{cases} \]
\end{exercisebox}

\begin{exercisebox}[Stetige Fortsetzung]
	Vervollst�ndigen Sie $f$ so, dass $f$ auf ganz $\mathbb R$ stetig ist:
	\[ f: x\mapsto \begin{cases} x+1, &\quad x\leq 1 \\x^2, &\quad x> 2 \,.\end{cases} \]
\end{exercisebox}

%\begin{theorembox}[absolute Konvergenz]
%  absolut konvergene Reihen (Reihenfolge der Glieder egal): wenn die Summe �ber die Betr�ge der Koeffizienten konvergiert.  
%\end{theorembox}
%
%\begin{theorembox}[Potenzreihen stellen Funktionen dar]
%  Konvergenzradius: f�r welche $x$ konvergiert die Potenzreihe?
%\end{theorembox}
%
%\begin{theorembox}[Funktionenr�ume, Betrag, Metrik]
%	Abstandsbegriffe, Energie Physik
%\end{theorembox}
%
%\begin{theorembox}[Lipschitz-Stetigkeit]
%\end{theorembox}

