
\ifdefined\MOODLE 
\section*{Eigenschaften von Funktionen Aufgaben}\label{MA1-05-Aufgaben}
	
	Regeln und Beispiele, die zur Bearbeitung der Aufgaben hilfreich sind, finden Sie in Abschnitt \ref{MA1-05} und, falls Sie nicht weiterkommen, schauen Sie \hyperref[MA1-05-Aufgaben-Loesungen]{hier}.
\else 
\section{Eigenschaften von Funktionen Aufgaben}\label{MA1-05-Aufgaben}
	
	Regeln und Beispiele, die zur Bearbeitung der Aufgaben hilfreich sind, finden Sie im Skript und, falls Sie nicht weiterkommen, schauen Sie \hyperref[MA1-05-Aufgaben-Loesungen]{hier}.
\fi 

%\begin{theorembox}[Umkehrbarkeit]{}
%Es gilt:
%	\begin{itemize}
%		\item Eine Funktion $f:x\mapsto y$ mit der Definitionsmenge $D_f$ und der Wertemenge $W_f$ hei�t umkehrbar, falls es zu jedem $y\in W_f$ genau ein $x\in D_f$ mit $f(x)=y$ gibt.
%		\item Ist eine Funktion $f:x\mapsto y$ umkehrbar, so ist die umgekehrte Zuordnung $y\mapsto x$ eine Funktion. Diese hei�t Umkehrfunktion von $f$ und wird $f^{-1}$ bezeichnet. Ihre Definitionsmenge ist $W_f$ und ihre Wertemenge ist $D_f$.
%		\item Jede streng monotone Funktion $f$ ist umkehrbar. 
%	\end{itemize}
%\end{theorembox}

%\begin{theorembox}[Umkehrbarkeit]
%	Strenge Monotonie und damit Umkehrbarkeit kann man anhand der Ableitung diskutieren. Jede in einem Intervall $I$ differenzierbare Funktion $f$ mit $f'(x) > 0$  ($f'(x) < 0$) ist auf $I$ streng monoton steigend (fallend) und damit f�r alle $x\in I$ umkehrbar.
%\end{theorembox}

\begin{exercisebox}[Symmetrie]
  Welche der folgenden Funktionen ist gerade, welche ungerade?
  \begin{eqnarray*} \begin{array}{lll} 
		  (a)\; x\mapsto -2\,x^6 + 3\,x^2  \quad& (b)\; x\mapsto 2-3\,x^4 \quad &(c)\; x\mapsto 2-3\,x^3 \\[1ex]
		  (d)\; x\mapsto 3\,x^3-x+1 \quad & (e)\; x\mapsto x\,(2\,x^2 - \frac13\, x^4) \quad & (f)\; x\mapsto (x-1)\cdot (x-2) \\[1ex]
		  (g)\; x\mapsto (x-1)^3+3\,x^2+1 \quad &(h)\; x\mapsto (1-3\,x^2)^2 & (i)\; x\mapsto (x-x^2)^2 
	  \end{array}
  \end{eqnarray*}
\end{exercisebox}


\begin{exercisebox}[Umkehrfunktion]
	Gegeben sei die (diskrete) Funktion $f:\{1,2,3,4\} \to \mathbb N$ mit der Wertetabelle 
	\begin{center}
\begin{tabular}{l|l|l|l|l}
$x$ & $1$ & $2$ & $3$ & $4$ \\ 
\hline
$f(x)$ & $9$ & $6$ & $7$ & $8$
\end{tabular}
\end{center}
Bestimmen sie die Umkehrfunktion $f^{-1}$.
\end{exercisebox}

\begin{exercisebox}[Umkehrfunktion]
  Auf welchen Intervallen ist $f$ umkehrbar? Bestimmen Sie $f^{-1}$ und geben Sie $D_{f^{-1}}$ an
  \begin{eqnarray*}
	  \begin{array}{lll} (a)\; f: x \mapsto 4x-x^2  & \quad \quad (b)\; f:x \mapsto \displaystyle{\frac{2x}{x-1}} & \quad \quad (c)\; f:x\mapsto x\cdot |x| \end{array}
  \end{eqnarray*}
\end{exercisebox}

\begin{exercisebox}[Anwendung: Verpackungsreduktion]
	Umweltbewusste Studenten behaupten: Bei Dosen-Limo ist die Verpackung teurer als der Inhalt. F�r die �berpr�fung treffen die Studenten folgende Annahmen: Die Dose ist ein Zylinder, dessen H�he doppelt so gro� ist wie sein Durchmesser. F�r den Hersteller kostet $1$ Liter Limo $15\,\mathrm{ct}$. Die Kosten f�r $1\,\mathrm{dm}^2$ Blech betragen $3 \,\mathrm{ct}$
  \begin{itemize}
	  \item[(a)] �berpr�fen Sie die Behauptung f�r den Dosenradius $3 \,\mathrm{cm}$.
	  \item[(b)] Ab welchem Dosenradius ist der Inhalt teurer als die Dose?
  \end{itemize}
\end{exercisebox}

\begin{exercisebox}[Anwendung: Kostenreduktion]
	Die Herstellungskosten eines Airbus-Seitenleitwerks aus Metall werden angen�hert durch \[ k_1: x\mapsto \frac{20\cdot x+5000}{x+50}\,,\] wobei $x$ die Anzahl der hergestellten Leitwerke und $k_1(x)$ eine willk�rliche Geldeinheit beschreibt. Nachdem $300$ Leitwerke hergestellt sind, wird erwogen, die Produktion auf Kunststoffleitwerke umzustellen. Die St�ckkosten betragen dann n�herungsweise \[k_2:x\mapsto \frac{15\cdot x-2500}{x-250}\,, \quad x>300\,.\]
  \begin{enumerate}
  	\item Zeichnen Sie die beiden Graphen, zum Beispiel mit Hilfe von Geogebra o.�..
	\item Wie verhalten sich die St�ckkosten bei sehr gro�en Produktionszahlen?
	\item Ab welcher St�ckzahl ist das Kunststoffleitwerk billiger?
  \end{enumerate}
\end{exercisebox}
