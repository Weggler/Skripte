%%%% Pingo
%\begin{exercisebox}[Funktion - ja oder nein?]
%	Entscheiden Sie, ob die folgenden Skizzen Graphen einer Funktion \(f: X \to Y\) sein k�nnen. 
%	\centering
%	\includegraphics[width=0.9\textwidth]{../Mathematik_1/Bilder/Funktionen_ja_oder_nein.png}
%\end{exercisebox} 
%\begin{exercisebox}[Eigenschaften]
%	\begin{center}
%		\includegraphics[width=0.5\textwidth]{../Mathematik_1/Bilder/Rennstrecke.png}\hspace{1cm}
%	\includegraphics[width=0.3\textwidth]{../Mathematik_1/Bilder/Rennstrecken.png}
%	\end{center}
%  Der Graph zeigt, wie sich die Geschwindigkeit $\boldsymbol v$ eines Rennwagens w�hrend seiner zweiten Runde auf einer ebenen Rennstrecke ver�ndert.
%  \begin{enumerate}
%    \item Wie lang ist eine Runde?
%    \item Nennen Sie die niedrigste Geschwindigkeit und die h�chste Geschwindigkeit des Rennwagens.
%    \item Wie h�ufig musste der Rennfahrer abbremsen? Was k�nnten m�gliche Gr�nde sein?
%    \item Auf welcher Strecke war er unterwegs?
%\end{enumerate} 
%\end{exercisebox} 

\ifdefined\MOODLE 
\section*{Funktionen Aufgaben}\label{MA1-04-Aufgaben}

Regeln und Beispiele, die zur Bearbeitung der Aufgaben hilfreich sind, finden Sie in Abschnitt \ref{MA1-04} und, falls Sie nicht weiterkommen, schauen Sie \hyperref[MA1-04-Aufgaben-Loesungen]{hier}. 

\else
\section{Funktionen Aufgaben}\label{MA1-04-Aufgaben}

Regeln und Beispiele, die zur Bearbeitung der Aufgaben hilfreich sind, finden Sie im Skript und, falls Sie nicht weiterkommen, schauen Sie \hyperref[MA1-04-Aufgaben-Loesungen]{hier}. 
\fi 

\begin{exercisebox}[Funktionale Abh�ngigkeiten]
  Die Tabelle gibt den Verpackungsverbrauch pro Jahr von Privathaushalten und Kleinbetrieben in Millionen Tonnen an. 
  \begin{enumerate}
    \item Veranschaulichen Sie die Werte der Tabelle durch einen Graphen.
    \item Wann wurde vermutlich die neue Verpackungsverordnung (Gr�ner Punkt) eingef�hrt?
    \item Wie hoch w�re der Verpackungsverbrauch $1993$ wohl gewesen, wenn man die neue Verpackungsverordnung nicht eingef�hrt h�tte? Wie viel Prozent an Verpackungsmaterial wurde dadurch etwa eingespart?
  \end{enumerate}
  \begin{center}
  \begin{tabular}{c|c}
    Jahr & Verbrauch in Mio. Tonnen \\
    \hline
    $1988$ & $5,7$\\
    $1989$ & $6,4$\\
    $1990$ & $7,1$\\
    $1991$ & $7,6$\\
    $1992$ & $7,2$\\
    $1993$ & $7,0$\\
    $1994$ & $6,9$\\
    $1995$ & $6,7$
  \end{tabular}
  \end{center}
\end{exercisebox} 

\begin{exercisebox}[Mathematische Kurzschreibweise]
  Dr�cken Sie die Aussagen in mathematischer Kurzschreibweise aus.
  \begin{enumerate}
    \item Durch die Funktion $f$ wird der Zahl $3$ die Zahl $10$ zugeordnet.
    \item Die Funktion $g$ nimmt an der Stelle $5$ den Funktionswert $12$ an.
    \item Die Zahl $3$ geh�rt nicht zur Definitionsmenge der Funktion $f$.
    \item Die Funktion $f$ ordnet der Zahl $4$ einen gr��eren Funktionswert zu als der Zahl $5$.
    \item Die Funktionen $f$ und $g$ nehmen f�r $x=2$ denselben Funktionswert an.
    \item Alle Funktionswerte der Funktion $g$ sind positiv.
  \end{enumerate}
\end{exercisebox}

\begin{exercisebox}[Definitionsbereich]
	Bestimmen Sie den maximalen Definitionsbereich der folgenden Abbildungen:
  \begin{eqnarray*}
	  \begin{array}{lll}
		  (a) \; x\mapsto (x-1)^2 &\quad \quad (b)\; x\mapsto 3-5x-x^2 & \quad \quad (c)\; x\mapsto \displaystyle{\frac{1}{x}} \\[2ex]
		  (d) \; x\mapsto \displaystyle{\frac{1}{3-x}} & \quad \quad (e)\; x\mapsto \displaystyle{\frac{1}{(x-1)^2}} &\quad \quad (f) \; x\mapsto \displaystyle{\frac{1}{x^2-1}} \\[2ex]
		  (g) \; x\mapsto \sqrt{x-3} &\quad \quad (h) \; x\mapsto \displaystyle{\frac{1}{\sqrt{x-3}}} &\quad \quad (i)\; x\mapsto \ln( x +3) 
	  \end{array}
  \end{eqnarray*}
\end{exercisebox}

\begin{exercisebox}[Bildbereich]
	Bestimmen Sie den Bildbereich (die Bildmenge) der folgenden Abbildungen:
  \begin{eqnarray*}
	  \begin{array}{lll}
		  (a)\; x\mapsto x^2 & \quad \quad (b)\; x\mapsto x^2 +1 & \quad \quad (c)\; x\mapsto 2 - x^2 \\[2ex]
		  (d)\; x\mapsto -(x+2)^2+3 &\quad \quad (e)\; x\mapsto 3x-0,5 &\quad\quad (f)\; x\mapsto \sin x \\[2ex]
    (g)\; x\mapsto 3^x &\quad \quad (h)\; x\mapsto 3 
    \end{array}
  \end{eqnarray*}
\end{exercisebox}

%\begin{exercisebox}[Graphen lesen, Maxima, Minima, Symmetrie]
%	\begin{center}
%	\includegraphics[width=0.4\textwidth]{../Mathematik_1/Bilder/Mountainbike.png}
%	\end{center}
%  Eine Mountainbike Tour in dem spanischen El-Ports-Gebirge hat das abgebildete H�henprofil. 
%  \begin{enumerate}
%    \item Wie viele H�henmeter sind beim ersten Anstieg zu �berwinden?
%    \item Wie gro� ist der Gesamtanstieg, der bei der Tour zu �berwinden ist?
%    \item Wie steil ist die letzte Abfahrt ab Streckenkilometer $26$? Geben Sie das Gef�lle in Prozent und den Steigungswinkel an.
%    \item Zur Quelle Canaleta f�hrt nur eine Sackgasse. Wie �u�ert sich das im Graphen? Wie lang ist vermutliche diese Sackgasse?
%    \item Wo kann es noch eine Weitere Sackgasse geben?
%  \end{enumerate}
%
%%  \begin{tabular}{c|c}
%%    Kilometer & H�henmeter \\
%%    $0$ & $550$\\
%%    $1$ & $700$\\
%%    $4$ & $800$\\
%%    $5$ & $630$\\
%%    $6$ & $\ldots$
%%  \end{tabular}
%\end{exercisebox} 
