
%\begin{exercisebox}[Verkettung]  %Moodle
%	Wenn $b\circ c$ die Funktion $2\cdot c + b$ ist, welche Funktion muss dann $b$ bezeichnen, damit $b\circ c = c\circ b$ ?
%\end{exercisebox}


\ifdefined\MOODLE 
\section*{Komposition von Funktionen Aufgaben}\label{MA1-12-Aufgaben}

	Regeln und Beispiele, die zur Bearbeitung der Aufgaben hilfreich sind, finden Sie in Abschnitt \ref{MA1-12} und, falls Sie nicht weiterkommen, schauen Sie \hyperref[MA1-12-Aufgaben-Loesungen]{hier}.
\else
\section{Komposition von Funktionen Aufgaben}\label{MA1-12-Aufgaben}

	Regeln und Beispiele, die zur Bearbeitung der Aufgaben hilfreich sind, finden Sie im Skript und, falls Sie nicht weiterkommen, schauen Sie \hyperref[MA1-12-Aufgaben-Loesungen]{hier}.
\fi 

\begin{exercisebox}[Verkettungen]
	Bestimmen Sie Funktionen $u_1, u_2$ so, dass $f=u_1\circ v_1 = u_2\circ v_2$ ist:
  \begin{itemize}
    \item[(a)] $f(x) = (2x+6)^3, \quad v_1(x) = 2x+6, \quad v_2(x) = x+3$
    \item[(b)] $f(x) = \frac{3}{(x-1)^2}, \quad v_1(x) = x-1, \quad v_2(x) = (x-1)^2$
  \end{itemize}
\end{exercisebox}

\begin{exercisebox}[Definitionsbereich]
  Geben Sie den maximalen Definitionsbereich der Funktion $u\circ v$ an
  \begin{eqnarray*}
    (a)\; u(x) = \sqrt{x}, \quad v(x) = 3-x \quad\quad
    (b)\; u(x) = \frac{1}{x}, \quad v(x) = 4-x^2 \quad\quad
    (c)\; u(x) = \sqrt{1-x}, \quad v(x) = x^2 
  \end{eqnarray*}
\end{exercisebox}

\begin{exercisebox}[Definitionsbereich]
	Gegeben sind die Funktionen $v: \mathbb R^+ \to \mathbb R,\; x\mapsto \ln(x)$ und $u:[0,2\pi] \to \mathbb [-1,1], \,x\mapsto \sin(x)$. Betrachten Sie die Funktion $f:v \circ u$. Geben Sie den maximalen Definitionsbereich von $f$ an. 
\end{exercisebox}

\begin{exercisebox}[\(g: x\mapsto 0,7\cdot (x+2)^{-2}+4\)]
	Gegeben sind die Funktionen $f:x\mapsto x^{-2}$ und $g:x\mapsto 0,7\cdot (x+2)^{-2} +4$.
	\begin{itemize}
		\item[(a)] Geben Sie den Definitions- und den Bildbereich der Funktionen $f$ und $g$ an. 
		\item[(b)] Beschreiben Sie, wie die Funktion $g$ aus der Funktion $f$ hervorgeht. 
		\item[(c)] Bestimmen Sie die Asymptoten der Funktionen $f$ und $g$.
	\end{itemize}
\end{exercisebox}

\begin{exercisebox}[Manipulationen eines Graphen]
	Gegeben sei eine Funktion $f:\mathbb R \to \mathbb R, x\mapsto f(x)$. Geben Sie eine Abbildungsvorschrift an, um den Graphen von $f$ \dots
	\begin{itemize}
		\item[(a)] um den Faktor $5$ horizontal zu stauchen,
		\item[(b)] um den Faktor $\pi$ vertikal zu strecken,
		\item[(c)] um $2$ L�ngeneinheiten horizontal nach links zu verschieben,
		\item[(d)] um $3\sqrt{7}$ L�ngeneinheiten vertikal nach oben zu verschieben.
		\item[(e)] um den Graphen an der $x$-Achse zu spiegeln.
	\end{itemize}
\end{exercisebox}

\begin{exercisebox}[Anwendung: Temperaturskalen]
  Temperaturangaben der Kelvin Skala rechnet man in solche der Celsius Skala nach der Vorschrift $x\mapsto x- 273$ um, solche der Celsius Skala in die Fahrenheit Skala durch $x\mapsto 1,8x+32$. Wie lautet die Vorschrift, um von der Kelvin Skala direkt in die Fahrenheit Skala umzurechnen?
\end{exercisebox}
