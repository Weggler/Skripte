
\ifdefined\MOODLE 
\section*{Folgen und Grenzwerte Aufgaben}\label{MA1-22-Aufgaben}

	Regeln und Beispiele, die zur Bearbeitung der Aufgaben hilfreich sind, finden Sie in \ref{MA1-22} und, falls Sie nicht weiterkommen, schauen Sie \hyperref[MA1-22-Aufgaben-Loesungen]{hier}.
\else 
\section{Folgen und Grenzwerte Aufgaben}\label{MA1-22-Aufgaben}

	Regeln und Beispiele, die zur Bearbeitung der Aufgaben hilfreich sind, finden Sie im Skript und, falls Sie nicht weiterkommen, schauen Sie \hyperref[MA1-22-Aufgaben-Loesungen]{hier}.
\fi 

\begin{exercisebox}[Folgen]
	Betrachten Sie einen Kreis und zeichnen Sie die Fl�che $A_3$ des durch den Kreis einbeschriebenen regelm��igen Dreiecks ein, dann die Fl�che $A_6$ eines regelm��iges Sechsecks, $A_{12}$ eines regelm��igen Zw�lfecks usw. 
	\begin{itemize}
		\item Konvergiert die Folge der Fl�cheninhalte? 
		\item Wenn ja, wie lautet der Grenzwert? 
	\end{itemize}
\end{exercisebox}

\begin{exercisebox}[Grenzwerte]
	\begin{itemize}
		\item[(a)] �bersetzen Sie die folgende Aussage: \newline Es sei $(a_n)_{n\in\mathbb N^+}$ und es gelte: \( \forall\, \varepsilon > 0 \; \exists\, n_\varepsilon \in \mathbb N \; \forall\, n\geq n_\varepsilon  \; |a_n - a| < \varepsilon \quad (*)\)
		\item[(b)] Erf�llt die Folge $(a_n)_{n\in \mathbb N}$ mit \( a_n = \dfrac{2n}{n+4}\) die Eigenschaft $(*)$? Falls ja, wie lautet $n_\varepsilon$ bei gegebenem $\varepsilon > 0$?
	\end{itemize}
\end{exercisebox}

\begin{exercisebox}[Grenzwerts�tze]
	Geben Sie f�r $c=0, \; c=1, \; c=\infty$ jeweils Folgen $(a_n)_{n\in \mathbb N} $ und $(b_n)_{n\in \mathbb N}$ an, so dass 
	\begin{eqnarray*} 
		\lim\limits_{n\to \infty} \left( a_n \cdot b_n\right) = c \quad \textnormal{mit}\; a_n \to \infty, b_n\to 0 \; \textnormal{f\"ur } n\to \infty\,, \\
		\lim\limits_{n\to \infty} \frac{ a_n }{ b_n} =  c \quad \textnormal{mit}\; a_n \to 0, b_n\to 0 \; \textnormal{f\"ur } n\to \infty\,, 
	\end{eqnarray*}
\end{exercisebox}

\begin{exercisebox}[Grenzwerte von Folgen]
	Bestimmen Sie ob die folgenden Grenzwerte existieren und berechnen Sie sie gegebenenfalls. 
  \begin{eqnarray*}
		  (a)\; \lim\limits_{n\to \infty} \displaystyle{\frac{(n+1)^2-n^2 }{n}} \quad 
		  (b)\; \displaystyle{\lim\limits_{n\to \infty}} \displaystyle{\frac{9n^2+n+1}{3n^3 +1}} \quad
		  (c)\; \lim\limits_{n\to \infty} \displaystyle{\frac{1 + 2 + \cdots + n }{n^2}} \quad 
		  (d)\; \lim\limits_{n\to \infty} \left( \sqrt{n+1} - \sqrt{n}\right) 
	  \end{eqnarray*}
		  %(d)\; \displaystyle{\lim\limits_{n\to \infty}} \displaystyle{\frac{1+2^2 + \cdots + n^2}{n^3}} \\
Hinweis zu $(b)$: Es gilt 
	$\displaystyle{\sum\limits_{k=1}^n k} =\displaystyle{\frac12}\cdot n \cdot (n+1)$
  \end{exercisebox}

\begin{exercisebox}[Grenzwerte von Folgen]
	�berpr�fen Sie, ob die Zahlenfolge $\{ f_n\}_{n\in\mathbb N^+}$ konvergiert und berechnen Sie ggf. den Grenzwert: 
	\[ f_1=0, \; f_{n+1} = \frac13 ( f_n +1)\,.\]
\end{exercisebox}

\begin{exercisebox}[Grenzwerte von Funktionen]
	Bestimmen Sie ob die folgenden Grenzwerte existieren und berechnen Sie sie gegebenenfalls.
  \begin{eqnarray*}
	  \begin{array}{llll}
	  (a)\; \lim\limits_{x\to\infty} \dfrac{-x^2+1}{x^3+2x^2+x} &\quad 
	  (b)\; \lim\limits_{x\to-\infty} \dfrac{-(x+1)^2}{2x^2-x+1} &\quad 
	  (c)\; \lim\limits_{x\to-\infty} \dfrac{2x^3+x^2-x}{-x^2+4} &\quad
	  (d)\; \lim\limits_{x\to \infty} \left( 2^{-x}\cdot x^4 \right) \\
	(e)\; \lim\limits_{x\to \infty} \dfrac{ \left( \ln(x)\right)^3}{x^2} &\quad
	(f)\; \lim\limits_{x\downarrow 0} \dfrac{ 2^{\frac1x}}{\ln(x)} &\quad
	(g)\; \lim\limits_{x\to 0} \dfrac{ x}{|x|} &\quad
	(h)\; \lim\limits_{x\to 0} \dfrac{\cos(x)}{x}
	  %(f)\; \lim\limits_{x\to \infty} \left( 2- \frac{1}{(x+3)^2}\right) &\quad
	  %(g)\; \lim\limits_{x\to \infty} \left( -\frac{1}{2x}+ 3\right) &\quad
	  %(h)\; \lim\limits_{x\to \infty} \frac{1}{\sqrt{x+1}} &\quad 
	  %(i)\; \lim\limits_{x\to \infty} \frac{1}{\sqrt{x}}+1 
  \end{array}
  \end{eqnarray*}

\end{exercisebox}

%\begin{exercisebox}[Teleskopsumme]
%\end{exercisebox}

%\begin{exercisebox}[Konvergenz Binomialreihe $e$]
%\end{exercisebox}
%
%\begin{exercisebox}[Konvergenz Potenzreihe $\exp$]
%\end{exercisebox}
%
%\begin{exercisebox}[Harmonische Reihe]
%\end{exercisebox}
%
%\begin{exercisebox}[Riemann Integral]
%\end{exercisebox}

%\begin{theorembox}[Konvergenz Reihe]
%  Reihe mit alternierender Nullfolge ist konvergent (nicht absolut konvergent). (Leibnitz Kriterizm)
%\end{theorembox}
%
%\begin{theorembox}[Teleskop Summe]
%  Reihe mit alternierender Nullfolge ist konvergent (nicht absolut konvergent). (Leibnitz Kriterizm)
%\end{theorembox}
%
%\begin{theorembox}[absolute Konvergenz]
%  absolut konvergene Reihen (Reihenfolge der Glieder egal): wenn die Summe �ber die Betr�ge der Koeffizienten konvergiert.  
%\end{theorembox}
%
%\begin{theorembox}[Potenzreihen stellen Funktionen dar]
%  Konvergenzradius: f�r welche $x$ konvergiert die Potenzreihe?
%\end{theorembox}
%
%\begin{theorembox}[Funktionenr�ume, Betrag, Metrik]
%	Abstandsbegriffe, Energie Physik
%\end{theorembox}
%
%\begin{theorembox}[Lipschitz-Stetigkeit]
%\end{theorembox}
%
%\begin{theorembox}[Mittelwertsatz]
%\end{theorembox}
%
%\begin{theorembox}[Chauchy-Folge]
%\end{theorembox}
%\begin{theorembox}[Vollst�ndigkeit]
%\end{theorembox}
%\begin{theorembox}[Banachscher Fixpunktsatz]
%\end{theorembox}
%
%\begin{theorembox}[Iterationsverfahren]
%  Was wollen wir �ber das Iterationsverfahren wissen?
%  \begin{itemize}
%	  \item Konvergiert das Verfahren �berhaupt? $\Rightarrow $ Banachscher Fixpunktsatz
%	  \item Gegen welchen Wert konvergiert es? $\Rightarrow$ Fixpunkt
%	  \item Wie schnell konvergiert das Verfahren? $\Rightarrow$ Konvergenzordung
%  \end{itemize}
%\end{theorembox}
