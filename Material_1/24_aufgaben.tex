
\ifdefined\MOODLE 
\section*{Differentiation Aufgaben}\label{MA1-24-Aufgaben}

	Regeln und Beispiele, die zur Bearbeitung der Aufgaben hilfreich sind, finden Sie in \ref{MA1-24} und, falls Sie nicht weiterkommen, schauen Sie \hyperref[MA1-24-Aufgaben-Loesungen]{hier}.
\else
\section{Differentiation Aufgaben}\label{MA1-24-Aufgaben}

	Regeln und Beispiele, die zur Bearbeitung der Aufgaben hilfreich sind, finden Sie im Skript und, falls Sie nicht weiterkommen, schauen Sie \hyperref[MA1-24-Aufgaben-Loesungen]{hier}.
\fi 

\begin{exercisebox}[Differenzierbarkeit]
  \begin{itemize}
	  \item[(a)] Untersuchen Sie, ob $f$ an der Stelle  $x_0 = 1$ differenzierbar ist:
		  \[ f:\mathbb R \to \mathbb R, x\mapsto  
			  \begin{cases} 1-x^2, \; x \leq 1\,, \\
    x^2 -1, \; x > 1\,.\end{cases} 
      \]
      \item[(b)] Bestimmen Sie die Parameter $a,b\in\mathbb R$ so, dass $g$ auf ganz $\mathbb R$ differenzierbar ist: 
		  \[ g:\mathbb R \to \mathbb R, x\mapsto  
			  \begin{cases} a\cdot x^2, \; x \geq 1\,, \\
    x +b, \; x < 1\,.\end{cases} 
      \]
    %\item $ f(x) = x\cdot |x|, \; x_0 = 0  $
    %\item $ f(x) = |x| \cdot (x-3), \; x_0 = 3  $
  \end{itemize}
\end{exercisebox} 

\begin{exercisebox}[Differenzierbarkeit]
	Bestimmen Sie die Ableitung der Funktion $f:x\mapsto \sqrt{x}$ mit Hilfe des Differenzenquotienten.
\end{exercisebox}

\begin{exercisebox}[Bedeutung der Ableitung]
	Skizzieren Sie die Ableitung f�r Funktion.
	\begin{center}
	\includegraphics[width=0.75\textwidth]{../Mathematik_1/Bilder/Funktion_Ableitung.png}
\end{center}
\end{exercisebox} 

\begin{exercisebox}[Differenzieren]
	Berechnen Sie von den folgenden Funktionen die Ableitungen. Es ist $a\in \mathbb R$ ein Parameter.
	\begin{eqnarray*}
		\begin{array}{ll}
			(a)\; f(x) = \mathrm{cosh}(x) = \displaystyle{\frac{e^x + e^{-x}}{2}} &\quad (b)\; g(u) = \displaystyle{\frac{u^2}{u^2+a^2}} \\[2ex]
			(c)\; h(p) = \mathrm{exp}\left(-p^2/(2a)^2\right) &\quad (d)\; i(x) = \displaystyle{\frac{(2x+3)^2-2}{\sqrt{x-1}}} \\
		\end{array}
	\end{eqnarray*}
\end{exercisebox}

\begin{exercisebox}[Anwendung: Monotonie-Satz]
Angenommen Sie wissen �ber eine Funktion $f$, dass ihre Ableitung auf einem Intervall $I$ negativ ist, also $f'(x) \leq 0$. Was k�nnen Sie dann �ber das Monotonie-Verhalten aussagen?
\end{exercisebox}

\begin{exercisebox}[Anwendungen: Lineare N�herung]
	Die Mittellinie einer Rennstrecke der Breite $2\, \mathrm{m}$ wird durch Funktion $x\mapsto 4-\frac12 x^2$ beschrieben. Wir beobachten einen Rennwagen der die Bahnkurve im Uhrzeigersinn entlang rast. Bei spiegelglatter Fahrbahn rutscht das Fahrzeug und landet im Punkt $Y(0|6)$ in den Strohballen. Wo hat das Fahrzeug die Stra�e verlassen? 
\end{exercisebox} 


%
%%\begin{exercisebox}[Differenzierbarkeit]
%%  Untersuchen Sie, ob $f$ an der Stelle $x_0$ differenzierbar ist.
%%  \begin{enumerate}
%%    \item $f(x) = \begin{cases} x^2-2x, x<3\\ \frac16x^3-\frac12x, x\geq 3\end{cases}, x_0 = 3$
%%    \item $f(x) = \begin{cases} \frac{x-2}{x}, x>2\\ 2-x, x\leq 2\end{cases}, x_0 = 2$
%%    \item $f(x) = |x-2|, \; x_0 = 2$
%%    \item $f(x) = (x-1)\cdot |x-1|, \; x_0 = 1$
%%    \item $f(x) = \sqrt{(x^2-1)^2}, \; x_0 = 1$
%%    \item $f(x) = \frac{x}{|x|+1}, \; x_0 = 0$
%%  \end{enumerate}
%%\end{exercisebox}
%
%\begin{exercisebox}[Kettenregel]
%  Leiten Sie ab und vereinfachen Sie das Ergebnis soweit wie m�glich:
%  \begin{eqnarray*}
%	  \begin{array}{lll}  
%		  (a)\; g(x) = \left( f(x) \right)^n, \quad n \in \mathbb N
%    &\quad (b)\; f(x) = \frac{1}{18}(3x+2)^6
%    &\quad (c)\; f(x) = 2(5-x)^{-1}\\
%    (d)\; f(r) = \sqrt{7r-r^2}
%    &\quad (e)\; f(x) = (1+\sqrt{x})^2
%    &\quad (f)\; f(x) = \frac{1}{x^2}+\sin(\frac{1}{x})
%    %&\quad (a) f(x) = \sin\left( (ax)^2 \right)$
%    %\item $f(x) = \frac{1}{(x-2)^2}$
%    %\item $f(x) = (x+x^2)^2$
%    %\item $f(t) = \frac{5}{(t^2-1)^2}$
%    %\item $f(x) = \frac{1}{a+bx}$
%  \end{array}
%  \end{eqnarray*}
%\end{exercisebox} 
%
%\begin{exercisebox}[Prduktregel]
%  Leiten Sie ab und vereinfachen Sie das Ergebnis soweit wie m�glich:
%  \begin{eqnarray*}
%	  \begin{array}{lll}  
%		  (a)\; f(x) = x\sqrt{x} 
%	    & \quad (b) \; g(t) = (2t^2-3)\sqrt{t} 
%     & \quad (c) \; f(x) = x\cdot \sin(x) \\
%     (d) \; f(x) = (kx+1)\cdot \sin(x) 
%     & \quad (e) \; f(x) = (x^2+1)\cdot \cos(x)
%     & \quad (f) \; f(x) = (2x+3)^3(2x-1)^2
%     %& \quad (g) \; f(x) = \sqrt{x}(x-t)$
%     %& \quad (h) \; f(x) = x^2\sqrt{x}$
%     %& \quad (i) \; f(x) = \sin(2x)\cdot \cos(2x)$
%     %& \quad (j) \; f(a) = \sqrt{a}(1-2a^3)$
%  \end{array}
%  \end{eqnarray*}
%\end{exercisebox} 
%
%\begin{exercisebox}[Quotientenregel]
%  Leiten Sie ab und vereinfachen Sie das Ergebnis soweit wie m�glich:
%  \begin{eqnarray*}
%	  \begin{array}{lll}  
%	(a)\; g(t) = \frac{2-t^3}{2+t^3}  
%    &\quad (b)\; h(r) = \frac{2r^4}{r^2-1}  
%    &\quad (c)\; s(t) = \frac{4t^2-5}{2t+1} \\ 
%    (d)\; z(t) = \frac{t^2-1,5t}{1+0,8t}  
%    &\quad (e)\; f(x) = \frac{\sqrt{x}+1}{\sqrt{x}-1}  
%    &\quad (f)\; f(x) = \frac{\sin(x)}{\cos(x)} \\ 
%    %&\quad (g)\; f(a) = \frac{3a}{1+x^2} $ 
%  \end{array}
%  \end{eqnarray*}
%\end{exercisebox} 
%
%\begin{exercisebox}[Ableitungen]
%  Bestimmen Sie die ersten drei Ableitungen
%  \begin{eqnarray*}
%	  \begin{array}{lll}  
%		  (a)\; f(x) = \frac34 x^4 - \frac54 x^2 + 2x  & \quad 
%		  (b)\; f(x) = (2sx^2-3tx)^2 & \quad 
%	 	  (c)\; f(x) = \frac{a^2x^2+2ax-a}{4ax} \\
%		  (d)\; f(s) = \frac{4rs^4+8r^4s^2}{2r^2s^3} & \quad 
%		  (e)\; f(x) = -2a\sin(\frac34 ax) & \quad 
%		  (f)\; f(x) = \frac{2}{t}\cos(-2tx) 
%	%\item $ f(x) = 0,5x^3+3x^2-4  $
%	%\item $ f(x) = 2\left( \frac15 x^7 + \frac35 x^6\right)$
%	%\item $ f(x) = ax^3 + bx^2 +cx + d$
%	%\item $ f(x) = 2r^2x^4-4rx^3+r$
%	%\item $ f(s) = \frac{a^2x^2+2ax-a}{4ax} + s$
%	  \end{array}
%  \end{eqnarray*}
%\end{exercisebox} 

%\begin{exercisebox}[Ableitung]
%	Exponentialfunktionen, trigonometrische Funktionen - mittels Potenzreihen ? 
%\end{exercisebox} 
%
%\begin{exercisebox}[Reihendarstellung von $\exp$ aus Ableitungseigenschaft]
%	$\exp' = \exp$ Entwicklung der Reihe. Zeige zuerst dass an der Stelle $x=0$ Tangente mit Steigung $m=1$.
%\end{exercisebox} 

%\begin{exercisebox}[Differenzierbarkeit verlangt Stetigkeit]
%  Ist die Stetigkeit eine notwendige oder hinreichende Bedingung f�r Differenzierbarkeit?
%\end{exercisebox}

%\begin{exercisebox}[Ableitung von Area-Funktionen, mit Hinweis]
%\end{exercisebox}

%\begin{exercisebox}[Umkehrfunktion, strenge Montonie, strenge Montonie]
%\end{exercisebox}
%
%\begin{exercisebox}[Zwischenwertsatz, Mittelwertsatz]
%\end{exercisebox}

%\begin{theorembox}[Merkregel: Kettenregel]
%  �u�ere Ableitung mal innere Ableitung.
%\end{theorembox}

