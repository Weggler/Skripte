
%\begin{exercisebox}[Umkehrfunktion] %Pingo
%  Ist eine Funktion $f:x\mapsto y$ mit Umkehrfunktion $f^{-1}$. Welche der folgenden Aussagen ist richtig?
%  \begin{enumerate}
%    \item $W_{f^{-1}}=W_f$ 
%    \item $D_{f^{-1}}=W_f$ 
%    \item $W_{f^{-1}}=D_f$ 
%    \item $D_{f^{-1}}=D_f$ 
%  \end{enumerate}
%\end{exercisebox}

\section*{Eigenschaften von Funktionen L�sungen}\label{MA1-05-Aufgaben-Loesungen}

\ifdefined\MOODLE 
	Regeln und Beispiele, die zum Verst�ndnis der L�sungswege hilfreich sind, finden Sie in Abschnitt \ref{MA1-05}.
\else 
	Regeln und Beispiele, die zum Verst�ndnis der L�sungswege hilfreich sind, finden Sie im Skript.
\fi 

\begin{exercisebox}[Symmetrie]
  Welche der folgenden Funktionen ist gerade, welche ungerade?
  \begin{eqnarray*} \begin{array}{lll} 
		  (a)\; x\mapsto -2\, x^6 + 3\, x^2  \quad& (b)\quad x\mapsto 2-3\, x^4  \quad & (c)\; x\mapsto 2-3\,x^3 \\[1ex]
		  (d)\; x\mapsto 3\,x^3-x+1 \quad & (e)\; x\mapsto x\,(2\,x^2 - \frac13 \,x^4) \quad & (f)\; x\mapsto (x-1)\cdot(x-2) \\[1ex]
		  (g)\; x\mapsto (x-1)^3+3\,x^2+1 \quad &(h)\; x\mapsto (1-3\,x^2)^2 & (i)\; x\mapsto (x-x^2)^2 
	  \end{array}
  \end{eqnarray*}
 

  \hspace{0.3cm}
  \newline
  {\bf L�sung:}
  \begin{center}
	  \begin{tabular}{c|c|c|c|c|c|c|c|c|c}
	  & $a)$ & $b)$ & $c)$ & $d)$ & $e)$ & $f)$ & $g)$ & $h)$ & $i)$ \\
	  \hline
	  \hline
	  gerade & $\bullet$ & $\bullet$ &  &  &  &  &  & $\bullet$ & \\
	  \hline
	  ungerade &  &  &  &  & $\bullet$ &  &  $\bullet^{(*)}$&  &  \\
	  \hline
	  weder noch &  &  & $\bullet$ & $\bullet$ & & $\bullet$ &  & & $\bullet$ 
  \end{tabular}
  \end{center}
  $(*)$ $(x-1)^3+3x^2+1  = (x-1)(x^2-2x+1) = x^3-2x^2+x^2 - x^2 +2x^2 -1  +3x^2+1  = x^3 -2x$
\end{exercisebox}

\begin{exercisebox}[Umkehrfunktion] %Andreas
Gegeben sei die (diskrete) Funktion $f:\{1,2,3,4\} \to \mathbb N$ mit der Wertetabelle 
\begin{center}
\begin{tabular}{l|l|l|l|l}
$x$ & $1$ & $2$ & $3$ & $4$ \\ 
\hline
$f(x)$ & $9$ & $6$ & $7$ & $8$
\end{tabular}
\end{center}

	\vspace{0.3cm}
	\noindent{\bf L�sung:}\newline
Die Umkehrfunktion $f^{-1}:\{6,7,8,9\} \to \{1,2,3,4\}$ besitzt die folgende Wertetabelle 
\begin{center}
\begin{tabular}{l|l|l|l|l}
$x$ & $9$ & $6$ & $7$ & $8$ \\
\hline
$f^{-1}(x)$ & $1$ & $2$ & $3$ & $4$ 
\end{tabular}
\end{center}
\end{exercisebox}

\begin{exercisebox}[Umkehrfunktion] 
  Auf welchen Intervallen ist $f$ umkehrbar? Bestimmen Sie $f^{-1}$ und geben Sie $D_{f^{-1}}$ an
  \begin{eqnarray*}
	  \begin{array}{lll} (a)\; f: x \mapsto 4x-x^2  & \quad \quad (b)\; f:x \mapsto \displaystyle{\frac{2x}{x-1}} & \quad \quad (c)\; f:x\mapsto x\cdot |x| \end{array}
  \end{eqnarray*}

	\vspace{0.3cm}
	\noindent{\bf L�sung:}\newline
	\begin{itemize}
		\item[(a)] Offenbar gilt \[f: \mathbb R \to (-\infty, 4], x\mapsto x\cdot ( 4-x)\] $f$ besitzt dasselbe asymptotische Verhalten wie \(x\mapsto -x^2\) und ist demnach auf \(\mathbb R\) nicht injektiv und damit nicht invertierbar. Aber man kann \(f\) auf zwei Intervalle einschr�nken, auf denen Injektivit�t sicher gestellt ist, n�mlich auf dem Intervall $I_1:= (-\infty,2]$ und auf dem Intervall $I_2:=[2,\infty)$. Die Abbildungsvorschriften der entsprechenden Umkehrfunktionen erh�lt man, in dem man die Gleichung \(y=4x-x^2\) nach \(x\) aufl�st: 
\begin{equation*}
	\begin{array}{rr rcl l}
		\forall x\geq 2: && y &=& 4x-x^2 &  \\
		&\Leftrightarrow &y &=& \left( -x^2+4x \color{red}{-4} \right) \color{red}{+ 4} & \\
		&\Leftrightarrow &y &=& -(x-2)^2 + 4 & || - 4 \\
		&\Leftrightarrow &4-y &=& (x-2)^2 & || \sqrt{\cdot} \\[2ex]
		&\stackrel{ x\geq2 !}{\Leftrightarrow} &\sqrt{4-y} &=& x-2 & || +2 \\
		&\Leftrightarrow &\pm\sqrt{4-y}+2 &=& x & 
	\end{array}
  \end{equation*}
  Entsprechend l�st man f�r \(x<2\) auf und erh�lt:
  \begin{eqnarray*}
	  \begin{array}{l l l}
  f_1^{-1}: &(-\infty, 4] \to (-\infty, 2], &\; x\mapsto 2-\sqrt{4-x} \\  
  f_2^{-1}: &(-\infty, 4] \to  [2, \infty),&\; x\mapsto 2+\sqrt{4-x}  \\
  \end{array}
  \end{eqnarray*}
  \item[(b)] $f$ ist auf $\mathbb R\setminus\{1\}$ invertierbar und es gilt:
	  \begin{eqnarray*}
  f^{-1}: \mathbb R\setminus\{1\} \to \mathbb R\setminus\{ 1\}, \, x\mapsto \frac{x}{x-2}  
  \end{eqnarray*}
  \item[(c)] Um die Abbildung \(f\) zu verstehen, l�sen wir den Betragsstrich auf: 
	  \begin{equation*}
		  f(x) = \begin{cases} x^2, \; x\geq0\,,\\ -x^2, \; x<0\,.\end{cases}
	  \end{equation*}
	  \(f\) ist eine auf ganz \(\mathbb R\) definierte streng monoton steigende Funktion (f�r \(x\geq 0\) entspricht sie der Normalparabel und f�r \(x<0\) der, an der \(x\)-Achse gespiegelten Normalparabel). Also $f$ ist auf $\mathbb R$ invertierbar und wir bestimmen die Abbildungsvorschrift der Umkehrfunktion st�ckweise wie folgt: 
	  \begin{eqnarray*}
		  x\geq 0 \Rightarrow y\geq 0: \quad y = x^2 \Rightarrow +\sqrt{y} = x\,.\\
		  x< 0 \Rightarrow y<0: \quad y = -x^2 \Leftrightarrow |y| = x^2 \Leftrightarrow -\sqrt{|y|} = x\,.
	  \end{eqnarray*}
	  Nach der Variablenumbennenung erhalten wir insgesamt
	  \begin{eqnarray*}
  f^{-1}: \mathbb R \to \mathbb R, x\mapsto \begin{cases} \sqrt{x}, &x\geq 0 \\ -\sqrt{|x|}, &x<0 \end{cases} 
  \end{eqnarray*}
  \end{itemize}
\end{exercisebox}


\begin{exercisebox}[Anwendung: Verpackungsreduktion] %LS p.37
Umweltbewusste Studenten behaupten: Bei Dosen-Limo ist die Verpackung teurer als der Inhalt. F�r die �berpr�fung treffen die Studenten folgende Annahmen: Die Dose ist ein Zylinder, dessen H�he doppelt so gro� ist wie sein Durchmesser. F�r den Hersteller kostet $1$ Liter Limo $15\,\mathrm{ct}$. Die Kosten f�r $1\,\mathrm{dm}^2$ Blech betragen $3 \,\mathrm{ct}$
  \begin{itemize}
	  \item[(a)] �berpr�fen Sie die Behauptung f�r den Dosenradius $r=3 \,\mathrm{cm}$.
	  \item[(b)] Ab welchem Dosenradius ist der Inhalt teurer als die Dose?
  \end{itemize}

	\vspace{0.3cm}
	\noindent{\bf L�sung:}\newline
	\begin{enumerate}
		\item Die Behauptung trifft zu!  Mit $h=4r [\mathrm{dm}]$ erh�lt man 
			\begin{center}
	\begin{tabular}{l|l|l}
		& Funktionaler Zusammenhang & Einheit \\[1ex]
		\hline
		Materialbedarf $M$ & $\displaystyle{M: r\mapsto 2 \cdot(\pi r^2) + (2\pi r)\cdot h =  10\pi r^2}$  &$ [\mathrm{dm}^2]$\\[1ex]
		Doseninhalt $V$ & $\displaystyle{V: r\mapsto (\pi r^2)\cdot h=4\pi r^3 }$  &$[\mathrm{dm}^3] = [\mathrm{l}]$\\[1ex]
		Materialkosten $K_M$ & $\displaystyle{K_M: r\mapsto 3 \cdot M(r) = 30\pi r^2 }$ &$[\mathrm{ct}] $\\[1ex]
		Kosten f�r Limo $K_L$ & $\displaystyle{K_L: r\mapsto 15 \cdot V(r) = 60\pi r^3 }$ &$[\mathrm{ct}] $
	\end{tabular}
\end{center}
Und f�r $r = 3 \,\mathrm{cm} = 0,3\, \mathrm{dm}$ folgt
	\begin{center}
	\begin{tabular}{l|l}
		Materialkosten & $\displaystyle{K_M(0,3\, \mathrm{dm} ) = 2,7\pi \approx 8,48 \,\mathrm{ct} }$ \\
		Kosten f�r Limo & $\displaystyle{K_L(0,3 \, \mathrm{dm}) = 1,62\pi \approx 5,09 \, \mathrm{ct} }$ \\
	\end{tabular}
\end{center}
\item  Die Bedingung $K_L(r) > K_M(r)$ wird f�r $r>\frac12 \mathrm{dm}$ erf�llt.
	\end{enumerate}
\end{exercisebox}

\begin{exercisebox}[Anwendung: Kostenreduktion] %LS p.38
	Die Herstellungskosten eines Airbus-Seitenleitwerks aus Metall werden angen�hert durch \[ k_1: x\mapsto \frac{20\cdot x+5000}{x+50}\,,\] wobei $x$ die Anzahl der hergestellten Leitwerke und $k_1(x)$ eine willk�rliche Geldeinheit beschreibt. Nachdem $300$ Leitwerke hergestellt sind, wird erwogen, die Produktion auf Kunststoffleitwerke umzustellen. Die St�ckkosten betragen dann n�herungsweise \[k_2:x\mapsto \frac{15\cdot x-2500}{x-250}\,, \quad x>300\,.\]
  \begin{enumerate}
 \item Zeichnen Sie die beiden Graphen, zum Beispiel mit Hilfe von Geogebra o.�..
    \item Wie verhalten sich die St�ckkosten bei sehr gro�en Produktionszahlen?
    \item Ab welcher St�ckzahl ist das Kunststoffleitwerk billiger?
  \end{enumerate}

	\vspace{0.3cm}
	\noindent{\bf L�sung:}\newline
	\begin{center}
	\includegraphics[width=0.9\textwidth]{../Mathematik_1/Bilder/Airbus.png}
	\end{center}
	\begin{itemize}
	\item[2.] $\displaystyle{\lim\limits_{x\to\infty} k(x) = 20}$ und $\displaystyle{\lim\limits_{x\to\infty} k^*(x) = 15}$ und 
	\item[3.] $k^*(x) < k(x) $ f�r $x>330$.
  \end{itemize}
	
\end{exercisebox}
