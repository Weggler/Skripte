
\ifdefined\MOODLE 
\section*{Geradengleichung Aufgaben }\label{MA1-18-Aufgaben}

	Regeln und Beispiele, die zur Bearbeitung der Aufgaben hilfreich sind, finden Sie in \ref{MA1-18} und, falls Sie nicht weiterkommen, schauen Sie \hyperref[MA1-18-Aufgaben-Loesungen]{hier}.
\else 
\section{Geradengleichung Aufgaben }\label{MA1-18-Aufgaben}

	Regeln und Beispiele, die zur Bearbeitung der Aufgaben hilfreich sind, finden Sie im Skript und, falls Sie nicht weiterkommen, schauen Sie \hyperref[MA1-18-Aufgaben-Loesungen]{hier}.
\fi 

\begin{exercisebox}[Analytische Geometrie]
	Es sei $\boldsymbol r, \boldsymbol n\in\mathbb R^3$. Welches geometrische Objekt stellt die L�sungsmenge der folgenden Gleichung(en) dar? 
	\[ \boldsymbol x \in \mathbb R^3: \; \left( \boldsymbol x - \boldsymbol r \right) \boldsymbol \cdot \boldsymbol n = 0\]
\end{exercisebox}

\begin{exercisebox}[Lineare Funktion]
	Wie l��t sich eine Funktion $f:\mathbb R^n \to \mathbb R$ beschreiben, die in jedem Argument linear ist? Es gelte also \[ f( x_1, x_2, \dots, \alpha x_i, \dots, x_n) = \alpha f(x_1,x_2,\dots, x_i, \dots, x_n)\quad i=1,\dots,n\,.\]
\end{exercisebox}

\begin{theorembox}[Tangente und Normale]
	Es sei eine differenzierbare Funktion $f: \mathbb R \to \mathbb R$ gegeben. 
	\begin{enumerate}	
		\item Eine Gerade $g$, die an der Stelle $x_0$ denselben Funktionswert wie $f$ und dieselbe Steigung wie $f$ aufweist, nennt man Tangente an der Stelle $x_0$. Die Geradengleichung lautet \[ g: \mathbb R \to \mathbb R,\; x\mapsto f'(x_0) (x-x_0) + f(x_0)\,.\]
		\item Eine Gerade $n$, die an der Stelle $x_0$ denselben Funktionswert wie $f$ und die negative Steigung wie $f$ aufweist, nennt man Normale an der Stelle $x_0$. Die Geradengleichung lautet \[ n: \mathbb R \to \mathbb R,\; x\mapsto -f'(x_0) (x-x_0) + f(x_0)\,.\]
	\end{enumerate}
\end{theorembox}

\begin{exercisebox}[Tangenten]
	Gegeben ist die Funktion $f: \mathbb R \to \mathbb R, \; x\mapsto \frac14 x^2$. Ihr Graph sei $K$. 
  \begin{enumerate}
    \item Zeigen Sie, dass die Tangenten an $K$ in den Punkten $B_1\left(-3|\frac94\right)$ und  $B_2\left(\frac43|\frac49\right)$ orthogonal zueinander sind. Zeichnen Sie $K$ zusammen mit diesen Tangenten. 
    \item Es seien $S_1\left(a|f(a)\right)$ und  $S_2\left(b|f(b)\right)$ Punkte auf $K$. Welche Beziehung besteht zwischen $a$ und $b$, wenn die Tangenten an $K$ in $S_1$ und $S_2$ orthogonal sind?
    \item Bestimmen Sie die Punkte  $S_1\left(a|f(a)\right)$ und  $S_2\left(b|f(b)\right)$ so, dass die Tangenten in $S_1$ und $S_2$ orthogonal zueinander sind und die Strecke $S_1S_2$ parallel zur $x$-Achse ist. 
  \end{enumerate}
\end{exercisebox}

\begin{exercisebox}[Tangenten]
  \begin{enumerate}
	  \item Gegeben ist die Funktion \[ f: \mathbb R \to \mathbb R, x \mapsto \frac14 x^2 + cx\,.\] 
		  Der Graph von $f$ schneidet die $x$-Achse im Ursprung $O(0|0)$ und im Punkt $A(a|0)$. Die Tangenten an den Graphen in $O$ und $A$ schneiden sich im Punkt $B$. Zeigen Sie, dass das Dreieck $OAB$ stets gleichschenklig ist. Bestimmen Sie den Koeffizienten $c$ so, dass das Dreieck $OAB$ auch rechtwinklig ist.
	  \item Gegeben ist die Funktion \[ g: \mathbb R \to \mathbb R, \; x \mapsto 2 \sin(cx), \quad c \in (0,\infty)\,. \] Der Graph von $g$ schneidet die $x$-Achse im Ursprung $O=(0|0)$ und im Punkt $A(\frac{\pi}{c}|0)$. Die Tangenten an den Graphen in $O$ und $A$ schneiden sich im Punkt $B$. Bestimmen Sie den Koeffizienten $c$ so, dass das Dreieck $OAB$ gleichseitig ist. 
  \end{enumerate}
\end{exercisebox}

%\begin{exercisebox}[Mathematik lesen]
%	$\forall \boldsymbol u, \boldsymbol v\mathbb R^n: \quad |\boldsymbol u \boldsymbol \cdot \boldsymbol v| \leq |\boldsymbol u| \cdot |\boldsymbol v| $
%\end{exercisebox}
%
%\begin{exercisebox}
%	Wie lautet die Dreiecksungleichung?
%	$|\boldsymbol u + \boldsymbol v| \leq |\boldsymbol u| +|\boldsymbol v|$
%\end{exercisebox}

%\begin{theorembox}[Wichtige Ungleichungen] % Hinweis: Recherche. (als MC)
%	\begin{itemize}
%		\item Dreiecksungleichung
%		\item Cauchy-Schwarze Ungleichung %Skalarprodukt
%	\end{itemize}
%\end{theorembox}

%\begin{exercisebox}[Analytische Geometrie]
%	Es gibt drei verschiedene M�glichkeiten, eine Ebene in $\mathbb R^3$ festzulegen. Welche fallen Ihnen ein?
%\end{exercisebox}

