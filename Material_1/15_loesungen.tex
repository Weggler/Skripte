\section*{Komplexe Zahlen II L�sungen}\label{MA1-15-Aufgaben-Loesungen}

\ifdefined\MOODLE 
	Regeln und Beispiele, die zum Verst�ndnis der L�sungswege hilfreich sind, finden Sie in \ref{MA1-15}.
\else
	Regeln und Beispiele, die zum Verst�ndnis der L�sungswege hilfreich sind, finden Sie im Skript.
\fi 

\begin{exercisebox}[Radizieren]
	Bestimmen Sie alle L�sungen der folgenden Gleichungen
	\begin{eqnarray*}
		(a)\; z^3 = -8 \quad \quad \quad 
		(b)\; z^2 = i \quad \quad \quad  
		(c)\; z^4 = -16 
  \end{eqnarray*}

\vspace{0.3cm}
\noindent{\bf L�sung:}\newline
\begin{itemize}
	\item[(a)] 
		\begin{eqnarray*} 
			-8 = 8 \cdot e^{i\pi}\; \Rightarrow z_k &=& \sqrt[3]{8} \cdot e^{i(\pi+2\pi \,k)/3}, \quad k=0,1,2\\[1ex]
			z_0 &=& 2\cdot e^{i\pi/3} = 1+ \sqrt{3}i\,,\\ 
			z_1 &=& e\cdot e^{i\pi} = -2\,, \\ 
		z_2 &=& 2\cdot e^{5/3\pi \, i} = 1-\sqrt{3}i\,.
		\end{eqnarray*}
		
	\item[(b)]
  \begin{eqnarray*}
	  i = e^{i\pi/2} \; \Rightarrow z_k &=& e^{i(\pi/2+2\pi \,k)/2}, \quad k=0,1\\[1ex]
    z_0 &=&  e^{i \, \pi/4} = \frac{\sqrt{2}}{2} + i\frac{\sqrt{2}}{2}\,,\\
    z_1 &=&  e^{i \,5 \pi/4} =  -\frac{\sqrt{2}}{2} - i\frac{\sqrt{2}}{2}\,.
  \end{eqnarray*}
	\item[(c)] 
  \begin{eqnarray*}
	  -16 = 16 \cdot e^{i\pi} \; \Rightarrow z_k &=&  \sqrt[4]{16} \,e^{i(\pi+2\pi\,k)/4}, \quad k=0,1,2,3 \\[1ex]
	  z_0 &=&  2 e^{\pi/4 \,i} = 2e^{\pi/4 \,i}=\sqrt{2}+i\sqrt{2}\,,\\
  z_1 &=&  2 e^{3/4\pi \, i}=-\sqrt{2}+i\sqrt{2}\,,\\
  z_2 &=&  2 e^{5/4\pi\,i}=-\sqrt{2}-i\sqrt{2}\,,\\
  z_3 &=&  2 e^{7/4\pi\,i}=\sqrt{2}-i\sqrt{2}
  \end{eqnarray*}
\end{itemize}
\end{exercisebox}

\begin{exercisebox}[Eulersche Identit�t]
Zeigen Sie, dass die folgenden Darstellungen f�r $\sin$ und $\cos$ gelten
			\begin{eqnarray*} 
				(a)\; \sin(\varphi) = \frac{e^{i\varphi} - e^{-i\varphi}}{2i} \quad\quad\quad (b)\;\cos(\varphi) = \frac{e^{i\varphi} + e^{-i\varphi}}{2} 
			\end{eqnarray*}

\vspace{0.3cm}
\noindent{\bf L�sung:}\newline
\begin{eqnarray*}
	e^{ix} &=& \cos(x) + i \sin(x) \\
	e^{-ix} &=& \cos(-x) + i \sin(-x) = \cos(x) - i \sin(x) \\
	\Rightarrow e^{ix} + e^{-ix} &=& \cos(x) + i \sin(-x) + \cos(x) - i \sin(-x)  = 2 \cos(x) \Rightarrow \frac12  \left( e^{ix} + e^{-ix}  \right) = \cos(x) \\
	\Rightarrow e^{ix} - e^{-ix} &=& \cos(x) + i \sin(-x) - \cos(x) + i \sin(-x)  = 2 i  \sin(x) \Rightarrow \frac{1}{2i}  \left( e^{ix} - e^{-ix}  \right) = \sin(x)
\end{eqnarray*}
\end{exercisebox}

\begin{exercisebox}[Fundamentalsatz]
	Betrachten Sie f�r $n\in \mathbb N$ ein Polynom \[ p:\mathbb C\to \mathbb C, \; z\mapsto p(z)=a_nz^n+a_{n-1}z^{n-1} +\cdots a_1z+a_0\] mit reellen Koeffizenten $a_0, \cdots, a_n\in \mathbb R$ und $a_n\not=0$. 
	\begin{itemize}
		\item[(a)] Zeigen Sie: wenn $z_0 \in \mathbb C$ eine Nullstelle von $p$ ist, dann ist auch $\overline{z_0}$ eine Nullstelle von $p$.
		\item[(b)] Wenden Sie obigen Satz an, um alle Nullstellen des Polynoma \[ p:\mathbb C\to \mathbb C, \; z\mapsto p(z) = z^4-2z^3+2z^2-10z+25 \] zu bestimmen. Zeigen Sie zu erst, dass $z_1= 2+j$ eine Nullstelle ist.
	\end{itemize}

  \hspace{0.3cm}
  \newline
  {\bf L�sung:}
  \begin{itemize}
	  \item[(a)] Sei $z_0$ eine Nullstelle. Zu zeigen ist $p(z_0)=0 \; \Rightarrow \, p(\overline{z_0}) = 0$:
		  \begin{eqnarray*} p(\overline{z_0}) &=& a_n\left( \overline{z_0}\right)^n +  a_{n-1}\left( \overline{z_0}\right)^{n-1} + \cdots + a_1\overline{z_0} + a_0 \\ 
		                                      &=& a_n\left( \overline{z_0}^n\right) +  a_{n-1}\left( \overline{z_0}^{n-1}\right) + \cdots + a_1\overline{z_0} + a_0 \\ 
						      &=& \overline{\left(a_n z_0^n\right)} +  \overline{\left(a_{n-1} z_0^{n-1}\right)} + \cdots + \overline{a_1z_0} + \overline{a_0} \quad\quad (*)\\
						      &=& \overline{ \underbrace{\left(a_n z_0^n\right) +  \left(a_{n-1} z_0^{n-1}\right) + \cdots + a_1z_0 + a_0}_{\displaystyle{ = p(z_0)=0}}}\\
							      &=& \overline{0} = 0\,.
					      \end{eqnarray*}
					      Die Umformung $(*)$ ist g�ltig, weil alle Koeffizienten reell sind, also $\overline{a_i} = a_i$ f�r alle $i=0,\cdots, n$. 
				      \item[(b)] Wegen Aufgabenteil (a) wissen wir, dass $z_1$ und $z_2=\overline{z_1}$ Nullstellen sind und das gegebene Polynom ohne Rest durch die entsprechenden Linearfaktoren teilbar ist. Man kann nacheinander mit je einem Linearfaktoren dividieren oder gleich das quadratische Polynom betrachten: 
					      \[ (z-z_1) \cdot (z-\overline{z_1}) = (z-(2+i))\cdot (z-(2-i)) = z^2-4z+5\]
					      Die Polynomdivision mit dem quadratische Polynom f�hrt auf 
					      \[ (z^4-2z^3+2z^2-10z+25):(z^2-4z+5) = z^2+2z+5 \]
					      Wir erhalten zwei weitere Nullstellen durch die allgemeine L�sungsformel zu $z_3=-1+2i$ und $z_4=-1-2i$. Die Linearfaktorzerlegung von $p$ lautet \[ p(z) = (z-(2+i))\cdot (z-(2-i)) \cdot (z-(-1+2i))\cdot (z-(-1-2i))\,.\]
  \end{itemize}
\end{exercisebox}

\begin{exercisebox}[Fundamentalsatz]
	\begin{itemize}
		\item[(a)]
	L�sen Sie die Gleichung: $4z^2 + (8+12i)z - 5+11i = 0$.
\item[(b)] Bestimmen Sie alle Nullstellen des komplexen Polynoms $p:z\mapsto z^2+\left(2\sqrt{2}i\right) \cdot z - 2\sqrt{3} i$.
	\end{itemize}

  \hspace{0.3cm}
  \newline
  {\bf L�sung:}
  \begin{itemize}
	  \item[(a)] Diese Aufgabe l�st man mit Hilfe einer quadratischen Erg�nzung:
	\begin{eqnarray*}
		p(z) &=& \underbrace{4}_{a_2}z^2 + \underbrace{(8+12i)}_{a_1}z \underbrace{- 5+11i}_{a_0} \\
		&=& (2z)^2 + 2\cdot 2z \cdot (2+3i) + (2+3i)^2 -(2+3i)^2-5+11i \\
		&=& (2z+2+3i)^2-i \\
		&&\\
		\Rightarrow && (2z+2 + 3i)^2 = i = e^{i\pi/2} \\
		\Rightarrow && (2z+2+3i)^2 = \begin{cases} e^{i\pi/4} \\ 2z_1+2+3i =  e^{i5/4 \pi} = e^{-i3/4 \pi} \end{cases} \\
		\Rightarrow && z = \begin{cases} \frac12( \frac{1}{\sqrt{2}} - 2 + i(\frac{1}{\sqrt{2}} -3)) \\ 
		                                 \frac12(- \frac{1}{\sqrt{2}} - 2 - i(\frac{1}{\sqrt{2}} +3))  \end{cases}
			\end{eqnarray*}
		\item[(b)] Mit der allgemeinen L�sungsformel erhalten wir 
			\[ 
				z = \dfrac{-2\sqrt{2}i + \omega}{2}\,,
			\]
			wobei \(\omega\) die L�sungen der folgenden Gleichung sind 
			\begin{eqnarray*}
				\omega^2 &=&= (2\sqrt{2}i)^2 - 4\cdot (-2\sqrt{3}i) \\  &=&  -8 + 8\sqrt{3}i = 16\cdot e^{i2/3\pi} \; \widehat = \;  \left( 16, \frac23 \pi\right) \,.
			\end{eqnarray*}
			Die komplexen Wurzeln lauten:
			\[\omega_0 = 8\cdot e^{i\pi/3} = 4+4\sqrt{3}i \quad \lor \quad \omega_1 = 8\cdot e^{i8/3\pi} = 8\cdot e^{-i2/3\pi} = -4-4\sqrt{3}i\] Setzen wir die L�sungen \(\omega_{0,1}\) ein, so erhalten wir  
			
			\[ z_1= \left(-\sqrt{3}-\sqrt{2}\right)i-1 \quad \lor \quad z_2= (\sqrt{3}-\sqrt{2})i+1 \,.\]
  \end{itemize}
\end{exercisebox}


\begin{exercisebox}[Anwendungen: Additionstheoreme]
Zeigen Sie, dass f�r alle $x\in \mathbb R$ gilt
	\begin{eqnarray*}
	(a)\; \sin(x \pm y) = \sin(x)\cdot \cos(y) \pm \cos(x)\cdot\sin(y),\quad \quad (b)\; \cos(3x) = 4 (\cos(x))^3 - 3\cos(x) 
\end{eqnarray*}

\vspace{0.3cm}
\noindent{\bf L�sung:}\newline
Verwenden wir im Folgenden die Abk�rzungen $c_x:= \cos(x)$ und $s_x:=\sin(x)$. 
\begin{itemize}
	\item[(a)] 
\begin{eqnarray*}
	z:=e^{i(x \pm y)} &=& e^{ix} \cdot e^{\pm iy} = = \left( c_x+is_x\right)\cdot \left(c_y\pm is_y\right) \\ 
	&=& c_xc_y \pm ic_xs_y +is_xc_y \mp s_xs_y = c_xc_y + \mp s_xs_y +i( s_x c_y \pm c_xs_y) \\
	&&\Rightarrow \begin{cases} \cos(x\pm y) = \mathrm{Re}(z) = \cos(x)\cos(y) \mp  \sin(x)\sin(y) \\  \sin(x\pm y) = \mathrm{Im}(z)= \sin(x)\cos(y) \pm \cos(x)\sin(y)   
	\end{cases}
	\end{eqnarray*}
	\item[(b)] 
	\begin{eqnarray*}
		z:= e^{i(3x)} &=& \left( e^{ix} \right)^3= = \left( c_x+is_x\right)^3 \\ 
		&=& c_x^3 + 3i c_x^2 s_x + i^2 3c_x s_x^2 + i^3 s_x^3  \\
		&=& c_x^3 - 3c_x s_x^2 + i( 3 c_x^2 s_x - s_x^3) \\
		&=& c_x^3 - 3c_x (1-c_x^2) + i( 3 (1-s_x^2) s_x - s_x^3) \\
		&&\Rightarrow \begin{cases} \cos(3x) = \mathrm{Re}(z) = 4\cos^3(x) - 3 \cos(x)  \\  \sin(3x) = \mathrm{Im}(z) = 3\sin(x) - 4\sin^3(x)\,.
	\end{cases}
	\end{eqnarray*}
\end{itemize}
\end{exercisebox}

\begin{exercisebox}[Anwendungen: Harmonische Schwingungen]
	Gegeben seien die harmonischen Schwingungen \[ s_1(t) = \sqrt{2} \cdot \sin(10t+\pi/4), \quad s_2(t)=2\cos(10t+\pi/6)\,.\]
	\begin{itemize}
		\item[(a)] Berechnen Sie die �berlagerung von $s_1$ und $s_2$. 
			
			Hinweis: $s_1(t) = \mathrm{Im}\left( \sqrt{2} e^{i\,( 10 t+ \pi/4)}\right), \;s_2(t) = 2\sin(10t + \pi/6+\pi/2) = \mathrm{Im}\left( 2 e^{i\,( 10 t+ 2/3 \pi)}\right)$. 
		\item[(b)] Wie muss eine harmonische Schwingung \[ s_3(t) = A_3\sin(\omega_3 t+ \varphi_3)\] gew�hlt werden, dass die �berlagerung von $s_1$ und $s_3$ eine Amplitude von $1$ und eine Nullphase von $\pi/2$ besitzt?
	\end{itemize}

\vspace{0.3cm}
\noindent{\bf L�sung (Direkte Anwendung der Formeln m�glich, siehe Unterabschnitt "Harmonische Schwingungen")}

\begin{itemize}
	\item[(a)]
		\begin{eqnarray*}
			s_1(t) + s_2(t) &=& \mathrm{Im}\left( \sqrt{2} e^{i\,( 10 t+ \frac14\pi)} + 2 e^{i\,( 10 t+ \frac23\pi)}\right) \\[1ex]
			&=& \mathrm{Im}\left(\sqrt{2} e^{\frac14 \pi \, i} e^{10 t \, i} + 2 e^{\frac23\pi \, i} e^{10 t \,i} \right) \\[1ex]
				&=& \mathrm{Im}\left(  \left( \sqrt{2} e^{\frac14\pi\,i}  + 2 e^{\frac23\pi\,i} \right)  e^{i\, 10 t} \right) \\[1ex]
				&=& \mathrm{Im}\left(  \left( \sqrt{2} \left((\frac{1}{\sqrt{2}} + i + \frac{1}{\sqrt{2}}\right)  + 2 \left( -\frac12 + \frac{\sqrt{3}}{2} i \right) \right)  e^{i\, 10 t} \right) \\[1ex]
				&=& \mathrm{Im}\left(  \left( 1 + i  -1  \sqrt{3} i \right)  e^{i\, 10 t} \right) \\[1ex]
				&=& \mathrm{Im}\left(  i \left( 1 + \sqrt{3}\right)   e^{i\, 10 t} \right), \quad i=e^{i\pi/2} \\[1ex]
				&=& \mathrm{Im}\left(  \left( 1 + \sqrt{3}\right)  e^{i\,( 10 t + \pi/2 )} \right) \\[1ex]
				&=& \left( 1 + \sqrt{3}\right) \sin( 10 t + \pi/2 )
	\end{eqnarray*}
	\item[(b)] 
		F�r die komplexe Amplitude der �berlagerung gilt \[ A= 1\cdot e^{i\pi/2} = i \]
		Suche $A_3$ so, dass \[ A_1 + A_3 = i \Leftrightarrow A_3 = i + A_1\,.\] Das hei�t \[ A_3 = i-(1+i) = -1 = 1\cdot e^{i\pi}\,. \] Also muss gelten $s_3(t) = \sin(10t+\pi)$.
\end{itemize}
\end{exercisebox}

