
\ifdefined\MOODLE 
\section*{Determinante, inverse Matrix Aufgaben}\label{MA1-21-Aufgaben}

	Regeln und Beispiele, die zur Bearbeitung der Aufgaben hilfreich sind, finden Sie in \ref{MA1-21} und, falls Sie nicht weiterkommen, schauen Sie \hyperref[MA1-21-Aufgaben-Loesungen]{hier}.
\else
\section{Determinante, inverse Matrix Aufgaben}\label{MA1-21-Aufgaben}

	Regeln und Beispiele, die zur Bearbeitung der Aufgaben hilfreich sind, finden Sie im Skript und, falls Sie nicht weiterkommen, schauen Sie \hyperref[MA1-21-Aufgaben-Loesungen]{hier}.
\fi 

\begin{exercisebox}[Determinante]
	Berechnen Sie die Determinante der Matrix \[ A = \left( \begin{array}{ccc} 2 & -6 & 2 \\ 0 & 0 & 2 \\ -1 & 2 & 1 \end{array}\right)\,.  \]

\end{exercisebox}

\begin{exercisebox}[Determinante]
	Betrachten Sie die Schar der Matrizen \[ A_a = \left( \begin{array}{ccc} 1 & 1 & a \\ 1 & a & 1 \\ 1 & a & a \end{array}\right)\,, \quad a \in \mathbb R\,. \] 
	\begin{itemize}
		\item[(a)] Berechnen Sie die Determinante von $A_a$.
		\item[(b)] F�r welche $a\in \mathbb R$ ist $A_a$ regul�r?
		\item[(c)] Berechnen Sie mit Hilfe des Gau�-Algorithmus die inversen Matrizen $A_a^{-1}$ f�r alle regul�ren Matrizen $A_a$.
		\item[(d)] Berechnen Sie mit Hilfe der Cramerschen Regel die erste Spalte der inversen Matrizen $A_a^{-1}$ f�r alle regul�ren Matrizen $A_a$. Hinweis: L�sen Sie $ A_a \boldsymbol v = \boldsymbol e_1, \quad \boldsymbol e_1 = (1, 0, 0)^\intercal\,.$

\end{itemize}

\end{exercisebox}
\begin{exercisebox}[Basis]
	�berpr�fen Sie mit Hilfe der Determinante, ob die Vektoren 
	\[ 
		\boldsymbol v_1 = ( 2, -1, 0)^\intercal, \quad  
		\boldsymbol v_2 = ( 1, 1, 2 )^\intercal, \quad 
	\boldsymbol v_3 = ( -5, 1,-2 )^\intercal \]
	eine Basis des $\mathbb R^3$ sind.
\end{exercisebox}

\begin{exercisebox}[Gleichungssysteme]
	\begin{itemize}
		\item[(a)] Zeigen Sie, dass die Matrix 
			\[ A = \left( \begin{array}{ccc} 1 & 2 & 0 \\ 1 & 7 & 4 \\ 3 & 13 & 4 \end{array}\right) \]
			regul�r ist. 

		\item[(b)] L�sen Sie das lineare Gleichungssystem f�r $x, y, z \in \mathbb R$
	\[ \begin{array}{ccccccc} 
			x_1 & + & 2x_2 & && = & x \\ 
			x_1 & + & 7x_2 &+ & 4x_3 & = & y \\ 
			3x_1 & + & 13x_2 &+ & 4x_3 & = & z 
	\end{array} \]
		\item[(c)] L�sen Sie das lineare Gleichungssystem f�r $x=3, y=8, z=0$
	\end{itemize}
\end{exercisebox}

\begin{exercisebox}[Anwendungen: Eigenwerte]
	Berechnen Sie alle $\lambda \in \mathbb C$, f�r die die Matrix $A- \lambda I_2$ nicht regul�r ist, wobei \[ A = \left( \begin{array}{cc} -11 & 12 \\ -9 & 10 \end{array}\right)\]. 
\end{exercisebox}

%\begin{itemize}
%	\item Erkl�rung: antisymmetrische Multilinearform
%	\item Spatprodukt, Vektorprodukt, Skalarprodukt
%	\item Kleinste Fehlerquadrate - lineaeres Gleichungssystem
%\end{itemize}

%\subsection{Details und Extras}
%\begin{mybox}[Inverse Matrix]
%  Es seien $A,B \in \mathbb R^{n\times n}$ invertierbar. Es gilt 
%  \begin{itemize}
%    \item $\left( A^{-1}\right)^{-1} = A$
%    \item $\left( A B\right)^{-1} = B^{-1} A^{-1}$
%  \end{itemize}
%\end{mybox}
