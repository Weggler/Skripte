\section*{Vektorr�ume L�sungen}\label{MA1-17-Aufgaben-Loesungen}

\ifdefined\MOODLE 
	Regeln und Beispiele, die zum Verst�ndnis der L�sungswege hilfreich sind, finden Sie in \ref{MA1-17}.
\else
	Regeln und Beispiele, die zum Verst�ndnis der L�sungswege hilfreich sind, finden Sie im Skript.
\fi 

\begin{exercisebox}[K�rperaxiome]
  Ist die folgende Aussage wahr oder falsch? 
  \begin{center} "\,Das Skalarprodukt ist eine Multiplikation von Vektoren."\end{center}
	\noindent{\bf L�sung:}\newline
	Diese Aussage ist falsch. W�re es eine Multiplikation von Vektoren, dann m�sste das Ergebnis wieder ein Vektor sein. 
\end{exercisebox}

\begin{exercisebox}[Vektorraum]
	Pr�fen Sie, ob die folgenden Mengen Vektorr�ume �ber \(\mathbb R\) sind oder nicht, geben Sie ggf die Dimension und eine Basis an. 
	\begin{itemize}
		\item[(a)] \( M = \{ \left(  a,\; 0, \;b\right)^\intercal,\; a, b\in\mathbb R\} \)
		\item[(b)] \( M = \{ \left( x, \; y\right)^\intercal:\; 2x-3y = 0,\; x,y\in\mathbb R\} \)
		\item[(c)] \( M = \{ \left( a, \; a^2\right)^\intercal,\; a\in\mathbb R\} \)
	\end{itemize}
	\noindent{\bf L�sung:}\newline
	\begin{itemize}
		\item[(a)] \(M\) ist ein Vektorraum �ber \(\mathbb R\), weil 
			\begin{itemize}
				\item die Summe zweier beliebiger Vektoren aus \(M\) wieder in \(M\) liegt: \[ \forall\; \boldsymbol v_1 = \left(\begin{array}{c} a_1 \\ 0 \\ b_1\end{array}\right), \boldsymbol v_2 = \left(\begin{array}{c} a_2 \\0 \\ b_2\end{array}\right)  \in M:\quad   \left(\begin{array}{c} a_1 \\ 0 \\ b_1\end{array}\right) + \left(\begin{array}{c} a_2 \\0 \\ b_2\end{array}\right) = \left( \begin{array}{c} a_1 + a_2 \\0 \\ b_1 + b_2 \end{array}\right) \in M\,. \]    \( M = \{ \left(  a,\; 0, \;b\right)^\intercal,\; a, b\in\mathbb R\} \)
				\item jedes skalare Vielfache eines Vektors aus \(M\) wieder in \(M\) liegt: \[ \forall\; \lambda \in \mathbb R: \; \lambda \left( \begin{array}{c} a \\0 \\b\end{array}\right) = \left( \begin{array}{c} \lambda a \\0 \lambda b\end{array}\right) \in M \,.\]
			\end{itemize}
			Jeder Vektor aus \(M\) ist als lineare Kombination der Einheitsvektoren \(\boldsymbol e_1 \) und \( \boldsymbol e_3\) darstellbar: \[ \forall \; \boldsymbol v= \left(\begin{array}{c} a \\ 0 \\ b\end{array}\right)\in M:\quad  \boldsymbol v = a\left( \begin{array}{c} 1 \\0 \\0\end{array}\right) + b\left( \begin{array}{c} 0 \\0 \\1\end{array}\right) \] und damit spannt \( \{ \boldsymbol e_1, \boldsymbol e_3\} \) den Vektorraum \(M\) auf und \(M\) ist von der Dimension zwei. 
		\item[(b)] \(M\) ist ein Vektorraum �ber \(\mathbb R\), weil 
			\begin{itemize}
				\item die Summe zweier beliebiger Vektoren \(\boldsymbol v_1, \boldsymbol v_2\) aus \(M\) wieder in \(M\) liegt: \(\boldsymbol v_1 + \boldsymbol v_2 \in M\)? Zwei beliebige Vektoren aus \(M\) sehen so aus \[\boldsymbol v_1 = \left(\begin{array}{c} x_1 \\ y_1\end{array}\right), \boldsymbol v_2 = \left(\begin{array}{c} x_2 \\y_2\end{array}\right) \] und erf�llen \[  2x_1 - 3y_1 = 0 \; \land 2x_2 - 3y_1 = 0\,.\] Die Koeffizienten der Summe der Vektoren, \[ \boldsymbol v_s = \left(\begin{array}{c} x_s \\ y_s \end{array}\right)= \boldsymbol v_1 + \boldsymbol v_2 = \left(\begin{array}{c} x_1 + x_2 \\ y_1 + y_2 \end{array}\right) \] erf�llen also \[  2 x_s - 3y_s = 2\left( x_1 + x_2\right) - 3 \left(y_1+y_2\right) = \left( 2x_1 -3y_1 \right) +  \left( 2x_2 -3y_2 \right) = 0+0 = 0\] und ist damit  Element der Menge \(M\). 
			\item jedes skalare Vielfache eines Vektors aus \(M\) wieder in \(M\) liegt: \[ \forall\; \lambda \in \mathbb R: \; \lambda \left( \begin{array}{c} x \\y\end{array}\right) = \left( \begin{array}{c} \lambda x \\\lambda y\end{array}\right) \in M \textnormal{mit}\; 2(\lambda x) - 3 ( \lambda y) = \lambda (2x-3y) = \lambda \cdot 0 = 0 \,.\]
			\end{itemize}
			Die Bedingung, die die Vektoren aus \(M\) erf�llen bedeutet aber, dass die Komponenten der Vektoren linear voneinander abh�ngen und zwar so: \[ 2x-3y= 0 \Rightarrow x = \dfrac32 y\,.\] Im anderen Worten: jeder Vektor, der in \(M\) liegt sieht so aus \[ \boldsymbol v = \left(\begin{array}{c} \dfrac32 \lambda \\  \lambda \end{array}\right) = \lambda \left( \begin{array}{c} \dfrac32 \\1 \end{array}\right)\,, \quad \lambda \in \mathbb R\,.\] Der Vektorraum \(M \subsetneq \mathbb R^2\) wird also von dem Vektor \(\boldsymbol e = \left( \dfrac32, \; 1\right)^\intercal \) aufgespannt (erzeugt) und \(M\) ist von der Dimension eins: \( \mathrm{dim}(M) = 1\).
		\item[(c)] \( M \) ist kein Vektorraum und das zeigt man am schnellsten, indem man zwei konkrete Vektoren w�hlt und zeigt, dass deren Summe nicht wieder in \(M\) liegt. Beispielsweise f�r \(a_1=2\) und \(a_2=1\) \[ \boldsymbol v_1 + \boldsymbol v_2 = \left( \begin{array}{c} 2 \\4\end{array}\right) + \left( \begin{array}{c} 1 \\1\end{array}\right) =  \left( \begin{array}{c} 3 \\5\end{array}\right) \not\in M\,,\] da \(3^2 \not=5\). 
	\end{itemize}
\end{exercisebox}

\begin{exercisebox}[Vektorraum]
	Es seien $\boldsymbol p$ und $\boldsymbol n$ Vektoren des $\mathbb R^3$. \begin{itemize}
		\item[(a)] Welches geometrische Objekt stellt die L�sungsmenge \(E\) der folgenden Gleichung(en) dar? 
	\[ \boldsymbol x \in \mathbb R^3: \; \langle \left( \boldsymbol x - \boldsymbol p \right), \boldsymbol n \rangle = 0\]
\item[(b)] Ist \(E\) ein reeller Vektorraum? Falls ja, geben Sie die Dimension und eine Basis an.
	\end{itemize}
	\noindent{\bf L�sung:}\newline
	 \begin{itemize}
		 \item[(a)]  
	\begin{minipage}{0.40\textwidth}
	\includegraphics[width=\textwidth]{../Mathematik_1/Bilder/17_Ebene_Normalform.png}
\end{minipage}
	\begin{minipage}{0.55\textwidth}
\( \langle \left( \boldsymbol x - \boldsymbol p \right), \boldsymbol n \rangle = 0 \) bedeutet geometrisch,
		\begin{itemize}
			\item dass alle Vektoren \(\boldsymbol x - \boldsymbol p\) senkrecht zu dem Vektor \(\boldsymbol n\) liegen.
			\item dass alle Vektoren der Form \(\boldsymbol x -\boldsymbol p\) komplett in einer Ebene liegen (\(\boldsymbol p\) ist der Aufh�ngepunkt der Ebene).
			\item dass die Vektoren \(\boldsymbol x\) gerade die Raumpunkte sind, die zu der Ebene geh�ren. 
\end{itemize}
\end{minipage}
Eine Ebene in \(\mathbb R^3\) ist also die Angabe eines Aufh�ngepunkts und eines Normalenvektors eindeutig bestimmt. \[ E= \left\{ \boldsymbol x\in \mathbb R^3: \quad \langle \boldsymbol x- \boldsymbol p, \boldsymbol n\rangle = 0 \right\}\,.\]
\item[(b)] 
	\begin{minipage}{0.40\textwidth}
	\includegraphics[width=\textwidth]{../Mathematik_1/Bilder/17_Ebene_Vektorform.png}
\end{minipage}
	\begin{minipage}{0.55\textwidth}
		\begin{itemize}
			\item Eine Ebene ist ein Vektorraum �ber \(\mathbb R\) mit Dimension zwei. 
		\item Zwei beliebige linear unabh�ngigen Vektoren \(\boldsymbol v, \boldsymbol u\), die in der Ebene liegen (also senkrecht zu \(\boldsymbol n\) stehen), sind eine Basis der Ebene.
		\item \( E = \mathrm{LH}\left\{ \boldsymbol u, \boldsymbol v\right\} \,.\)  	
\end{itemize}
\end{minipage}
\end{itemize}
\end{exercisebox}

\begin{exercisebox}[Basis]
		Gegeben sei der Vektor \( \boldsymbol a^\intercal = \left( \sqrt{3}, 1\right)^\intercal\). Bestimmen Sie zwei Vektoren \(\boldsymbol  v\) und \(\boldsymbol w\), die eine Orthonormalbasis des \(\mathbb R^2\) sind und \(\boldsymbol v \parallel \boldsymbol a\).
		\vspace{0.2cm}
	\noindent{\bf L�sung:}\newline
	Merken Sie sich: zu gegebenen Vektor in \(\mathbb R^2\) ist ein senkrechter Vektor einfach zu bestimmen: man dreht einfach die Koeffizienten um und f�gt zu einem ein Minuszeichen hinzu:\marginpar{Nur \(\mathbb R^2\)}   
	\[ \forall\; x_1, x_2 \in \mathbb R:\quad  \left(\begin{array}{c} x_1 \\x_2 \end{array} \right) \perp \left( \begin{array}{c} -x_2 \\x_1\end{array}\right) \]
	Und damit l�sen wir die Aufgabe in zwei Schritten:
	\begin{enumerate} 
		\item Normierung des Vektors \(\boldsymbol a\) liefert den ersten Basisvektor: \[ \hat {\boldsymbol e}_2 = \dfrac{1}{\sqrt{ (\sqrt{3})^2 + 1^2} } \left( \begin{array}{c} \sqrt{3} \\ 1\end{array}\right) = \dfrac12 \left( \begin{array}{c} \sqrt{3} \\ 1\end{array}\right) \,.\]
		\item Bestimmung eines normierten Vektors, der zu \(\boldsymbol a\) senkrecht ist: \[ \boldsymbol e_2 = \left( \begin{array}{c}-1 \\  \sqrt{3} \end{array}\right) \Rightarrow \hat{ \boldsymbol e}_2 = \dfrac{1}{2}\left( \begin{array}{c}-1 \\  \sqrt{3} \end{array}\right)\,.\]
	\end{enumerate}
\end{exercisebox}

\begin{exercisebox}[Dimension]
	Welche Dimension haben die folgenden reellen Vektorr�ume?
	\begin{itemize}
		\item[(a)] $ \mathbb V_1 = \left\{ \boldsymbol x \in \mathbb R^n: \boldsymbol x = \alpha \boldsymbol a\,, \; \boldsymbol a \in \mathbb R^n\setminus\{\boldsymbol 0\} \right\} $
		\item[(b)] $ \mathbb V_2 = \left\{ \boldsymbol x \in \mathbb R^n: \boldsymbol x = \alpha \boldsymbol a + \beta \boldsymbol b\,, \; \boldsymbol b,\boldsymbol a \in \mathbb R^n\setminus\{ \boldsymbol 0 \}, \; \boldsymbol a \times \boldsymbol b =0 \right\}$
	\end{itemize}

\noindent {\bf L�sung:}
	\begin{itemize}
		\item[(a)] Die Menge $M_1:=\left\{ \boldsymbol x \in \mathbb R^n: \boldsymbol x = \alpha \boldsymbol a\,, \; \boldsymbol a \in \mathbb R^n\setminus\{\boldsymbol 0\}\right\}$ beschreibt eine Gerade im Raum, die durch den Ursprung verl�uft. Es gibt einen frei w�hlbaren Parameter, also gilt $\mathrm{dim}M = 1$.
		\item[(b)] Die Menge $M_2:=\left\{ \boldsymbol x \in \mathbb R^n: \boldsymbol x = \alpha \boldsymbol a + \beta \boldsymbol b\,, \; \boldsymbol a, \boldsymbol b \in \mathbb R^n\setminus\{\boldsymbol 0\}, \boldsymbol a\times \boldsymbol b = 0\right\}$ beschreibt dieselbe Gerade wie die Menge $M_1$ und hat damit auch Dimension eins. Da die Vektoren $\boldsymbol a$ und $\boldsymbol b$ parallel zueinander sind, gilt \[ \boldsymbol x = \alpha \boldsymbol a + \beta \boldsymbol b = \alpha \boldsymbol a + \beta (\lambda \boldsymbol a) = \underbrace{(\alpha + \beta\cdot \lambda)}_{\displaystyle{=t}} \boldsymbol a\] 
				Bemerkung: w�ren die Vektoren $\boldsymbol a$ und $\boldsymbol b$ linear unabh�ngig, dann w�re $M_2$ eine Ebene im Raum mit Dimension zwei.
	\end{itemize}
\end{exercisebox}

\begin{exercisebox}[Anwendung: Krummlinige Koordinaten]
	\begin{itemize}
		\item[(a)] Bestimmen Sie die Polarkoordinaten eines beliebigen Vektors \(\boldsymbol x \in \mathbb R^2\setminus \{\boldsymbol 0\}\).
		\item[(b)] Zeigen Sie, dass  die Vektoren \[ \boldsymbol e_{\varphi} = \left( \begin{array}{c} -\sin(\varphi) \\\cos(\varphi) \end{array}\right), \; \boldsymbol e_{r} = \left( \begin{array}{c} \cos(\varphi) \\\sin(\varphi) \end{array}\right) \; \textnormal{mit}\; r\in [0,\infty), \; \varphi\in (-\frac{\pi}{2}, \frac{\pi}{2}] \] eine Orthonormalbasis des \(\mathbb R^2\) sind.
			\item[(c)] Skizzieren Sie f�r \(r=1\) und \(\varphi = \pi/4\) die Orthonormalbasen \( \{\boldsymbol e_x, \boldsymbol e_y\}  \) und \( \{\boldsymbol e_r, \boldsymbol e_\varphi\}  \) in einem gemeinsamen Koordinatensystem.  
	\end{itemize}
\noindent {\bf L�sung:}
	\begin{itemize}
		\item[(a)] Die kartesischen Koordinaten von \(\boldsymbol x\) seien \(x_1\) und \(x_2\), also \( \boldsymbol x \widehat = (x_1, x_2\). Die Polarkoordinaten \( (r, \varphi)\) lauten \[ (r, \varphi) = ( \sqrt{ x_1^2 + x_2^2} , \mathrm{atan2}\left(x_2, x_1\right) \] 
				\item[(b)] Zu zeigen ist, dass die Vektoren aufeinander senkrecht stehen (Skalarprodukt!) und jeweils die Norm eins besitzten, also 
					\begin{eqnarray*} \langle \boldsymbol e_\varphi, \boldsymbol e_r\rangle &=& -\sin(\varphi) \cdot \cos(\varphi) + \cos(\varphi) \cdot \sin(\varphi) = 0\,,\\ 
						|\boldsymbol e_\varphi| &=& \sqrt{ (-\sin(\varphi))^2 + (\cos(\varphi))^2} =  \sqrt{1} = 1\,,\\ 
						|\boldsymbol e_r| &=& \sqrt{ (\cos(\varphi))^2 + (\sin(\varphi))^2} =  \sqrt{1} = 1\,. 
					\end{eqnarray*}
				\item[(c)] 
	\begin{minipage}{0.45\textwidth}
	\includegraphics[width=\textwidth]{../Mathematik_1/Bilder/17_ONS_Bahnkurve.png}
\end{minipage}
	\begin{minipage}{0.55\textwidth}
		\begin{itemize}
			\item Der Vektor \(\boldsymbol e_r\) ist parallel zum Ortsvektor \(\boldsymbol P\).
			\item Der Vektor \(\boldsymbol e_\varphi\) ist senkrecht zum Ortsvektor \(\boldsymbol P\).
	\end{itemize}
\end{minipage}
					
Bemerkung: In der Praxis geht man oft �ber zu einem mitbewegenden Dreibein. Das Dreibein bewegt sich entlang der Bahnkurve und \(\boldsymbol e_r\) ist tangential und \(\boldsymbol e_\varphi\) senkrecht zur Bahnkurve. Beide Vektoren �ndern sich in jedem Punkt.  
	\end{itemize}
\end{exercisebox}

