\documentclass[a4paper, 10pt]{article}
\usepackage[top=1cm,bottom=1cm,left=1cm,right=1cm]{geometry}

\usepackage[ngerman,english]{babel}
\usepackage[ansinew]{inputenc}
\usepackage[colorlinks=true, urlcolor=blue]{hyperref}

\usepackage{graphicx}

\usepackage{enumitem}
\usepackage{multicol}
\usepackage{amssymb}
\usepackage{amsmath}
\usepackage{amstext}
\usepackage{amsfonts}
\usepackage{eurosym}

\begin{document}
{
	\begin{enumerate}
		\item F�hren Sie die folgenden Polynomdivisionen aus: 
			\[ \begin{array}{lcr}
					(x^3-3x-2) &:& (x-2) \\[2ex]
					(x^3-2x^2-3x+10)&:& (x+2)
				\end{array}
			\]
		\item Geben Sie die folgenden Polynome in faktorisierter Form an (Hinweis zu \(p_3\) und \(p_4\): vorige Aufgabe!): 
			\[ \begin{array}{rcl}
					p_1: x &\mapsto & -3x^2-9x-6 \\
					p_2: x &\mapsto & 2x^2-8x+26 \\
					p_3: x &\mapsto & x^3-3x-2 \\
					p_4: x &\mapsto & x^3-2x^2-3x+10
			\end{array}
		\]
	\item Zeigen Sie: falls gilt \[ x^2+px+q = (x-x_1)\cdot (x-x_2)\,,\] dann gilt \[ \begin{cases} -x_1 -x_2 &= p \\ x_1 \cdot x_2 &= q \end{cases}\] und bestimmen Sie mit dieser Idee die Faktorisierung des Polynoms \(p_5: x\mapsto x^2-x-2\,.\)
		\item Geben Sie das folgende Polynom in faktorisierter Form an: 
			\[
					p_6: x \mapsto  x^4-13x^2+36
		\]
\end{enumerate}

	\begin{enumerate}
		\item 
			\begin{eqnarray*}
				(x^3-3x-2) : (x-2)  &=& x^2+2\cdot x -1  \\[2ex]
					(x^3-2x^2-3x+10): (x+2) &=& x^2 - 4x+5 
				\end{eqnarray*}
		\end{enumerate}


}
\end{document}
