
\ifdefined\MOODLE 
\section*{Lineare Funktionen Aufgaben}\label{MA1-06-Aufgaben}
	
	Regeln und Beispiele, die zur Bearbeitung der Aufgaben hilfreich sind, finden Sie in Abschnitt \ref{MA1-06} und, falls Sie nicht weiterkommen, schauen Sie \hyperref[MA1-06-Aufgaben-Loesungen]{hier}.

\else
\section{Lineare Funktionen Aufgaben}\label{MA1-06-Aufgaben}
	
	Regeln und Beispiele, die zur Bearbeitung der Aufgaben hilfreich sind, finden Sie im Skript und, falls Sie nicht weiterkommen, schauen Sie \hyperref[MA1-06-Aufgaben-Loesungen]{hier}.
\fi 

\begin{exercisebox}[Abbildungsvorschrift] %Andreas
	Bestimmen Sie die Abbildungsvorschrift (Funktionsgleichung) in der Form \[ f(x) = m\cdot x + b\] einer linearen Funktion durch die Punkte \( (1,5| 2)\) und \( (-3|3)\).
\end{exercisebox}

\begin{exercisebox}[Umkehrfunktion]
	F�r welche reellen Werte von \(m\) und \(b\) ist die lineare Funktion $f: x\mapsto m\cdot x+b$ umkehrbar? 
\end{exercisebox}

\begin{exercisebox}[Abbildungsvorschrift] %Lovis
	Die durch die Gleichung \( x +3y= 8\) beschriebene Gerade im \(\mathbb R^2\) wird um zwei Einheiten nach rechts und um eine Einheit nach oben geschoben. Geben Sie die Abbildungsvorschrift f�r die verschobene Gerade an.  
\end{exercisebox}

\begin{exercisebox}[Betragsfunktion] %Lovis
	Skizzieren Sie die Menge \( \{ (x,y) \in \mathbb R^2: y<1+|x|\}\). Kennzeichnen Sie, ob die R�nder dazu geh�ren oder nicht. 
\end{exercisebox}

\begin{exercisebox}[Betragsfunktion] % LS p.21
		Stellen Sie die folgenden Funktionen ohne Betragszeichen dar. Zeichnen Sie den Graph: 
  \begin{eqnarray*}
	  \begin{array}{llll}
		  (a)\; f:x \mapsto 2\cdot |x| & \quad (b)\; f:x \mapsto |2-x| & \quad (c)\; f:x \mapsto |2+x| &\quad (d)\; f:x \mapsto x-|x|
	  \end{array}
  \end{eqnarray*}
\end{exercisebox}

\begin{exercisebox}[Anwendung: Interpolation] % Andreas
	Gegeben sind die Messpunkte $(x_i,y_i), \; i= 0,1,2,3,$ mit $(-1,5| 2),(-0,25| 3,5),(0,5| 3,3),(1,2| 2,8)$. Bestimmen Sie den linearen Spline und werten Sie ihn am Punkt $x= 0,2$ aus. Welchen Wert erhalten Sie?
\end{exercisebox}

\begin{exercisebox}[Anwendung: Ausgleichsgerade] %Lucy
	Gegeben sind die Messpunkte $(x_i,y_i), \; i= 0,1,2,3,$ mit $(-1,5| 2),(-0,25| 3,5),(0,5| 3,3),(1,2| 2,8)$. Bestimmen Sie die Ausgleichsgerade und werten Sie ihn am Punkt $x= 0,2$ aus. Welchen Wert erhalten Sie?
\end{exercisebox}

\begin{exercisebox}[Anwendung: Dimensionierung Solarzellen] %LS p.21
	Sonnenkollektoren wandeln Lichtenergie in W�re um, die an den Warmwasserspeicher abgef�hrt wird. Die ben�tigte Kollektorfl�che h�ngt linear vom Volumen des Speichers ab. Bei einer Speichertemperatur von $45^\circ \mathrm{C}$ wird f�r $200$ Liter eine Kollektorfl�che von $3\mathrm{m}^2$, f�r eine $500$ Liter $7\mathrm{m}^2$ empfohlen. Pro Person wird mit einem Verbrauch von $50$ Liter Warmwasser am Tag gerechnet. Das Speichervolumen sollte $50\%$ �ber dem Verbrauch liegen. 
  \begin{enumerate}
    \item Bestimmen Sie die Funktion, die der Personenanzahl die Kollektorfl�che zuordnet.
    \item Wie gro� sollte die Kollektorfl�che bei einem $4$-Personen Haushalt sein?
    \item F�r wie viele Personen reicht eine Kollektorfl�che von $8\mathrm{m}^2$?
  \end{enumerate}
\end{exercisebox}


%\begin{exercisebox}[Elastizit�t]
%\end{exercisebox}
%
%\begin{exercisebox}[Federkraft]
%\end{exercisebox}
