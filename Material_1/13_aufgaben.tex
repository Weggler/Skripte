
\ifdefined\MOODLE 
\section*{Trigonometrische Funktionen Aufgaben}\label{MA1-13-Aufgaben}

	Regeln und Beispiele, die zur Bearbeitung der Aufgaben hilfreich sind, finden Sie in \ref{MA1-13} und, falls Sie nicht weiterkommen, schauen Sie \hyperref[MA1-13-Aufgaben-Loesungen]{hier}.
\else 
\section{Trigonometrische Funktionen Aufgaben}\label{MA1-13-Aufgaben}

	Regeln und Beispiele, die zur Bearbeitung der Aufgaben hilfreich sind, finden Sie im Skript und, falls Sie nicht weiterkommen, schauen Sie \hyperref[MA1-13-Aufgaben-Loesungen]{hier}.
\fi 

\begin{exercisebox}[Harmonische Schwingungen] %Andreas
	Bestimmen Sie die Amplitude, die Periode und die Nullphase der folgenden harmonischen Schwingungen und skizzieren Sie den Funktionsverlauf.
	\begin{eqnarray*}
		(a)\; t\mapsto 2\cdot \sin\left(3t-\frac{\pi}{6}\right) \quad \quad \quad \quad  (b)\; t\mapsto 10\cdot \sin(\pi t - 3\pi)
	\end{eqnarray*}
\end{exercisebox}

\begin{exercisebox}[Interpolation] %Andreas
	Eine Funktion der Form \[ f: x\mapsto \frac{a_0}{2} + a_1 \cos(x) + b_1\sin(x)\] soll durch die Punkte $(x_i, y_i),\, i=1,2,3$ mit \[ (0|1), (\pi/6|-1), (\pi/2|-3)\]
	gelegt werden. Bestimmen Sie ein Gleichungssystem f�r die Koeffizienten $a_0, a_1, b_1$ und l�sen Sie es.
\end{exercisebox}

\begin{exercisebox}[Gleichungen] %Andreas
	Bestimmen Sie alle L�sungen der folgenden Gleichungen:
	\begin{eqnarray*}
		\begin{array}{ll}
			(a) \; \sin(2x+5) = 0,4 \quad \quad \quad & (b) \;\tan\left( 2(x+1)\right) = 1 \\[1ex] (c)\; \sqrt{\cos(x-1)} = 2^{-1/4} \quad \quad \quad & (d) \; \sin(x) = \sqrt{1-\sin^2(x)} 
		\end{array}
	\end{eqnarray*}
\end{exercisebox}

\begin{exercisebox}[Kreisgleichung] %Lucy
	Bestimmen Sie die Schnittpunkte der angegebenen Geraden $g$ mit dem Kreis $k:=\{(x,y):\; x^2+y^2=4\}$:
	\begin{eqnarray*}
		\begin{array}{lll}
		(a)\; g_1: x\mapsto x-2 \quad\quad (b) \; g_2: x\mapsto x-4 \quad\quad (c)\; g_3: x\mapsto 2 \end{array}
	\end{eqnarray*}
\end{exercisebox}

\begin{exercisebox}[Kreisgleichung] %Lucy
	Welche der Gleichungen beschreiben einen Kreis? Bestimmen Sie gegebenenfalls die Mittelpunkte und die Radien.
			\begin{eqnarray*} 
				\begin{array}{lll} 
					(a) \; (x+2)^2 + y^2 = 64 \quad \quad & (b)\; (x-5)^2 + (y+2)^2 = 0 \quad \quad & (c) \; x^2 + y^2 - 2x + 4y -20 = 0 \\[1ex] (d) \;x^2 + y^2 - 2x + 2y + 14 = 0 \quad \quad & (e) \; x^2 + y^2 +y = 0 &
				\end{array}
			\end{eqnarray*}
\end{exercisebox}

\begin{exercisebox}[Qualitativer Graph] %Lovis
	\begin{itemize}
		\item[(a)] Zeichnen Sie die Funktion $f: x\mapsto 2\cdot \sin(x-\frac{\pi}{2})-2$ auf dem Intervall  $[-\pi,3\pi]$ und lesen Sie aus dem Graphen den Bildbereich, die Nullstellen und die Extremstellen von $f$ ab.
		\item[(b)] Zeichnen Sie die Funktion $f: x\mapsto -\sin(x-\pi)$ auf dem Intervall  $[0,\frac{5\pi}{2}]$ und lesen Sie aus dem Graphen den Bildbereich, die Nullstellen und die Extremstellen von $f$ ab.
	\end{itemize}
\end{exercisebox}

%\begin{exercisebox}[Newton-Verfahren, $\pi$]
%Bestimme $\pi$ mit Hilfe des Newton-Verfahrens auf vier Dezimalstellen genau.
%\end{exercisebox}

%\begin{exercisebox}[Titanic]
%	Gro�kreise, Dreiecke auf gekr�mmten Mannigfaltigkeiten
%\end{exercisebox}
%
%\begin{exercisebox}[Schwingungen]
%	Basis, Akustik
%\end{exercisebox}


%\begin{exercisebox}[Dreiecke]
%\end{exercisebox}
