\ifdefined\MOODLE 
\section*{Potenz- und Wurzelfunktionen Aufgaben}\label{MA1-07-Aufgaben}
	
	Regeln und Beispiele, die zur Bearbeitung der Aufgaben hilfreich sind, finden Sie in Abschnitt \ref{MA1-07} und, falls Sie nicht weiterkommen, schauen Sie \hyperref[MA1-07-Aufgaben-Loesungen]{hier}.
\else 
\section{Potenz- und Wurzelfunktionen Aufgaben}\label{MA1-07-Aufgaben}
	
	Regeln und Beispiele, die zur Bearbeitung der Aufgaben hilfreich sind, finden Sie im Skript und, falls Sie nicht weiterkommen, schauen Sie \hyperref[MA1-07-Aufgaben-Loesungen]{hier}.
\fi 

\begin{exercisebox}[$f:x\mapsto  3\sqrt{4-x^2}$]
  Gegeben ist die Funktion $f:x\mapsto  3\sqrt{4-x^2}$.
  \begin{itemize}
	  \item[(a)] Geben Sie die Funktionswerte an den Stellen $-1$ und $\sqrt{3}$ an.
    \item[(b)] Berechnen Sie $f(0)$, $f(2)$, und $f(\frac23)$.
    \item[(c)] Bestimmen Sie die maximale Definitionsmenge und die Wertemenge von $f$.
    \item[(d)] Pr�fen Sie rechnerisch, ob der Punkt $P(-\sqrt 2| 3\sqrt 2)$ auf dem Graphen von $f$ liegt.
  \end{itemize}
\end{exercisebox}

\begin{exercisebox}[Symmetrie]
	Bestimmen Sie das Symmetrieverhalten der folgenden Funktionen in ihrem maximalen Definitionsbereich
  \begin{eqnarray*}
	  \begin{array}{lll} 
		  (a)\, f:x\mapsto 4x^2-16 &\quad (b)\, f:x\mapsto \displaystyle{\frac{x^3}{x^2+1}} &\quad (c)\; f:x\mapsto |x^2-4|\\ (d)\; f:x\mapsto \displaystyle{\frac{x^2-1}{1+x^2}} &\quad (e)\, f:x\mapsto \sqrt{x^2-25} &\quad (f)\,f:x\mapsto \displaystyle{\frac{1}{x-1}}
	  \end{array}
  \end{eqnarray*}
\end{exercisebox}

\begin{exercisebox}[Monotonie]
	Untersuchen Sie die folgenden Funktionen auf Monotonie. Hinweis: Benutzen Sie die bekannten Symmetrieeigenschaften der Potenz- und Wurzelfunktionen.
  \begin{eqnarray*}
	  \begin{array}{llll} 
		  (a)\, f:x\mapsto x^4 &\quad (b)\; f:x\mapsto \displaystyle{\sqrt{x-1}} &\quad (c)\; f:x\mapsto x^3+2x \quad (d)\; f:x\mapsto \displaystyle{| x^2-2x+1|} 
	  \end{array}
  \end{eqnarray*}
\end{exercisebox}

\begin{exercisebox}[Umkehrfunktion]
  \begin{itemize}
	  \item[(a)] Welche der Potenzfunktionen $f$ mit $f(x) = x^n,\; n\in\mathbb N\setminus\{0\}$  sind umkehrbar?
	  \item[(b)] Wie lauten die Umkehrfunktionen von folgenden Funktionen? \[ (a) \; f:x\mapsto \frac{1}{2x}\quad \quad (b)\; g:x\mapsto \sqrt{3x}\]
  \end{itemize}
\end{exercisebox}

\begin{exercisebox}[Gleichungen]
	L�sen Sie folgenden Gleichungen
  \begin{eqnarray*}
	  \begin{array}{llll} 
		  (a)\, \sqrt{ -3 +2x} = 2 &\quad (b)\; \sqrt{x^2+4}-x = -2 &\quad (c)\; \sqrt{x-1} = \sqrt{x+1}\quad& (d)\; \sqrt{2x^2-1}+x=0 
	  \end{array}
  \end{eqnarray*}
\end{exercisebox}

\begin{exercisebox}[Gleichungen]
	\begin{enumerate}
		\item[(a)] Bestimmen Sie alle L�sungen der Gleichung \[ -2x^2+6 = 4x \]
		\item[(b)] F�r welchen Parameter $p$ hat die folgende Gleichung genau eine L�sung? \[ x^2 + px+3p = 0\]
		\item[(c)] F�r welche Werte von $c$ hat die folgende Gleichung keine reelle L�sung? \[ x^2 + 2cx+c = 0\]
  \end{enumerate}
\end{exercisebox}
