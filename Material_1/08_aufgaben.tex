\ifdefined\MOODLE 
\section*{Polynome Aufgaben}\label{MA1-08-Aufgaben}

	Regeln und Beispiele, die zur Bearbeitung der Aufgaben hilfreich sind, finden Sie in Abschnitt \ref{MA1-08} und, falls Sie nicht weiterkommen, schauen Sie \hyperref[MA1-08-Aufgaben-Loesungen]{hier}.
\else
\section{Polynome Aufgaben}\label{MA1-08-Aufgaben}

	Regeln und Beispiele, die zur Bearbeitung der Aufgaben hilfreich sind, finden Sie im Skript und, falls Sie nicht weiterkommen, schauen Sie \hyperref[MA1-08-Aufgaben-Loesungen]{hier}.
\fi 

\begin{exercisebox}[Symmetrie] %LS
	Zeigen Sie: Wenn der Funktionsterm einer ganzrationalen Funktion einen konstanten Summanden enth�lt, dann ist die Funktion nicht ungerade. 
\end{exercisebox}

\begin{exercisebox}[Umkehrfunktion]
      Die Funktion $f$ mit $f(x)=ax^3+bx,\; a,b\in \mathbb R,$ sei umkehrbar. Welche Bedingung erf�llen die Koeffizienten $a$ und $b$?
\end{exercisebox}

\begin{exercisebox}[Polynom?] %LS p.23 Aufg. 3
	Entscheiden Sie, welche der folgenden Funktionen Polynome sind:
	\begin{eqnarray*} \begin{array}{lll}
			(a)\; x\mapsto 1+\sqrt{2}x & \quad (b)\; x\mapsto 1+2\sqrt{x} &\quad (c)\; x\mapsto (x-1)^2(x-7) \\[1ex]
			(d)\; x\mapsto x^2-\frac3x & \quad (e)\; x\mapsto x^2-\frac{x}{3} &\quad (f)\; x\mapsto  x^2+\sin(x)
		\end{array}
	\end{eqnarray*}
\end{exercisebox}

\begin{exercisebox}[Darstellungen von Parabeln] %Andreas
	Bestimmen Sie von folgenden Parabeln die Linearfaktordarstellung und die Scheitelpunktform.
  \begin{eqnarray*} \begin{array}{ll} 
		  (a)\; x\mapsto -2x^2 -4x+3 \quad \quad   \quad  \quad & (b)\; x\mapsto 5x^2+20x+20 \\[1ex]
		  (c)\; x\mapsto 2x^2+10x  \quad \quad \quad   \quad &(d)\; x\mapsto 4x^2+8x-60
	  \end{array}
  \end{eqnarray*}
\end{exercisebox}

\begin{exercisebox}[Abbildungsvorschrift] %Andreas
	\begin{itemize}
		\item[(a)] Wie m�ssen die Koeffizienten $a,b,c$ lauten, wenn die Parabel $y(x) = ax^2+bx+c$ an den Stellen $x_1 = 1$ und $x_2=-5$ verschwindet und der Funktionswert am Scheitelpunkt $y_S = 18$ betr�gt?
		\item[(b)] Bestimmen Sie ein Polynom $p$ vierten Grades mit den folgenden Eigenschaften: 
			\begin{itemize}
				\item $p$ ist eine gerade Funktion.
				\item $p$ besitzt die Nullstellen $x_1=3$ und $x_2=6$.
				\item $p(0)=-3$
			\end{itemize}
	\end{itemize}
\end{exercisebox}

\begin{exercisebox}[Polynomdivision]
	Zeigen Sie, dass $x_0 = -5$ eine Nullstelle von $p(x) = 3x^3 + 18 x^2 + 9x -30$ ist. Berechnen Sie mit Hilfe der Polynomdivision das Polynom \(q\), f�r das gilt $p(x) = (x-x_0)\cdot q(x)$.
\end{exercisebox}

\begin{exercisebox}[Polynomdivision]
	Bestimmen Sie alle Nullstellen der folgenden Polynome mit Hilfe einer Polynomdivision und stellen Sie die Polynome in Linearfaktorzerlegung dar. Die Polynome besitzen mindestens eine ganzzahlige Nullstelle. 
  \begin{eqnarray*} \begin{array}{ll} 
		  (a)\; x\mapsto x^3-2x^2-5x+6 \quad & (b)\; t\mapsto -2t^4-2t^3-4t+8 \\[1ex]
		  (c)\; x\mapsto x^4-x^3-x^2-x-2 \quad &(d)\; t\mapsto 2t^4+8t^3-12t^2-8t+10
	  \end{array}
  \end{eqnarray*}
\end{exercisebox}

%\begin{exercisebox}[BMI]
%	F�r das Idealgewicht sind zwei g�ngige Formeln in Gebrauch: \begin{center} Idealgewicht in $\mathrm{kg}$ $=$ K�rpergr��e in $\mathrm{cm}$ minus $100$ \end{center}
%	und der Body-Mass-Index, \begin{center} BMI $=$ $\displaystyle{\frac{\textnormal{Gewicht in }\mathrm{kg}}{\left( \textnormal{Koerpergroesse in }\mathrm{m} \right)^2}}\,,$ \end{center} 
%	sollte zwischen $20$ und $25$ liegen.
%	\begin{itemize}
%		\item[(a)] Zeichnen Sie ein Diagramm des Gewichts in Abh�ngigkeit der K�rpergr��e f�r die erste Formel und f�r BMI-Werte von jeweils $20$ und $25$. 
%		\item[(b)] F�r welche K�erpergr��e besitzt das Idealgewicht nach der ersten Formel einen BMI zwischen $20$ und $25$?
%	\end{itemize}
%\end{exercisebox}

%\begin{exercisebox}[Windkraftanlagen]
%	Nehmen Sie an ein Luftmolek�l trifft mit der kinetischen Energie \[ E_{L} = \frac12 m_L v^2\] auf die Rotorfl�chen eines Windkraftwerks. 
%	�berlegen Sie sich Formel, um hieraus n�herungsweise die Leistung $[\frac{\mathrm{N}}{{\mathrm{s}}}]$ eines Windkraftwerks in Abh�ngigkeit der Windgeschwindigkeit $v$ anzugeben.  
%\end{exercisebox}
