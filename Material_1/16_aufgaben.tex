
\ifdefined\MOODLE 
\section*{Vektoren Aufgaben}\label{MA1-16-Aufgaben}

	Regeln und Beispiele, die zur Bearbeitung der Aufgaben hilfreich sind, finden Sie in \ref{MA1-16} und, falls Sie nicht weiterkommen, schauen Sie \hyperref[MA1-16-Aufgaben-Loesungen]{hier}.
\else
\section{Vektoren Aufgaben}\label{MA1-16-Aufgaben}

	Regeln und Beispiele, die zur Bearbeitung der Aufgaben hilfreich sind, finden Sie im Skript und, falls Sie nicht weiterkommen, schauen Sie \hyperref[MA1-16-Aufgaben-Loesungen]{hier}.
\fi 

\begin{exercisebox}[Rechenregeln]
	Gegeben seien die Vektoren 
	\[ \boldsymbol a = \left( \begin{array}{c} -4 \\ 3 \end{array}\right)\,, \;  
		\boldsymbol b = \left( \begin{array}{c} 0 \\ -3 \end{array}\right)\,,  \;
	\boldsymbol c = \left( \begin{array}{c} -2 \\ -2 \end{array}\right)\,. \] 
	\begin{itemize}
		\item[(a)] Berechnen Sie folgende Ausdr�cke: 
			\( -\boldsymbol a + 3\cdot \boldsymbol b, \quad (2\cdot \boldsymbol a - \boldsymbol b) + 3\cdot \boldsymbol c, \quad (3\cdot \boldsymbol c - \boldsymbol b) + 2\cdot \boldsymbol a \)
		\item[(b)] Berechnen Sie die folgenden Betr�ge (L�ngen, Normen): 
			\( |\boldsymbol a|\,,\quad |\boldsymbol b|\,,\quad |\boldsymbol a+ \boldsymbol b|\,,\quad |-\boldsymbol a+3\cdot \boldsymbol b| \)
		\item[(c)] Bestimmen Sie einen Vektor der L�nge $10$, der in die gleiche Richtung wie der Gegenvektor von $\boldsymbol a$ zeigt. 
		\item[(d)] Berechnen Sie die folgenden Ausdr�cke:
			\( \boldsymbol a\boldsymbol \cdot \boldsymbol b, \quad 
			\boldsymbol a\boldsymbol \cdot \boldsymbol c, \quad 
			\boldsymbol a\boldsymbol \cdot ( 4\cdot \boldsymbol c), \quad 
			\boldsymbol a\boldsymbol \cdot ( \boldsymbol b\boldsymbol \cdot \boldsymbol c), \quad 
		(\boldsymbol a\boldsymbol \cdot  \boldsymbol b)\boldsymbol \cdot \boldsymbol c \)
	\item[(e)] Bestimmen Sie die Winkel im Bogenma� zwischen folgenden Vektoren \( \boldsymbol a\; \textnormal{und}\; \boldsymbol b, \quad \boldsymbol a \; \textnormal{und}\; \boldsymbol c\,.\)
		\item[(f)] Bestimmen Sie alle orthogonalen Vektoren zu $\boldsymbol a$.
		\item[(g)] Bestimmen Sie einen Vektor der Form $\boldsymbol b+\lambda\cdot \boldsymbol c$ mit $\lambda \in \mathbb R$, der senkrecht zu $\boldsymbol a$ ist.
		\item[(h)] Bestimmen Sie die orthogonale Zerlegung von $\boldsymbol b$ bzgl. $\boldsymbol a$, das hei�t die Vektoren $\boldsymbol b_\parallel$ und $\boldsymbol b_\perp$.
	\end{itemize}
\end{exercisebox}

\begin{exercisebox}[Rechenregeln]
	Gegeben sind die Vektoren \[ \boldsymbol a = \left( \begin{array}{c} -2 \\ 0 \\ -4\end{array}\right), \quad \boldsymbol b = \left( \begin{array}{c} -2 \\1 \\ -2 \end{array}\right), \quad \boldsymbol c = \left( \begin{array}{c} -1 \\ 1 \\0 \end{array}\right) \,. 
	\]
		\begin{itemize}
			\item[(a)] Berechnen Sie $-3\cdot(-2\cdot \boldsymbol a + \boldsymbol b) - \boldsymbol c$.
			\item[(b)] Berechnen Sie die folgenden Betr�ge (L�ngen, Normen): $|\boldsymbol a|, |\boldsymbol b|, |-3\cdot(-2\cdot \boldsymbol a + \boldsymbol b) - \boldsymbol c|$.
			\item[(c)] Berechnen Sie den Einheitsvektor in Richtung $\boldsymbol a$.
			\item[(d)] Berechnen Sie $\boldsymbol a\boldsymbol \cdot \boldsymbol b$ und $(-3\cdot \boldsymbol a)\boldsymbol \cdot \boldsymbol b$.
			\item[(e)] Berechenn Sie den Winkel $\varphi$ im Bogenma� zwischen $\boldsymbol a$ und $\boldsymbol b$ (Skalarprodukt!).
			\item[(f)] Berechnen Sie $\boldsymbol a\times \boldsymbol b$ und $\boldsymbol c\times (\boldsymbol a\times \boldsymbol b)$.
			%\item[(g)] Pr�fen Sie anhand der definierten Vektoren, dass gilt \[ \boldsymbol c\times ( \boldsymbol a\times \boldsymbol b) = (\boldsymbol c\boldsymbol \cdot \boldsymbol b)\boldsymbol \cdot \boldsymbol a -(\boldsymbol c\boldsymbol \cdot \boldsymbol a)\boldsymbol \cdot \boldsymbol b\] 
			\item[(g)] Berechnen Sie die Fl�che des durch $\boldsymbol a$ und $\boldsymbol b$ aufgespannten Parallelogramms. 
\end{itemize}
\end{exercisebox}

\begin{exercisebox}[Rechenregeln]
	Gegeben sind die folgenden Vektoren
	\[ \boldsymbol a = \left( \begin{array}{c} 1+2i \\ -i \\ 0 \\3 \end{array}\right), \quad \boldsymbol b = \left( \begin{array}{c} -3-i \\ 2i \\ 0 \\ -i\end{array}\right), \quad 
	\boldsymbol c = \left( \begin{array}{c} -2+i \\ 1+2i \end{array}\right), \quad \boldsymbol d = \left( \begin{array}{c} -i \\ -1-3i\end{array}\right)  \,. \]
	\begin{itemize}
		\item[(a)] Berechnen Sie $-i\boldsymbol a + (2-i)\boldsymbol b$.
		\item[(b)] Berechnen Sie $\langle \boldsymbol a, \boldsymbol b\rangle$.
		\item[(c)] Berechnen Sie $|\boldsymbol a|$.
		\item[(d)] Berechnen Sie die folgenden Ausdr�cke: \( |\boldsymbol c|, \quad \langle \boldsymbol c, \boldsymbol d\rangle, \quad \langle \boldsymbol d, \boldsymbol c\rangle\)
		\item[(e)] Bestimmen Sie eine Zerlegung von $\boldsymbol d$ der folgenden Art \( \boldsymbol d = \boldsymbol d_\perp + \boldsymbol d_\parallel, \quad \boldsymbol d_\perp \perp \boldsymbol c\,.\)
	\end{itemize}
\end{exercisebox}

%\begin{theorembox}[Linearkombination]
%  Ein Vektor $\boldsymbol x\in \mathbb R^n$ der Form
%  \begin{eqnarray*}
%    \boldsymbol x = \alpha_1 \boldsymbol x_1 + \alpha_2 \boldsymbol x_2 + \dots + \alpha_m\boldsymbol x_m, \quad \alpha_i \in \mathbb R, \boldsymbol x_i\in \mathbb R^n, \; i=1,\dots, m\,, 
%  \end{eqnarray*}
%  hei�t Linearkombination der Vektoren $\boldsymbol x_1, \dots, \boldsymbol x_m$.
%\end{theorembox}
%
%\begin{theorembox}[Lineare Unabh�ngigkeit, lineare Abh�ngigkeit]
%  Ein Vektorsystem $(\boldsymbol x_1, \boldsymbol x_2, \dots, \boldsymbol x_m)$ ($m$-Tupel) hei�t linear unabh�ngig, wenn gilt
%  \begin{eqnarray*}
%    \sum\limits_{i=1}^m \alpha_i\boldsymbol x_i = \boldsymbol 0 \Rightarrow  \forall i=1,\dots, m: \; \alpha_i = 0\,,
%  \end{eqnarray*}
%  andernfalls hei�en die Vektoren linear abh�ngig.
%\end{theorembox}
%
%\begin{theorembox}[Lineare Abh�ngigkeit]
%Merke: Sind die Vektoren $\boldsymbol x_1, \dots, \boldsymbol x_m\in \mathbb R^n$ linear abh�ngig, so kann man mindestens einen der Vektoren als lineare Kombination der anderen darstellen.
%\end{theorembox}
%
%\begin{theorembox}[Skalarprodukt]
%  Sei $\boldsymbol v =(v_1,\dots,v_n)^\intercal, \boldsymbol u =(u_1,\dots,u_n)^\intercal \in \mathbb R^n$. Das Skalarprodukt $\langle \boldsymbol u, \boldsymbol v\rangle$ ist definiert als  
%  
%  \begin{eqnarray*}
%    \langle \boldsymbol u, \boldsymbol v\rangle :=   \sum\limits_{i=1}^n v_i u_i = v_1 u_1+v_2 u_2 + \dots + v_n u_n\in \mathbb R\,.
%\end{eqnarray*}
%\end{theorembox}

%\begin{exercisebox}[Skalarprodukt in $\mathbb R^n$]
%  Ist die folgende Aussage wahr oder falsch? "Das Skalarprodukt ist keine Multiplikation von Vektoren."
%\end{exercisebox}

%\begin{exercisebox}[Kosinussatz]
%	Betrachten Sie ein Dreieck mit den Kantenvektoren $\boldsymbol a, \boldsymbol b$ und Winkel $\alpha$ und zeigen Sie:
%	\[c^2 = a^2 + b^2 - 2\, a\, b \, \cos(\alpha)\,.\]
%\end{exercisebox}
%\begin{exercisebox}[Satz des Thales]
%	Jeder Winkel im Halbkreis ist ein rechter Winkel. 
%\end{exercisebox}
%
%\begin{exercisebox}[Kreuzprodukt] 
%	Zeigen Sie allgemein, dass f�r $\boldsymbol u, \boldsymbol v\in \mathbb R^3$ der Vektor $\boldsymbol v$ senkrecht auf $\boldsymbol u \times \boldsymbol v$ steht. 
%\end{exercisebox}

%\begin{exercisebox}[Skalarprodukt]
%  Skalarpodukt als Projektion.
%\end{exercisebox}

%\begin{exercisebox}[Anwendung Kreuzprodukt]
%	\begin{itemize}
%		\item Mechanik: Drehimpuls 
%		\item Elektrotechnik: Maxwell Gleichungen
%	\end{itemize}
%\end{exercisebox}

%\begin{exercisebox}[ Mercator Projektion]
%\end{exercisebox}
%
%\begin{exercisebox}[Differentialgeometrie]
%	\begin{itemize}
%		\item Gau�sche Kr�mmung, Mittlere Kr�mmung,
%		\item Isometrie
%		\item Isomorphismen (Problem Kugel)
%	\end{itemize}
%\end{exercisebox}
%
%\begin{exercisebox}[Relativit�tstheorie]
%	Kr�mmung der Raumzeit, Ursprung der Gravitationskraft
%\end{exercisebox}
%
%\begin{exercisebox}[$\mathbb R^n$]
%  $V=\mathbb R^n$ ist ein Vektorraum �ber dem K�rper $G = \mathbb R$.
%\end{exercisebox}

%\begin{exercisebox}[Metrik]
%	Vektorraum
%\end{exercisebox}
