\ifdefined\MOODLE
\section*{Mengen und Aussagenlogik Aufgaben}\label{MA1-01-Aufgaben}

Regeln und Beispiele, die f�r die Bearbeitung der Aufgaben hilfreich sind, finden Sie in Abschnitt \ref{MA1-01} und, falls Sie nicht weiterkommen, schauen Sie \hyperref[MA1-01-Aufgaben-Loesungen]{hier}. 

\else
\section{Mengen und Aussagenlogik Aufgaben}\label{MA1-01-Aufgaben}

Regeln und Beispiele, die f�r die Bearbeitung der Aufgaben hilfreich sind, finden Sie im Skrpt und, falls Sie nicht weiterkommen, schauen Sie \hyperref[MA1-01-Aufgaben-Loesungen]{hier}. 
\fi

%\begin{exercisebox}[Brainteaser - Logik]
%	Welcher Tag ist morgen, wenn vorgestern der Tag nach Montag war?
%\end{exercisebox}
%
%\begin{exercisebox}[Brainteaser - Logik]
%	Das Einzelkind Matthias steht vor einem �lgem�lde. Er erkl�rt: Der Vater des Abgebildeten ist der Sohn meines Vaters. Wer ist auf dem Gem�lde zu sehen?
%\end{exercisebox}

%\begin{exercisebox}[Schubfachprinzip]
%	Beweisen Sie: Hat man \(n+1\) Objekte in \(n\) Schubladen verteilt, so gibt es mindestens eine Schublade, in der zwei Objekte liegen.
%\end{exercisebox}

%\begin{exercisebox}[M�chtigkeit von unendlichen Mengen]
%	In einem Hotel mit endlich vielen Zimmern k�nnen keine G�ste mehr aufgenommen werden, sobald alle Zimmer belegt sind (Schubfachprinzip). Hilberts Hotel hat nun unendlich viele Zimmer (durchnummeriert mit nat�rlichen Zahlen bei \(1\) beginnend). Man k�nnte nun annehmen, dass dasselbe Problem auch hier auftreten w�rde, n�mlich dann, wenn alle Zimmer durch (unendlich viele) G�ste belegt sind.
%	Es gibt jedoch einen Weg, Platz f�r einen weiteren Gast zu schaffen. Wie k�nnte das gehen?
%\end{exercisebox}

\begin{exercisebox}[Mengen]
	Geben Sie die Mengen explizit an:
	\begin{itemize}
		\item[(a)] $A = \{ x:\; x^2-3x+2 = 0\}$
		\item[(b)] $B = \{ x:\; x \, \textnormal{ist ein Buchstabe aus dem Wort Leipzig} \}$
		\item[(c)] $C = \{ x:\; x^2= 9 \land  x^3=27\}$
		\item[(d)] $D = \{ x:\; x^2= 9 \lor  x -3 =6\}$
		\item[(e)] $E = \{ x:\; x^2= 9 \land  x -3 =6\}$
	\end{itemize}
\end{exercisebox}

\begin{exercisebox}[Mengenoperationen]
	Wie l�sst sich anhand der beiden Mengen $ \mathbb{M}_1 = \{ 1,\, 2,\, 3,\, 4,\, 5,\, 6\}$ und $ \mathbb{M}_2 = \{4,\, 5,\, 6,\, 7,\, 8,\, 9, 10\}$ zeigen, dass bei einer Differenzmenge $\mathbb{M}_1 \setminus \mathbb{M}_2$ die Mengen $\mathbb{M}_1$ und $\mathbb{M}_2$ im Allgemeinen nicht vertauscht werden k�nnen?
\end{exercisebox}

\begin{exercisebox}[Aussagenlogik]
	Negieren Sie die folgenden Ausdr�cke und vereinfachen Sie die Negation so weit wie m�glich und entscheiden Sie, ob die negierte Aussage wahr oder falsch ist. Tipp: Wenden Sie dann die Regeln zur Negation quantorisierter Ausdr�cke und die De Morganschen Gesetze an. 
	\begin{itemize}
		\item[(a)] Alle gut vorbereiteten Studierenden bestehen die Mathematikklausur.
		\item[(b)] Es gibt eine reelle Zahl, die gerade und Teiler von $27$ ist.
		\item[(c)] Alle Zahlen, deren Quersumme durch $3$ teilbar ist, sind durch $3$ teilbar. 
	\end{itemize}
\end{exercisebox}

\begin{exercisebox}[Mengen und Aussagenlogik]
	Seien $A$ und $M$ Mengen. Unter welchen Bedingungen an die Elemente von \(A\) und \(M\) ist $A\not\subseteq M$ eine wahre Aussage?  {Hinweis:} \( A \not \subseteq M \Leftrightarrow \neg\left( A\subset M\right)\).
\end{exercisebox}

\begin{exercisebox}[Aussagenlogik]
Negieren Sie die folgenden Aussagen! Schreiben Sie dabei die Aussagen und ihre Negationen auch mit Quantoren und geben Sie an, ob die Aussage oder ihre Negation wahr ist! 
\begin{itemize}
\item[(a)] Jede nat�rliche Zahl hat einen Vorg�nger.
\item[(b)] Jede nat�rliche Zahl hat einen Nachfolger.
\item[(c)] Es gibt keine reelle Zahl, die zugleich positiv und negativ ist.
\item[(d)] Es gibt keine reelle Zahl, die weder positiv noch negativ ist.
\item[(e)] Jede nichtnegative reelle Zahl ist positiv.
\item[(f)] Die Gleichung $x^2+2x+3=0$ hat eine reelle L�sung.
\item[(g)] Jedes Viereck, das zugleich Rechteck und Drachenviereck ist, ist ein Quadrat.
%\item Jeder Student ist bei der Vorlesung anwesend.
%\item Es gibt einen Studenten, der nicht im Wohnheim wohnt.
\end{itemize}
\end{exercisebox}


\begin{exercisebox}[Brainteaser]
	An einer Weggabelung wohnen zwei Br�der, von denen einer die Angewohnheit hat, stets die Wahrheit zu sagen, w�hrend der andere immer l�gt. Ein Wanderer, der vorbeikommt, m�chte sich nach dem richtigen Weg erkundigen. Er hat von den eigent�mlichen Geschwistern geh�rt, wei� aber nicht, welcher von beiden die Wahrheit sagt. Wie kann er mit nur einer Frage an einen der M�nner herausfinden, welchen Weg er nehmen muss?
\end{exercisebox}
