\ifdefined\MOODLE 

\section*{Zahlenmengen Aufgaben}\label{MA1-02-Aufgaben}

Regeln und Beispiele, die zur Bearbeitung der Aufgaben hilfreich sind, finden Sie in Abschnitt \ref{MA1-02} und, falls Sie nicht weiterkommen, schauen Sie \hyperref[MA1-02-Aufgaben-Loesungen]{hier}.
\else 
\section{Zahlenmengen Aufgaben}\label{MA1-02-Aufgaben}

Regeln und Beispiele, die zur Bearbeitung der Aufgaben hilfreich sind, finden Sie im Skript und, falls Sie nicht weiterkommen, schauen Sie \hyperref[MA1-02-Aufgaben-Loesungen]{hier}.
\fi 

\begin{exercisebox}[K�rperaxiome]
Welche Rechenregeln erf�llen die ganzen Zahlen $\mathbb Z$ nicht im Vergleich zu $\mathbb R$?
\end{exercisebox}


\begin{exercisebox}[Summen- und Produktzeichen]% Definition Summe und Produkt. Rechenregeln.
	Schreiben Sie die folgenden Summen aus und berechnen Sie ihren Wert:
	\begin{equation}
		\begin{array} { lllll }
			(a) \; \displaystyle{\sum\limits_{i=1}^{7} i} & \quad  \quad (b)\; \displaystyle{\prod\limits_{i=1}^{2}} i &\quad \quad (c) \;\displaystyle{\prod\limits_{i=0}^{5}} i &\quad \quad (d)\; \displaystyle{\sum\limits_{i=1}^{5} i^2} &\quad\quad (e)\; \displaystyle{\sum\limits_{i=-3}^{3} i^2} \\
			(f)\; \displaystyle{\prod\limits_{i=-3}^{3}} i^2 &\quad \quad (g)\; \displaystyle{\sum\limits_{i=0}^{7} 5} &\quad \quad (h)\; \displaystyle{\prod\limits_{i=0}^{7}} 2 &\quad \quad (i)\; \displaystyle{\sum\limits_{i=3}^{9} (i-2)} &\quad \quad (j)\; \displaystyle{\prod\limits_{i=-1}^{3}} (i+2)
	\end{array}
\end{equation}
\end{exercisebox}

\begin{exercisebox}[Summenzeichen Indexverschiebung]
	Wie lautet das Argument f�r folgende Indexverschiebung?
	$\displaystyle{\sum\limits_{i=k}^l x_i} = \displaystyle{\sum\limits_{i=k+a}^{l+a}} \textnormal{???} $
\end{exercisebox}

\begin{exercisebox}[Summen- und Produktzeichen]
	Ersetzen Sie in den folgenden Gleichungen die Fragezeichen: % Definition Summe und Produkt. Rechenregeln.
	\begin{eqnarray*}
		\begin{array}{ll}
			(a) \;\; 1+x+\displaystyle{\frac{x^2}{2!}}+\displaystyle{\frac{x^3}{3!}}+\dots\displaystyle{\frac{x^n}{n!}} = \displaystyle{\sum\limits_{i=1}^{\textnormal{?}} \textnormal{?}}  = \displaystyle{\sum\limits_{i=-2}^{\textnormal{?}} \textnormal{?}}  &\quad \quad
			(b)\;\; \displaystyle{\sum\limits_{i=0}^{n}} x^{0\cdot i} = \textnormal{?} \\
			(c)\;\; 1+x+\displaystyle{\frac{x^2}{2!}}+\displaystyle{\frac{x^3}{3!}}+\dots\displaystyle{\frac{x^n}{n!}} =\textnormal{?} \cdot\displaystyle{\sum\limits_{i=0}^{\textnormal{?}}} \frac{x^{i-1}}{i!} &  \quad \quad  (d)\;\; \displaystyle{\frac{\displaystyle{\prod\limits_{i=1}^{n}} 4^i}{\displaystyle{\prod\limits_{i=2}^{n+1}} 2^i}} = \textnormal{?}
	\end{array}
	\end{eqnarray*}
\end{exercisebox}

\begin{exercisebox}[Binomialkoeffizient]
	Zeigen Sie
	$$\binom{n+1}{k} = \binom{n}{k}+ \binom{n}{k-1}, \; k\geq 1\,.$$
\end{exercisebox}

\begin{exercisebox}[Binomischer Satz]
	Bestimmen Sie mithilfe des Binomischen Satzes eine Formel f�r $(a-b)^n$.
\end{exercisebox}

\begin{exercisebox}[Binomischer Satz]
	Zeigen Sie
	$$\sum\limits_{k=0}^n \binom{n}{k} = 2^n$$
\end{exercisebox}

%\begin{exercisebox}[Rechenspiele]
%	Sie sollen die Zahlen \(21\) aus den Zahlen \(1,5,6\) und \(7\) errechnen. Alle diese Zahlen m�ssen in der Gleichung genau einmal vorkommen. Sie d�rfen nur die vier Grundrechenarten \((+,-,\cdot,:)\) sowie Klammern verwenden. Rechnen Sie nur mit ganzen Zahlen. Also zum Beispiel \( 1+5+6+7 = 19\) oder \(5\cdot(7-6+1) = 10\). 
%\end{exercisebox}
%
%\begin{exercisebox}[Vorstellungsgespr�ch]
%	In einem Vorstellungsgespr�ch bietet Ihnen der Personalchef folgendes Spiel an: "Wir addieren ausgehend von der Zahl zehn immer abwechselnd eine Zahl zwischen eins und zehn auf die neue Summe. Wer am Ende auf genau \(100\) addiert hat gewonnen. Ich fange an, um Ihnen das zu demonstrieren: Also \(10+2=12\). Das ist mein erster Zug. Nun ist es an Ihnen." K�nnen Sie das Spiel gewinnen? 
%\end{exercisebox}


%\begin{exercisebox}[Binomialkoeffizient]
%	Auf einer internationalen Konferenz treffen sich Vertreter mit $10$ unterschiedlichen Landessprachen. Wie viele Dolmetscher werden ben�tigt, wenn f�r jede �bersetzung zwischen zwei Sprachen ein eigener Dolmetscher anwesend sein soll?
%\end{exercisebox}

%\begin{theorembox}[Rundungsregeln]
%	\begin{itemize}
%		\item Ist die erste weggelassene Ziffer $0,1,2,3$ oder $4$, so bleibt die letzte geschriebene Ziffer unver�ndert (Abrunden).
%		\item Ist die erste weggelassene Ziffer $5,6,7,8$ oder $9$, so wird die letzte geschriebene Ziffer um eins erh�ht (Aufrunden).
%	\end{itemize}
%\end{theorembox}

%\begin{exercisebox}[Terme vereinfachen]
%	Vereinfachen Sie so weit wie m�glich:
%	%\begin{itemize}
%		%\item $a(c+b)-c(b+a)+b(c-a)$
%		%\item $a(c+b(a-c))-c(b(a-1)+a)-b(c-a(b-a))$
%		%\item $(2b-3a)(3a-2b)-(2a-b)^2$
%		%\item $\displaystyle{\frac{x+3}{x^2-4} + \frac{x-3}{x-2}}$
%		%\item $\left( \displaystyle{\frac{a^3x^5}{a^{-29x^3}}}\right)^4$ % Potenzregeln
%	%\end{itemize}
%\end{exercisebox}
%\begin{theorembox}[Binomische Formeln]
%	F�r alle $a,b$ gilt:
%	\begin{eqnarray*}
%		(a+b)^2 &=& a^2+2ab+b^2\\
%		(a-b)^2 &=& a^2-2ab+b^2\\
%		(a+b)(a-b) &=& a^2-b^2
%	\end{eqnarray*}
%\end{theorembox}

%\begin{theorembox}[Satz von Vieta]
%	Es gelte $\quad x^2+ax+b= (x+p)(x+q)$ f�r alle $x\in \mathbb R$. Welche Beziehungen m�ssen dann zwischen $a, b$ und $p,q$ gelten?
%	$\forall x\in \mathbb R: \quad x^2+ax+b= (x+p)(x+q)\; \Leftrightarrow \; a=p+q \land b=pq \,.$
%\end{theorembox}
