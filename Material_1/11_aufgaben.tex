
\ifdefined\MOODLE 
\section*{Rationale Funktionen Aufgaben}\label{MA1-11-Aufgaben}

	Regeln und Beispiele, die zur Bearbeitung der Aufgaben hilfreich sind, finden Sie in Abschnitt \ref{MA1-11} und, falls Sie nicht weiterkommen, schauen Sie \hyperref[MA1-11-Aufgaben-Loesungen]{hier}.
\else 
\section{Rationale Funktionen Aufgaben}\label{MA1-11-Aufgaben}

	Regeln und Beispiele, die zur Bearbeitung der Aufgaben hilfreich sind, finden Sie im Skript und, falls Sie nicht weiterkommen, schauen Sie \hyperref[MA1-11-Aufgaben-Loesungen]{hier}.
\fi 


\begin{exercisebox}[Abbildungsvorschrift] %LS p186 -Aufgabe 4
	Geben Sie eine gebrochenrationale Funktion an mit 
	\begin{itemize}
		\item[(a)] Nullstelle $1$
		\item[(b)] Polstelle $3$ mit Vorzeichenwechsel
		\item[(c)] Polstelle $3$ ohne Vorzeichenwechsel
		\item[(d)] Nullstelle $1$ und Polstelle $3$ ohne Vorzeichenwechsel
		\item[(e)] Nullstellen $2$ und $3$, Polstelle $4$ mit Vorzeichenwechsel
		\item[(f)] Nullstelle $-1$, Polstelle $-3$ mit Vorzeichenwechsel, Polstelle $4$ ohne Vorzeichenwechsel
	\end{itemize}
\end{exercisebox}

\begin{exercisebox}[Asymptoten f�r \(|x|\to \infty\)] %p.31 Aufgabe 3  und b, c unter http://mathenexus.zum.de/html/analysis/funktionen_gebrochenrationale/weiterfuehrendes/gebro_04_Asys.htm  
		Untersuchen Sie das Verhalten der Funktion f�r $x\to \pm\infty$. Geben Sie gegebenenfalls die Gleichung der Asymptoten an.
	\begin{eqnarray*} \begin{array}{lll}
	(a)\; f(x)= \dfrac{7}{x} &\quad 
     %(b)\; f(x)= \dfrac{5}{3x-1} &\quad 
     %(c)\; f(x)= \dfrac{2}{x-2}-3 \\
     (b)\; f(x)= \dfrac{-3x^3+4x+16}{4x^2} &\quad 
     (c)\; f(x)= \dfrac{x^3+1}{x-1} \\
     (d)\; f(x)= \dfrac{2}{x}+\sqrt{x} &\quad 
     (e)\; f(x)= \dfrac{2}{(x-1)^2} &\quad 
     (f)\; f(x)= \dfrac{4}{\sqrt{x-2}} 
     \end{array}
     \end{eqnarray*}
    %\item $f(x)= 2- \frac{3}{\sqrt{x+3}}$
    %\item $f(x)= \frac{1}{x}+2\sin(x)$
\end{exercisebox}


%\begin{exercisebox}[Wolfram Alpha] %p31, Aufgabe 5 -- Pingo
%  Wie kann man mithilfe von Wolfram Alpha zu Vermutungen �ber den Grenzwert f�r $x\to\pm \infty$ gelangen? Geben Sie f�r $f$ solche Vermutungen an.
%  \begin{enumerate}
%    \item $f(x)= \frac{x}{2^x}$
%    \item $f(x)= \frac{x^2}{2^x}$
%    \item $f(x)= \frac{x^3}{1,5^x}$
%    \item $f(x)= \frac{x+2}{x^2-2}$
%    \item $f(x)= \frac{2x-1}{x+1}$
%    \item $f(x)= \frac{10\sin(x)}{x}$
%    \item $f(x)= 2^{-x} \cdot \sin(x)$
%    \item $f(x)= x^{\frac{1}{x}}, \; x>0$
%  \end{enumerate}
%\end{exercisebox}

\begin{exercisebox}[Vertikale Asymptoten] %p33, Aufgabe 6
  Untersuchen Sie das Verhalten von $f$ bei Ann�herung an die Definitionsl�cke. Geben Sie die Geichung der senkrechten Asymptote an.
	\begin{eqnarray*} \begin{array}{lll}
			(a)\; f(x) = \dfrac{2}{x} & \quad 
			(b)\; f(x) = -\dfrac{2}{x^2} &\quad 
			(c)\; f(x) = \dfrac{1}{x-4} \\[1ex]
			(d) \; f(x) = \dfrac{2}{4-x} &\quad 
     (e)\; f(x) = 1- \dfrac{1}{x} &\quad 
     (f) \; f(x) = \dfrac{3}{(x-1)^2} 
     \end{array}
     \end{eqnarray*}
\end{exercisebox}

\begin{exercisebox}[Asymptoten] % LS p. 189, Aufgabe 2
  Geben Sie die Gleichungen aller Asymptoten an
	\begin{eqnarray*} \begin{array}{lll}
			(a)\; f(x) = \dfrac{4}{3x^2} \quad \quad  & %a) 
	  (b)\; f(x) = \dfrac{2x+1}{x^2+3x} \quad\quad\quad & % f) 
	  (c)\; f(x) = \dfrac{x^4-x^2-1}{x^3-1} % p)
     \end{array}
  \end{eqnarray*}
\end{exercisebox}


\begin{exercisebox}[Qualitativer Graph] %LS p. 186 Aufgabe 2 a)-f)
	Ermitteln Sie von den folgenden Funktionen die 
	\begin{itemize}
		\item die Definitionsmenge,
		\item die Achsenabschnitte,
		\item die Nullstellen,
		\item die Polstellen inklusive der Analyse des Verhaltens von $f$ an jeder Polstelle und 
		\item fertigen Sie mit diesen Informationen eine qualitative Skizze des jeweiligen Funktionsgraphen an. 
	\end{itemize}
	\begin{eqnarray*} \begin{array}{lll}
			(a)\; f(x) = \frac{3x-1}{x-1} \quad & 
			(b)\; f(x) = \frac{x^2+x}{x+1} \quad & 
			(c)\; f(x) = \frac{x^2-9}{(x-3)^2} 
			%\\[1ex]
			%(d)\; f(x) = \dfrac{x^2-2x-15}{x-5} \quad &
     %(e)\; f(x) = \dfrac{3x-3}{x-1} \quad &
     %(f)\; f(x) = \dfrac{x^2+5x+2}{(x+1)^2} 
     \end{array}
  \end{eqnarray*}
\end{exercisebox}

%\begin{exercisebox}[Vermischte Aufgaben] -- Aufgabensammlung p. 38, Aufg. 8
%  Gegeben ist die Funktion $f$ mit $f(x) = \frac{1}{2(x+2)}+1$.
%  \begin{enumerate}
%    \item Bestimmen Sie die maximale Definitionsmenge von $f$.
%    \item Untersuchen Sie den Graphen von $f$ auf waagerechte und senkrechte Asymptoten.
%    \item Wo schneidet der Graph die $x$-Achse?
%    \item Skizzieren Sie den Graphen von $f$ mit seinen Asymptoten.
%    \item Entnehmen Sie der Sikzze eine Vermutung �ber die Symmetrie des Graphen. Beweisen Sie die Vermutung rechnerisch.
%  \end{enumerate}
%\end{exercisebox}

%\begin{exercisebox}[Asymptoten horizontal und vertikal] --> Aufgabensammlung p. 33, Aufg. 9
%  Gegeben ist die Funktion $f$ mit $f(x) = \frac{1}{(x+1)(x-2)}$
%  \begin{enumerate}
%    \item Geben Sie den maximalen Definitionsbereich der Funktion $f$ an.
%    \item Untersuchen Sie das Verhalten von $f$ bei Ann�herung an die Definitionsl�cken.
%    \item Untersuche Sie das Verhalten von $f$ f�r $x\to \pm \infty$
%    \item Skizzieren Sie den Graphen von $f$ mithilfe der Asymptoten.
%  \end{enumerate}
%\end{exercisebox}
