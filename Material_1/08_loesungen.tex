\section*{Polynome L�sungen}\label{MA1-08-Aufgaben-Loesungen}

\ifdefined\MOODLE 
	Regeln und Beispiele, die zum Verst�ndnis der L�sungswege hilfreich sind, finden Sie in Abschnitt \ref{MA1-08}.
\else
	Regeln und Beispiele, die zum Verst�ndnis der L�sungswege hilfreich sind, finden Sie im Skript.
\fi 

\begin{exercisebox}[Symmetrie] %LS
	Zeigen Sie: Wenn der Funktionsterm einer ganzrationalen Funktion einen konstanten Summanden enth�lt, dann ist die Funktion nicht ungerade. 

  \hspace{0.3cm}
  \newline
  {\bf L�sung:}
  Es ist \[ f(x) = a_n x^n + a_{n-1} x^{n-1} + \dots + a_1 x + a_0, \quad a_n\not = 0\,.\] F�r eine ungerade Funktion gilt insbesondere $f(0) = 0$. Dies ist hier nicht erf�llt, da $f(0) = a_0 \not=0$. Also kann $f$ nicht ungerade sein. $f$ kann allerdings gerade sein, wenn $n$ gerade ist und $a_{n-1} = a_{n-3} = \dots = a_1 = 0$.
\end{exercisebox}

\begin{exercisebox}[Umkehrfunktion]
      Die Funktion $f$ mit $f(x)=ax^3+bx,\; a,b\in \mathbb R,$ sei umkehrbar. Welche Bedingung erf�llen die Koeffizienten $a$ und $b$?

  \hspace{0.3cm}
  {\bf L�sung:} 
  \begin{itemize}
	  \item F�r \(a\not=0\) gilt \( f(x) = ax^3+bx = a\cdot x\cdot (x^2+b/a)\), das hei�t, dass der Funktionsgraph in \(x=0\) die \(x\)-Achse schneidet. \(f\) ist bijektiv, wenn es bei der einen reellen Nullstelle bleibt, wenn also \(b/a>0\) gilt. Dies ist genau dann der Fall, wenn $a$ und $b$ haben gleiches Vorzeichen haben. 
	  \item F�r \(a = 0\) gilt \(f(x) = bx\) ist invertierbar, falls \(b\not=0\).
	  \item F�r \(b = 0\) gilt \(f(x) = ax^3\) ist invertierbar, falls \(a\not=0\).
  \end{itemize}
\end{exercisebox}

\begin{exercisebox}[Polynom?] %LS p.23 Aufg. 3
	Entscheiden Sie, welche der folgenden Funktionen Polynome sind:
	\begin{eqnarray*} \begin{array}{lll}
			(a)\; x\mapsto 1+\sqrt{2}x & \quad (b)\; x\mapsto 1+2\sqrt{x} &\quad (c)\; x\mapsto (x-1)^2(x-7) \\[1ex]
			(d)\; x\mapsto x^2-\frac3x & \quad (e)\; x\mapsto x^2-\frac{x}{3} &\quad (f)\; x\mapsto  x^2+\sin(x)
		\end{array}
	\end{eqnarray*}


	\vspace{0.3cm}
	\noindent{\bf L�sung:}\newline
	\begin{itemize}
		\item[(a)] $f$ ist ganzrationoal mit $a_0=1$, $a_1=\sqrt{2}$ und hat den Grad $1$.
		\item[(b)] $f$ ist nicht ganzrational.
		\item[(c)] $f(x) = x^3-9x^2+15x-7$. Das hei�t $f$ ist ganzrationoal mit $a_0=-7$, $a_1=15$, $a_2=-9$, $a_3=1$ und hat den Grad $3$.
		\item[(d)] $f$ ist nicht ganzrational.
		\item[(e)] $f$ ist ganzrationoal mit $a_0=1$, $a_1=-\displaystyle{\frac13}$, $a_2=1$ und hat den Grad $2$.
		\item[(f)] $f$ ist nicht ganzrational.
	\end{itemize}
\end{exercisebox}

\begin{exercisebox}[Dartellungen von Parabeln] %Andreas
	Bestimmen Sie von folgenden Parabeln die Linearfaktorzerlegung und die Scheitelpunktform.	
  \begin{eqnarray*} \begin{array}{ll} 
		  (a)\; x\mapsto -2x^2 -4x+3 \quad \quad \quad  \quad & (b)\; x\mapsto 5x^2+20x+20 \\[1ex]
		  (c)\; x\mapsto 2x^2+10x  \quad \quad \quad   \quad &(d)\; x\mapsto 4x^2+8x-60
	  \end{array}
  \end{eqnarray*}

  \hspace{0.3cm}
  \newline
  {\bf L�sung:}
  \begin{itemize}
	  \item[(a)] Nullstellen \[ x_{1,2} = \frac{-(-4) \pm \sqrt{(-4)^2 - 4\cdot (-2)\cdot 3}}{ 2\cdot (-2)} = -1 \pm \frac{\sqrt{10}}{2} \] Damit \[ y(x) = -2\left(x+1-\frac{\sqrt{10}}0{2} \right) \left( x+1+\frac{\sqrt{10}}{2}\right) \] Scheitelpunkt \[ x_S=\frac{-(-4)}{2\cdot (-2)} = -1. \; y_S = y(-1) = 5\,.\] Und damit \[ y(x) = -2(x+1)^2 + 5 \,.\]
	  \item[(b)] $y(x) = 5(x+1)(x+2) = 5(x+2)^2 $
	  \item[(c)] $y(x) = 2x(x+5) = 2(x+\frac52)^2-\frac{25}{2}$
		  \item[(c)] $y(x) = 4(x+5)(x-3) = 4(x+1)^2-64\,.$
  \end{itemize}
\end{exercisebox}

\begin{exercisebox}[Abbildungsvorschrift] %Andreas
	\begin{itemize}
		\item[(a)] Wie m�ssen die Koeffizienten $a,b,c$ lauten, wenn die Parabel $y(x) = ax^2+bx+c$ an den Stellen $x_1 = 1$ und $x_2=-5$ verschwindet und der Funktionswert am Scheitelpunkt $y_S = 18$ betr�gt?
		\item[(b)] Bestimmen Sie ein Polynom $p$ vierten Grades mit den folgenden Eigenschaften: 
			\begin{itemize}
				\item $p$ ist eine gerade Funktion.
				\item $p$ besitzt die Nullstellen $x_1=3$ und $x_2=6$.
				\item $p(0)=-3$
			\end{itemize}
	\end{itemize}

  \hspace{0.3cm}
  \newline
  {\bf L�sung:}
  \begin{itemize}
	  \item[(a)] Ansatz \[ y(x) = a\cdot (x-1)(x+5)\]
  Der Scheitelpunkt ist durch $x_S=(x_1 + x_2)/2 = (1+(-5)) /2 = -2$ gegeben: \[ y(x_S) = a\cdot (-2-1)\cdot(-2+5) = -9a\,.\] Es muss also $-9\cdot a = 18$ sein und damit $a=-2$. Eingesetzt und ausmultipliziert folgt \[ y(x) = -2\cdot (x-1)(x+5) = -2x^2 - 8x+10\,.\]
  \item[(b)] Da das Polynom gerade ist, sind auch $x_3=-3$ und $x_4=-6$ Nullstellen. Damit hat das Polynom die Form \[ p(x) = a\cdot (x+3)(x-3)(x+6)(x-6), \quad \mathrm{mit}\; a\in \mathbb R\,.\]
	  Es gilt \[ p(0) = a\cdot 3\cdot (-3) \cdot 6 \cdot (-6) = 324 a \] und wegen $p(0) = -3 $ gilt \[ a=\frac{-3}{423} = -\frac{1}{108}\,.\] Also gilt \[ p(x) = -\frac{1}{108} \cdot (x+3)(x-3)(x+6)(x-6)\,.\]
	  \end{itemize}
\end{exercisebox}

\begin{exercisebox}[Polynomdivision]
	Zeigen Sie, dass $x_0 = -5$ eine Nullstelle von $p(x) = 3x^3 + 18 x^2 + 9x -30$ ist. Berechnen Sie mit Hilfe der Polynomdivision das Polynom \(q\), f�r das gilt $p(x) = (x-x_0)\cdot q(x)$.

  \hspace{0.3cm}
  \newline
  {\bf L�sung:}
  Es gilt $p(-5) = 3(-5)^3 + 18 (-5)^1 + 9(-5) -30$. Die Polynomdivision ergibt \[ p(x) = (x+5)\cdot (3x^2+3x-6)\,.\]
\end{exercisebox}

\begin{exercisebox}[Polynomdivision]
	Bestimmen Sie alle Nullstellen der folgenden Polynome mit Hilfe einer Polynomdivision und stellen Sie die Polynome in Linearfaktorzerlegung dar. Die Polynome besitzen mindestens eine ganzzahlige Nullstelle. 
  \begin{eqnarray*} \begin{array}{ll} 
		  (a)\; h: x\mapsto x^3-2x^2-5x+6 \quad & (b)\; f: t\mapsto -2t^4-2t^3-4t+8 \\[1ex]
		  (c)\; g: x\mapsto x^4-x^3-x^2-x-2 \quad &(d)\; p: t\mapsto 2t^4+8t^3-12t^2-8t+10
	  \end{array}
  \end{eqnarray*}

  \hspace{0.3cm}
  \newline
  {\bf L�sung:}
  \begin{itemize}
	  \item[(a)] $h(x) = (x-3)(x-1)(x+2)$
	  \item[(b)] $f(t) = -2(t-1)(t+2)(t^2+2)$
	  \item[(c)] $g(x) = (x+1)(x-2)(x^2+1)$
	  \item[(d)] $p(t) = 2(t-1)^2(t+1)(t+5)$
  \end{itemize}
\end{exercisebox}

%\begin{exercisebox}[BMI]
%	F�r das Idealgewicht sind zwei g�ngige Formeln in Gebrauch: \begin{center} Idealgewicht in $\mathrm{kg}$ $=$ K�rpergr��e in $\mathrm{cm}$ minus $100$ \end{center}
%	und der Body-Mass-Index, \begin{center} BMI $=$ $\displaystyle{\frac{\textnormal{Gewicht in }\mathrm{kg}}{\left( \textnormal{Koerpergroesse in }\mathrm{m} \right)^2}}\,,$ \end{center} 
%	sollte zwischen $20$ und $25$ liegen.
%	\begin{itemize}
%		\item[(a)] Zeichnen Sie ein Diagramm des Gewichts in Abh�ngigkeit der K�rpergr��e f�r die erste Formel und f�r BMI-Werte von jeweils $20$ und $25$. 
%		\item[(b)] F�r welche K�erpergr��e besitzt das Idealgewicht nach der ersten Formel einen BMI zwischen $20$ und $25$?
%	\end{itemize}
%\end{exercisebox}

%\begin{exercisebox}[Windkraftanlagen]
%	Nehmen Sie an ein Luftmolek�l trifft mit der kinetischen Energie \[ E_{L} = \frac12 m_L v^2\] auf die Rotorfl�chen eines Windkraftwerks. 
%	�berlegen Sie sich Formel, um hieraus n�herungsweise die Leistung $[\frac{\mathrm{N}}{{\mathrm{s}}}]$ eines Windkraftwerks in Abh�ngigkeit der Windgeschwindigkeit $v$ anzugeben.  
%\end{exercisebox}
	
